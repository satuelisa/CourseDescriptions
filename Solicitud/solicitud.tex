\documentclass{article}
\usepackage[spanish]{babel}
\selectlanguage{spanish}
\usepackage[utf8]{inputenc}
\usepackage{hyperref}
\usepackage{xcolor}
\usepackage{multicol}
\usepackage{enumitem}
\usepackage[top=1cm,left=20mm,right=25mm,bottom=1cm]{geometry}


\makeatletter
\def\@maketitle{%
  \newpage
  \null
  \begin{center}%
  \let \footnote \thanks
  {\Large \@title \par}%
  {\large \@author \par}%
  {\sc \@date \par}
  \end{center}%
  \vskip 1mm}
\makeatother

\usepackage[affil-it]{authblk}
\setlength{\parindent}{0em}
\setlength{\parskip}{4pt}
\renewcommand{\baselinestretch}{1.4}
\pagestyle{empty}

\title{\Large Solicitud de admisión al Posgrado}
  
\author{División de Posgrado en Ingeniería de Sistemas}\affil{Facultad de Ingeniería Mecánica y
  Eléctrica}
\date{Universidad Autónoma de Nuevo León}

\begin{document}
\maketitle
\thispagestyle{empty}

{\em Favor de llenar directamente en computadora con Adobe Acrobat Reader o similar.}

\begin{Form}
  {Inicio de estudios deseado:}
  \ChoiceMenu[radio,radiosymbol=\ding{52},print,name=sem]{}{Otoño}
  \ChoiceMenu[radio,radiosymbol=\ding{52},print,name=sem]{}{Primavera}
  \TextField[name=year,multiline=false,bordercolor=black,align=0,width=2cm,height=1.2em]{Año}

  {Programa:}
  \ChoiceMenu[radio,radiosymbol=\ding{52},print,name=pe]{}{Maestría}
  \ChoiceMenu[radio,radiosymbol=\ding{52},print,name=pe]{}{Doctorado}

 \TextField[name=nom,multiline=false,bordercolor=black,align=0,width=5cm,height=1.2em]{Nombre(s)}
 \TextField[name=ape,multiline=false,bordercolor=black,align=0,width=7cm,height=1.2em]{Apellido(s)} \\
 \TextField[name=ape,multiline=false,bordercolor=black,align=0,width=7cm,height=1.2em]{Pronombres preferidos}

 
 \TextField[name=nac,multiline=false,bordercolor=black,align=0,width=7cm,height=1.2em]{Nacionalidad(es)}

 Fecha de nacimiento:
 \TextField[name=fn,multiline=false,bordercolor=black,align=0,width=1cm,height=1.2em]{Día}
 \TextField[name=fn,multiline=false,bordercolor=black,align=0,width=1cm,height=1.2em]{Mes}
 \TextField[name=fn,multiline=false,bordercolor=black,align=0,width=1cm,height=1.2em]{Año}

 \TextField[name=curp,multiline=false,bordercolor=black,align=0,width=5cm,height=1.2em]{CURP}
 \TextField[name=rfc,multiline=false,bordercolor=black,align=0,width=4cm,height=1.2em]{RFC}

  {Estado civil:}
  \ChoiceMenu[radio,radiosymbol=\ding{52},print,name=civil]{}{Soltero}
  \ChoiceMenu[radio,radiosymbol=\ding{52},print,name=civil]{}{Casado}
  \ChoiceMenu[radio,radiosymbol=\ding{52},print,name=civil]{}{Divorciado}
  \ChoiceMenu[radio,radiosymbol=\ding{52},print,name=civil]{}{Viudo}
  \ChoiceMenu[radio,radiosymbol=\ding{52},print,name=civil]{}{Otro}

  {Domicilio actual} \\ \\
  \TextField[name=act,multiline=true,bordercolor=black,align=0,width=12cm,height=6em]{}

  {Domicilio permanente (si diferente del actual)} \\  \\
  \TextField[name=perm,multiline=true,bordercolor=black,align=0,width=12cm,height=6em]{}
  
  {\em Teléfonos:}

  \TextField[name=casa,multiline=false,bordercolor=black,align=0,width=4cm,height=1.2em]{Casa}
  \TextField[name=oficina,multiline=false,bordercolor=black,align=0,width=4cm,height=1.2em]{Oficina}
  \TextField[name=cel,multiline=false,bordercolor=black,align=0,width=4cm,height=1.2em]{Celular}

  {\em Correo electrónico:}
  \TextField[name=email,multiline=false,bordercolor=black,align=0,width=4cm,height=1.2em]{Principal}
  \TextField[name=email,multiline=false,bordercolor=black,align=0,width=4cm,height=1.2em]{Auxiliar}

{\em Sitios web:}
  
  \TextField[name=website,multiline=false,bordercolor=black,align=0,width=4cm,height=1.2em]{Página
    personal}

  \TextField[name=repo,multiline=false,bordercolor=black,align=0,width=4cm,height=1.2em]{Repositorio
    GitHub o similar (público)}

\begin{flushright}
  {\em Favor de colocar NA en aquellos campos que no apliquen en su caso..}
\end{flushright}
  
\newpage

\begin{flushright}
  {\em Favor de colocar NA en aquellos campos que no apliquen en su caso..}
\end{flushright}
  
{\bf Fecha de presentación exámenes de admisión a posgrado de la UANL} (última o programada):


  \TextField[name=exani,multiline=false,bordercolor=black,align=0,width=4cm,height=1.2em]{EXANI
  (conocimientos generales)}
  \TextField[name=exci,multiline=false,bordercolor=black,align=0,width=4cm,height=1.2em]{EXCI
  (inglés)}

{\bf Beca anterior de CONACyT:}

\TextField[name=bg,multiline=false,bordercolor=black,align=0,width=4cm,height=1.2em]{Grado
  obtenido} \\
\TextField[name=fb,multiline=false,bordercolor=black,align=0,width=4cm,height=1.2em]{Fecha
  de obtención de grado} \\
\TextField[name=nb,multiline=false,bordercolor=black,align=0,width=4cm,height=1.2em]{Número de beca}

{\bf Grados académicos anteriores de estudios superiores}: \\
Indicar escuela, nombre del programa, periodo (desde-hasta mes/año), grado obtenido y
promedio.
\\ \\ \TextField[name=prev,multiline=true,bordercolor=black,align=0,width=\textwidth,height=10em]{}

{\bf Motivos de postulación}: 

A un posgrado en general: \\ \\
\TextField[name=gen,multiline=true,bordercolor=black,align=0,width=\textwidth,height=15em]{}

A este posgrado en particular: \\ \\
\TextField[name=part,multiline=true,bordercolor=black,align=0,width=\textwidth,height=15em]{}

\newpage

{\bf Conocimientos de programación:}
\begin{itemize}[itemsep=-2pt]
\item[\raisebox{-4pt}{{\ChoiceMenu[radio,radiosymbol=\ding{52},name=progra]{}{=1}}}]{Ninguno}
\item[\raisebox{-4pt}{{\ChoiceMenu[radio,radiosymbol=\ding{52},name=progra]{}{=2}}}]{Principiante}
\item[\raisebox{-4pt}{{\ChoiceMenu[radio,radiosymbol=\ding{52},name=progra]{}{=3}}}]{Intermedio}
\item[\raisebox{-4pt}{{\ChoiceMenu[radio,radiosymbol=\ding{52},name=progra]{}{=4}}}]{Avanzado}
\end{itemize}

Experiencia con lenguajes de programación: \\ \\
\TextField[name=lenguajes,multiline=true,bordercolor=black,align=0,width=\textwidth,height=15em]{}

{\bf Conocimientos de inglés:}

\begin{multicols}{3}

Lectura:
\begin{itemize}[itemsep=-2pt]
\item[\raisebox{-4pt}{{\ChoiceMenu[radio,radiosymbol=\ding{52},name=il]{}{=1}}}]{Ninguno}
\item[\raisebox{-4pt}{{\ChoiceMenu[radio,radiosymbol=\ding{52},name=il]{}{=2}}}]{Principiante}
\item[\raisebox{-4pt}{{\ChoiceMenu[radio,radiosymbol=\ding{52},name=il]{}{=3}}}]{Intermedio}
\item[\raisebox{-4pt}{{\ChoiceMenu[radio,radiosymbol=\ding{52},name=il]{}{=4}}}]{Avanzado}
\end{itemize}

\columnbreak

Redacción:
\begin{itemize}[itemsep=-2pt]
\item[\raisebox{-4pt}{{\ChoiceMenu[radio,radiosymbol=\ding{52},name=ir]{}{=1}}}]{Ninguno}
\item[\raisebox{-4pt}{{\ChoiceMenu[radio,radiosymbol=\ding{52},name=ir]{}{=2}}}]{Principiante}
\item[\raisebox{-4pt}{{\ChoiceMenu[radio,radiosymbol=\ding{52},name=ir]{}{=3}}}]{Intermedio}
\item[\raisebox{-4pt}{{\ChoiceMenu[radio,radiosymbol=\ding{52},name=ir]{}{=4}}}]{Avanzado}
\end{itemize}

\columnbreak

Conversación:
\begin{itemize}[itemsep=-2pt]
\item[\raisebox{-4pt}{{\ChoiceMenu[radio,radiosymbol=\ding{52},name=ic]{}{=1}}}]{Ninguno}
\item[\raisebox{-4pt}{{\ChoiceMenu[radio,radiosymbol=\ding{52},name=ic]{}{=2}}}]{Principiante}
\item[\raisebox{-4pt}{{\ChoiceMenu[radio,radiosymbol=\ding{52},name=ic]{}{=3}}}]{Intermedio}
\item[\raisebox{-4pt}{{\ChoiceMenu[radio,radiosymbol=\ding{52},name=ic]{}{=4}}}]{Avanzado}
\end{itemize}

\end{multicols}

Conocimientos de otros idiomas: \\  \\
\TextField[name=idiomas,multiline=true,bordercolor=black,align=0,width=\textwidth,height=15em]{}

\newpage

{\bf Describe brevemente en tus propias palabras qué significan para
  tí los siguientes conceptos y qué experiencia tienes con ellos hasta
  la fecha}

Ingeniería de sistemas: \\ \\
\TextField[name=sis,multiline=true,bordercolor=black,align=0,width=\textwidth,height=8em]{}

Investigación de operaciones: \\ \\
\TextField[name=or,multiline=true,bordercolor=black,align=0,width=\textwidth,height=8em]{}

Inteligencia computacional: \\ \\
\TextField[name=int,multiline=true,bordercolor=black,align=0,width=\textwidth,height=8em]{}

\quad

{\bf Plan de financiamiento de estudios:}

{\textquestiondown}Cómo te apoyarás económicamente durante los
estudios de posgrado?
\begin{itemize}[itemsep=-2pt]
\item[\raisebox{-4pt}{{\ChoiceMenu[radio,radiosymbol=\ding{52},name=ec]{}{=1}}}]{Solicito
    beca de CONACyT de estudios de tiempo completo}
\item[\raisebox{-4pt}{{\ChoiceMenu[radio,radiosymbol=\ding{52},name=ec]{}{=2}}}]{Solicito
    otra beca (especificar abajo)}
\item[\raisebox{-4pt}{{\ChoiceMenu[radio,radiosymbol=\ding{52},name=ec]{}{=3}}}]{Ya cuento
    con un apoyo (especificar abajo)}
\end{itemize}

Plan financiero (indicar cuántos dependientes económicos tienes): \\ \\
\TextField[name=fin,multiline=true,bordercolor=black,align=0,width=\textwidth,height=8em]{}

\newpage

{\textquestiondown}Tienes planes a futuro que quisieras compartir,
como por ejemplo estancias en el extranjero o emprendemiento con una
empresa propia? \\ \\
\TextField[name=fut,multiline=true,bordercolor=black,align=0,width=\textwidth,height=8em]{}

{\textquestiondown}Cuáles son tus pasatiempos favoritos? \\ \\
\TextField[name=pas,multiline=true,bordercolor=black,align=0,width=\textwidth,height=8em]{}

{\textquestiondown}Cómo te enteraste de este programa de posgrado? \\ \\
\TextField[name=ent,multiline=true,bordercolor=black,align=0,width=\textwidth,height=8em]{}

{\textquestiondown}Hay alguna otra cosa sobre ti que te gustaría que conocieramos? \\ \\
\TextField[name=ent,multiline=true,bordercolor=black,align=0,width=\textwidth,height=8em]{}

\quad

\quad

{\em Certifico que toda la información que he proveído en esta solicitud
  está completa y correcta.}

\quad

\begin{center}
  \begin{tabular}{cc}
    \rule{5cm}{0.5pt}
    &
 \TextField[name=fecha,multiline=false,bordercolor=black,align=0,width=3cm,height=1.2em]{Fecha}
    \\
    Firma 
\end{tabular}
\end{center}

\quad

\quad

\begin{flushright}
  {\em Favor de imprimir (guardar) en PDF una vez llenado.}
\end{flushright}

\newpage

{\sc Documentos por incluir en el expediente de solicitud en PDF}

Deben ser claramente nombrados; si un documento no existe en tu caso,
omítelo. Es importante que tu {\em curriculum vitae} sea actualizada,
de preferencia con fotografía, incluya una lista de publicaciones (si
es que tengas, indicando ISBN y DOI donde aplique), participación en
congresos y estancias académicas. 

Se puede incluir Hasta tres cartas de recomendación; son opcionales en
maestría, se necesita por lo menos dos en doctorado.

Para doctorado, es indispensable incluir un {\em anteproyecto} de 2--5
páginas, en español o en inglés, con las siguientes secciones:
Resumen, Introducción, Antecedentes, Estado del Arte, Contribución
Esperada, Metodología, Calendario de Actividades, Referencias; anexado
al anteproyecto, un visto bueno en escrito de parte del asesor (o en
su caso, los co-asesores), un o dos párrafos de texto, indicando las
fortalezas particulares del candidato y del proyecto propuesto. Por lo
menos uno de los asesores debe ser actualmente un profesor del
programa.

\begin{flushright}
\CheckBox{Esta solicitud} \\
\CheckBox{Curriculum vitae en formato libre} \\
\CheckBox{Copias de kárdex de estudios superiores a la preparatoria} \\
\CheckBox{Tesis de grados anteriores en PDF (si es que haya)} \\
\CheckBox{Cartas de recomendación (solamente para doctorado)} \\
\CheckBox{Anteproyecto (solamente para doctorado)}
\end{flushright}

Todos los documentos se colocan en formato PDF en una carpeta
compartida (Google Drive, Dropbox o similar) y la URL de acceso de
envia vía electrónica {\em antes del cierre de la convocatoria} a
\begin{itemize}
\item \url{admision.m@yalma.fime.uanl.mx} para el programa de maestría,
\item \url{admision.d@yalma.fime.uanl.mx} para el programa doctoral.
\end{itemize}
Es necesario que la documentación esté completa antes de enviar el
correo de acceso, pero se permite actualizar los archivos hasta el
cierre de la convocatoria en el caso que descubras algún error entre
el envió y el cierre,

Cada aspirante necesita además
\begin{enumerate}
\item presentar los {\bf exámenes de admisión del programa} y acudir a la
  {\em entrevista} en las fechas establecidas por la comisión de admisión
  correspondiente después del cierre de la convocatoria (se
  comunicarán por correo electrónico con los aspirantes);
\item presentar los {\bf exámenes generales de admisión de posgrado de
  la UANL}: conocimientos generales (EXANI) e inglés (EXCI), con la
  vigencia que requiere la UANL;
\item realizar todos los {\bf trámites administrativos} que requiera
  la facultad o la universidad (verificar con departamento escolar de
  posgrado, ;a somoisión de admisiones {\bf no} canaliza estos
  trámites);
\end{enumerate}
es importante informarse directamente con la FIME y la UANL sobre los
requisitos ya que la comisión de admisiones no tiene ningún poder o
conocimiento específico sobre los pasos, fechas o costos.

\end{Form}


\end{document}
