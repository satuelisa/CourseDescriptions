\section{Estructuraci\'{o}n en cap\'{\i}tulos, etapas o fases de la unidad de
  aprendizaje:}
\subsection{Desarrollo de las fases de la Unidad de Aprendizaje:}

\quad

Se cubren los principios te\'{o}ricos de la {\em complejidad
  computacional.} Se busca desarrollar habilidades en la
resoluci\'{o}n en casos pr\'{a}cticos concretos. Se necesita contar
con un buen entendimiento de varios los conceptos matem\'{a}ticos,
especialmente de matem\'{a}ticas discretas y probabilidad, o en el
caso contrario, estar preparado a estudiarlos seg\'{u}n
necesidad. Tambi\'{e}n se necesita conocimiento de programaci\'{o}n.

\paragraph{Unidades tem\'{a}ticas}

\begin{description}[itemsep=-2pt]
\item[UT1] Principios t\'{e}oricos de la complejidad computacional (7 semanas)
\item[UT2] Clases de complejidad (P, NP, PSPACE, etc.; 9 semanas)
\item[UT3] Aproximabilidad (1 semana)
\end{description}

\paragraph{Temario semanal}

\quad

La sesiones son de cuatro horas cada una y son veinte semanas en
total. La primera semana es introductoria y las \'{u}ltimas dos semanas
combinan elementos de las tres unidades tem\'{a}ticas en el contexto del
proyecto integrador.


\begin{enumerate}[itemsep=-3pt]
\item{Introducci\'{o}n; selecci\'{o}n de temas de proyecto (1 semana)}
\item{UT1: Problemas y algoritmos (2 semanas)}
\item{UT1: L\'{o}gica (2 semanas)}
\item{UT1: M\'{a}quinas Turing (3 semanas)}
\item{UT2: Clases de complejidad (9 semanas)}
\item{UT3: Esquemas de aproximaci\'{o}n (1 semana)}
\item{Presentaciones de proyectos (1 semana)}
\item{Revisi\'{o}n de portafolios de evidencia (1 semana)}
\end{enumerate}

\newpage

\subsubsection{Unidad tem\'{a}tica 1: Principios t\'{e}oricos de la
  complejidad computacional}

\paragraph{Periodo:} 7 semanas

\paragraph{Elementos de competencia:}

\quad

\begin{tabular}{|p{28mm}|p{30mm}|p{30mm}|p{30mm}|p{30mm}|}
  \hline
  \cellcolor{tableheader}Evidencias de aprendizaje 
  & \cellcolor{tableheader}Criterios de desempe\~{n}o
  & \cellcolor{tableheader}Actividades de aprendizaje
  & \cellcolor{tableheader}Contenidos
  & \cellcolor{tableheader}Recursos \\ \hline


Siete (7) tareas semanales consistiendo en reportes escritos de
resoluci\'{o}n de problemas de la teor\'{\i}a de la computaci\'{o}n.

&

Calidad de la redacci\'{o}n cient\'{\i}fica de los reportes.

&

Lectura de material de apoyo; modificaci\'{o}n de ejemplos;
an\'{a}lisis y reportaje de ejercicios realizados.

&

Conceptos fundamentales de la complejidad computacional.

&

Material en la p\'{a}gina web de la unidad y la literatura citada;
paquete {\LaTeX} para redacci\'{o}n cient\'{\i}fica; repositorios de
GitHub. \\ \hline
  
\end{tabular}

\subsubsection{Unidad tem\'{a}tica 2: Clases de complejidad}

\paragraph{Periodo:} 9 semanas

\paragraph{Elementos de competencia:}

\quad

\begin{tabular}{|p{28mm}|p{30mm}|p{30mm}|p{30mm}|p{30mm}|}
  \hline
  \cellcolor{tableheader}Evidencias de aprendizaje 
  & \cellcolor{tableheader}Criterios de desempe\~{n}o
  & \cellcolor{tableheader}Actividades de aprendizaje
  & \cellcolor{tableheader}Contenidos
  & \cellcolor{tableheader}Recursos \\ \hline


Nueve (9) tareas semanales consistiendo en reportes escritos de la
demostraci\'{o}n de complejidad computacional de problemas.

&

Calidad de la redacci\'{o}n cient\'{\i}fica de los reportes; validez y
claridad de las demostraciones.

&

Lectura de material de apoyo; modificaci\'{o}n de ejemplos;
an\'{a}lisis y reportaje de ejercicios realizados.

&

Definiciones de las clases de complejidad computacional.

&

Material en la p\'{a}gina web de la unidad y la literatura citada;
paquete {\LaTeX} para redacci\'{o}n cient\'{\i}fica; repositorios de
GitHub. \\ \hline
  
\end{tabular}

\newpage

\subsubsection{Unidad tem\'{a}tica 3: Aproximabilidad}

\paragraph{Periodo:} 1 semana

\paragraph{Elementos de competencia:}

\quad

\begin{tabular}{|p{28mm}|p{30mm}|p{30mm}|p{30mm}|p{30mm}|}
  \hline
  \cellcolor{tableheader}Evidencias de aprendizaje 
  & \cellcolor{tableheader}Criterios de desempe\~{n}o
  & \cellcolor{tableheader}Actividades de aprendizaje
  & \cellcolor{tableheader}Contenidos
  & \cellcolor{tableheader}Recursos \\ \hline


Una tarea consistiendo en un reporte escrito de la demostraci\'{o}n de
aproximabilidad de un problema.

&

Calidad de la redacci\'{o}n cient\'{\i}fica del reporte; validez y
claridad de la demostraci\'{o}n.

&

Lectura de material de apoyo; modificaci\'{o}n de ejemplos;
an\'{a}lisis y reportaje de ejercicios realizados.

&

Conceptos te\'{o}ricos de esquemas de aproximaci\'{o}n.

&

Material en la p\'{a}gina web de la unidad y la literatura citada;
paquete {\LaTeX} para redacci\'{o}n cient\'{\i}fica; repositorios de
GitHub. \\ \hline
  
\end{tabular}

\section{Evaluaci\'{o}n integral de procesos y productos:}

Las tareas son individuales; se recomienda estudiar juntos y discutir
las soluciones, pero no se tolera ning\'{u}n tipo de plagio en
absoluto, ni de otros estudiantes ni de la red ni de libros --- toda
referencia bibliogr\'{a}fica tiene que ser apropiadamente citada. La
entrega se realiza por un repositorio p\'{u}blico que debe reflejar todas
las fases del trabajo. 

No habr\'{a} examen.  Son 17 tareas (A1--A17) que reportan avances
semanales de aplicaci\'{o}n de la lectura de la semana para el
proyecto del estudiante, otorgando por m\'{a}ximo 5 puntos por
tarea: \begin{description}[itemsep=0em]
\item[NP]{= tarea omitida}
\item[5]{= excede lo que se esperaba}
\item[4]{= cumple con lo que se esperaba}
\item[3]{= d\'{e}bil en alcance y/o calidad}
\item[2]{= d\'{e}bil en ambos alcance y calidad}
\item[1]{= sin contribuciones o m\'{e}ritos aunque fue entregada}
\item[0]{= completamente inadecuado en alzance y calidad}
\end{description}
 El proyecto final (A18) otorga un
m\'{a}ximo de 15 puntos, evaluados en los siguientes
rubros \begin{enumerate}[itemsep=-1pt]
\item{Variedad de t\'{e}cnicas de empleadas}
\item{Cobertura y validez de la experimentaci\'{o}n}
\item{Claridad y relevancia de los resultados}    
\item{Calidad de visualizaci\'{o}n cient\'{\i}fica}
\item{Calidad de redacci\'{o}n cient\'{\i}fica}
\end{enumerate}
 con la escala: \begin{description}[itemsep=0em]
\item[3]{= cumple con lo que se esperaba}
\item[2]{= d\'{e}bil en alcance y/o calidad}
\item[1]{= d\'{e}bil en ambos alcance y calidad}
\item[0]{= inadecuado en alzance y calidad}
\end{description}


\paragraph{Ponderaci\'{o}n espec\'{\i}fica}

\quad

\scalebox{0.85}{
  \begin{tabular}{|c|cccccccccccccccccc|c|}
    \hline
    \rotatebox{90}{\cellcolor{tableheader}{\bf Actividad\phantom{xx}}}
    & A1
    & A2
    & A3
    & A4
    & A5
    & A6
    & A7
    & A8
    & A9
    & A10
    & A11
    & A12
    & A13
    & A14
    & A15
    & A16
    & A17
    & A18
    & {\bf Total}
    \\
    \hline
    \rotatebox{90}{\cellcolor{tableheader}{\bf Ponderaci\'{o}n\phantom{xx}}}
    & 5\%
    & 5\%
    & 5\%
    & 5\% 
    & 5\%
    & 5\%
    & 5\%
    & 5\%
    & 5\%
    & 5\%
    & 5\%
    & 5\% 
    & 5\%
    & 5\%
    & 5\%
    & 5\%
    & 5\%
    & 15\%
    & 100\%		
    \\ \hline
  \end{tabular}}
  


\section{Producto integrador de aprendizaje de la unidad:}

\subsection{Producto integrador de Aprendizaje:}

\quad

Portafolio en un repositorio digital p\'{u}blico que contiene los
reportes escritos y los c\'{o}digos de la implementaci\'{o}n de todas
las tareas y el proyecto integrador.


\section{Fuentes de apoyo y consulta:}
\subsection{Fuentes de apoyo y consulta}
\subsubsection{B\'{a}sicas}

\begin{itemize}[itemsep=0em]
  
\item{C.H.\ {\sc Papadimitriou}: {\em Computational complexity}. John
    Wiley and Sons Ltd., 2003.}
  
\item{S.\ {\sc Arora} \& B.\ {\sc Boaz}: {\em Computatioonal complexity:
      a modern approach}. Cambridge University Press, 2009.}
  
\item{M.R.\ {\sc Garey} \& D.S.\ {\sc Johnson}: {\em Computers and
      intractability}. Vol. 29. New York: Freeman, 2002.}
  
\end{itemize}

\subsubsection{Complementarias}

Art\'{\i}culos cient\'{\i}ficos especializados relacionados a los
temas tratados, de preferencia publicados en revistas internacionales
indizados recientes.


\label{final} % last page
%\newpage

%\pagestyle{plain}

%\vspace*{3cm}

%{\bf Autoriz\'{o}:} \coordinador

%\vspace*{2cm}

%  \begin{center}
%  {\sc Alere Flammam Veritatis}
  
%  Ciudad Universitaria, \today

%\vspace*{4cm}
  
%  \begin{tabular}{p{6cm}cp{7cm}}
%    \cline{1-1}
%    \cline{3-3}    
%    {\bf \coordinador}
% & \phantom{xxx} &{\bf Vo.\ Bo.\ \subdirector} \\
%    Coordinador Acad\'{e}mico &  &Subdirector de Estudios de Posgrado \\
%    Posgrado en Ingenier\'{\i}a de Sistemas & & Facultad de Ingenier\'{\i}a Mec\'{a}nica y El\'{e}ctrica
                                                                   
%  \end{tabular}
%\end{center}

\end{document}
