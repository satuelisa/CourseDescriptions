\section{Estructuraci\'{o}n en cap\'{\i}tulos, etapas o fases de la unidad de
  aprendizaje:}

\subsection{Desarrollo de las fases de la Unidad de Aprendizaje:}

\quad

Se busca desarrollar habilidades en la resoluci\'{o}n en casos
pr\'{a}cticos concretos. Se necesita contar con un buen entendimiento
de varios los conceptos matem\'{a}ticos, especialmente de
matem\'{a}ticas discretas y probabilidad, o en el caso contrario,
estar preparado a estudiarlos seg\'{u}n necesidad. Tambi\'{e}n se
necesita conocimiento de programaci\'{o}n.

\paragraph{Unidades tem\'{a}ticas}

\begin{description}[itemsep=-2pt]
\item[U1]{Preprocesamiento y visualizaci\'{o}n}
\item[U2]{M\'{e}todos estad\'{i}sticos en la ciencia de datos}
\item[U3]{Objectivos y t\'{e}cnicas para aplicaciones de la ciencia de datos}
\end{description}

La sesiones son de cuatro horas cada una y son veinte semanas en
total.

\newpage

\paragraph{Temario semanal}

\begin{enumerate}[itemsep=-3pt]
\item{Introducci\'{o}n; selecci\'{o}n de temas de proyecto}
\item{U1: Preparaci\'{o}n de datos}
\item{U1: Lectura y manipulaci\'{o}n de datos}
\item{U1: Estad\'{\i}stica descriptiva b\'{a}sica}
\item{U1: Detecci\'{o}n de anomalias}
\item{U1: Visualizaci\'{o}n est\'{a}tica de informaci\'{o}n}
\item{U1: Visualizaci\'{o}n din\'{a}mica de informaci\'{o}n}
\item{U2: Pruebas estad\'{\i}sticas}
\item{U2: Modelos lineales}
\item{U2: Regresi\'{o}n m\'{u}ltiple}
\item{U2: An\'{a}lisis de varianza y de componentes principales}
\item{U2: Pron\'{o}sticos}
\item{U3: Clasificaci\'{o}n de datos}
\item{U3: Agrupamiento de datos}
\item{U3: An\'{a}lisis de texto}
\item{U3: An\'{a}lisis de im\'{a}genes}
\item{U3: Datos grandes}
\item{U3: Procesamiento en tiempo real}
\item{Presentaciones de proyectos}
\item{Revisi\'{o}n de portafolios de evidencia}
\end{enumerate}

\paragraph{Elementos de competencia}

\quad

\paragraph{Elementos de competencia:}

\quad

\begin{tabular}{|p{28mm}|p{30mm}|p{30mm}|p{30mm}|p{30mm}|}
  \hline
  \cellcolor{tableheader}Evidencias de aprendizaje 
  & \cellcolor{tableheader}Criterios de desempe\~{n}o
  & \cellcolor{tableheader}Actividades de aprendizaje
  & \cellcolor{tableheader}Contenidos
  & \cellcolor{tableheader}Recursos \\ \hline


Reporte escrito y c\'{o}digo de la implementaci\'{o}n de un m\'{e}todo
de ciencia de datos.

&

Calidad de la redacci\'{o}n cient\'{\i}fica del reporte; 
precisi\'{o}n del algoritmo propuesto;
eficiencia de la implementaci\'{o}n del algoritmo;
cobertura de la experimentaci\'{o}n.

&

Experimentaci\'{o}n con ejemplos; lectura de material de apoyo;
modificaci\'{o}n de ejemplos; dise\~{n}o y ejecuci\'{o}n de experimentos;
an\'{a}lisis y reportaje de resultados obtenidos.

&

M\'{e}todos diversos de ciencia de datos.

&

Material en la p\'{a}gina web de la unidad y la literatura citada;
lenguaje Python o similar; paquete {\LaTeX} para redacci\'{o}n cient\'{\i}fica;
repositorios de GitHub. \\ \hline

\end{tabular}

\newpage

\section{Evaluaci\'{o}n integral de procesos y productos:}

Las tareas son individuales; se recomienda estudiar juntos y discutir
las soluciones, pero no se tolera ning\'{u}n tipo de plagio en
absoluto, ni de otros estudiantes ni de la red ni de libros --- toda
referencia bibliogr\'{a}fica tiene que ser apropiadamente citada. La
entrega se realiza por un repositorio p\'{u}blico que debe reflejar todas
las fases del trabajo. 

No habr\'{a} examen.  Son 17 tareas (A1--A17) que reportan avances
semanales de aplicaci\'{o}n de la lectura de la semana para el
proyecto del estudiante, otorgando por m\'{a}ximo 5 puntos por
tarea: \begin{description}[itemsep=-2pt]
\item[NP]{= tarea omitida}
\item[5]{= excede lo que se esperaba}
\item[4]{= cumple con lo que se esperaba}
\item[3]{= d\'{e}bil en alcance y/o calidad}
\item[2]{= d\'{e}bil en ambos alcance y calidad}
\item[1]{= sin contribuciones o m\'{e}ritos aunque fue entregada}
\item[0]{= completamente inadecuado en alzance y calidad}
\end{description}
 El proyecto final (A18) otorga un
m\'{a}ximo de 15 puntos, evaluados en los siguientes
rubros \begin{enumerate}[itemsep=0em]
\item{Variedad de t\'{e}cnicas de empleadas}
\item{Cobertura y validez de la experimentaci\'{o}n}
\item{Claridad y relevancia de los resultados}    
\item{Calidad de visualizaci\'{o}n cient\'{\i}fica}
\item{Calidad de redacci\'{o}n cient\'{\i}fica}
\end{enumerate}
 con la escala: \begin{description}[itemsep=0em]
\item[3]{= cumple con lo que se esperaba}
\item[2]{= d\'{e}bil en alcance y/o calidad}
\item[1]{= d\'{e}bil en ambos alcance y calidad}
\item[0]{= inadecuado en alzance y calidad}
\end{description}


\paragraph{Ponderaci\'{o}n espec\'{\i}fica}

\quad

\scalebox{0.85}{
  \begin{tabular}{|c|cccccccccccccccccc|c|}
    \hline
    \rotatebox{90}{\cellcolor{tableheader}{\bf Actividad\phantom{xx}}}
    & A1
    & A2
    & A3
    & A4
    & A5
    & A6
    & A7
    & A8
    & A9
    & A10
    & A11
    & A12
    & A13
    & A14
    & A15
    & A16
    & A17
    & A18
    & {\bf Total}
    \\
    \hline
    \rotatebox{90}{\cellcolor{tableheader}{\bf Ponderaci\'{o}n\phantom{xx}}}
    & 5\%
    & 5\%
    & 5\%
    & 5\% 
    & 5\%
    & 5\%
    & 5\%
    & 5\%
    & 5\%
    & 5\%
    & 5\%
    & 5\% 
    & 5\%
    & 5\%
    & 5\%
    & 5\%
    & 5\%
    & 15\%
    & 100\%		
    \\ \hline
  \end{tabular}}
  

  
\newpage

\section{Producto integrador de aprendizaje de la unidad:}

\subsection{Producto integrador de Aprendizaje:}

\quad

Portafolio en un repositorio digital p\'{u}blico que contiene los
reportes escritos y los c\'{o}digos de la implementaci\'{o}n de todas
las tareas y el proyecto integrador.


\section{Fuentes de apoyo y consulta:}

\subsection{Fuentes de apoyo y consulta}

\subsubsection{B\'{a}sicas}

\begin{itemize}[itemsep=0em]
  
\item{D.T.\ {\sc Larose} \& C.D.\ {\sc Larose}: {\em Data Science Using Python and R}
    Wiley, abril 2019, 256 p\'{a}ginas ISBN-13 978-1119526810}
  
\item{H.\ {\sc Wickham} \& G.\ {\sc  Grolemund}:
    {\em R for Data Science: Import, Tidy, Transform, Visualize, and Model Data},
    O'Reilly Media, enero 2017, 
    520 p\'{a}ginas, ISBN-13 978-1491910399}
  
\item{M.\ {\sc Trovati}, et al., eds.: {\em Big-Data Analy;cs and
      Cloud Compu;ng: Theory, Algorithms and Applications}. Springer,
    2016.}
\end{itemize}

\subsubsection{Complementarias}

Art\'{\i}culos cient\'{\i}ficos especializados relacionados a los
temas tratados, de preferencia publicados en revistas internacionales
indizados recientes.


\label{final} % last page
\newpage

\pagestyle{plain}

\vspace*{3cm}

{\bf Autoriz\'{o}:} \coordinador

\vspace*{2cm}

  \begin{center}
  {\sc Alere Flammam Veritatis}
  
  Ciudad Universitaria, \today

\vspace*{4cm}
  
  \begin{tabular}{p{6cm}cp{7cm}}
    \cline{1-1}
    \cline{3-3}    
    {\bf \coordinador} &
                                                            \phantom{xxx} &{\bf Vo.\ Bo.\ \subdirector} \\
    Coordinador Acad\'{e}mico &  &Subdirector de Estudios de Posgrado \\
    Posgrado en Ingenier\'{\i}a de Sistemas & & Facultad de Ingenier\'{\i}a Mec\'{a}nica y El\'{e}ctrica
                                                                   
  \end{tabular}
\end{center}

\label{final} % last page
\newpage

\pagestyle{plain}

\vspace*{3cm}

{\bf Autoriz\'{o}:} \coordinador

\vspace*{2cm}

  \begin{center}
  {\sc Alere Flammam Veritatis}
  
  Ciudad Universitaria, \today

\vspace*{4cm}
  
  \begin{tabular}{p{6cm}cp{7cm}}
    \cline{1-1}
    \cline{3-3}    
    {\bf \coordinador} &
                                                            \phantom{xxx} &{\bf Vo.\ Bo.\ \subdirector} \\
    Coordinador Acad\'{e}mico &  &Subdirector de Estudios de Posgrado \\
    Posgrado en Ingenier\'{\i}a de Sistemas & & Facultad de Ingenier\'{\i}a Mec\'{a}nica y El\'{e}ctrica
                                                                   
  \end{tabular}
\end{center}

\end{document}
