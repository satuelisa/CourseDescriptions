\section{Estructuraci\'{o}n en cap\'{\i}tulos, etapas o fases de la unidad de
  aprendizaje:}

\subsection{Desarrollo de las fases de la Unidad de Aprendizaje:}

\quad

Los estudiantes forman tres comit\'{e}s: el primero se encarga del
contacto directo con los ponentes en cuesti\'{o}n organizacional, el
segundo se encarga de la publicidad previa y la organizaci\'{o}n el
d\'{\i}a del evento y el tercero se encarga de redactar rese\~{n}as.
La sesiones son de cuatro horas cada una y son veinte semanas en
total: 16 sesiones con ponentes m\'{a}s cuatro en total para
planeaci\'{o}n y retroalimentaci\'{o}n. En las 16 sesiones con una
exposici\'{o}n, un ponente o un panel de ponentes expone un tema,
seguido por una sesi\'{o}n de preguntas.

\paragraph{Unidades tem\'{a}ticas:}

\quad

Se desarrollan en paralelo las dos unidades a lo largo de las veinte
sesiones.

\begin{description}[itemsep=-2pt]
\item[U1]{Organizaci\'{o}n de foros cient\'{\i}ficos.}
\item[U2]{Divulgaci\'{o}n y di\'{a}logo cient\'{\i}fico.}
\end{description}

\paragraph{Temario semanal:}

\quad

Una sesi\'{o}n del seminario por semana durante el semestre.

\newpage

\paragraph{Elementos de competencia}

\quad

\paragraph{Elementos de competencia:}

\quad

\begin{tabular}{|p{28mm}|p{30mm}|p{30mm}|p{30mm}|p{30mm}|}
  \hline
  \cellcolor{tableheader}Evidencias de aprendizaje 
  & \cellcolor{tableheader}Criterios de desempe\~{n}o
  & \cellcolor{tableheader}Actividades de aprendizaje
  & \cellcolor{tableheader}Contenidos
  & \cellcolor{tableheader}Recursos \\ \hline


Portafolio de evidencias de aportaciones y participaciones.

&

Claridad del portafolio reporte; frecuencia y calidad de aportaciones.

&

Participaci\'{o}n en los comit\'{e}s y las presentaciones.

&

La comunicaci\'{o}n de la ciencia.

&

Material en la p\'{a}gina web de la unidad y la literatura citada;
paquete {\LaTeX}, HTML, CSS, YouTube.

\\ \hline
\end{tabular}

\section{Evaluaci\'{o}n integral de procesos y productos:}

Los criterios de calificaci\'{o}n son los siguientes:
\begin{description}
\item[A1 --- Asistencia a una sesi\'{o}n:]{2 pts (1 pt si se une tarde o si se
    va temprano)}
\item[A2 --- Participaci\'{o}n en una sesi\'{o}n:]{hasta 3 pts dependiendo de la
    calidad y cantidad de las preguntas}
\item[A3 --- Contribuciones a un comit\'{e} (si aplica):]{hasta 4 pts
    semanales, dependiendo de la frecuencia y magnitud de las
    aportaciones}
\item[A4 --- Participaci\'{o}n en preparaciones (fuera del propio
  comit\'{e}):]{hasta 2 pts semanales, dependiendo de la frecuencia y
    magnitud de las aportaciones}
\item[A5 --- Participaci\'{o}n en la retroalimentaci\'{o}n del ciclo
  anterior:]{hasta 5 pts, dependiendo de la frecuencia y magnitud de
    las aportaciones}
\item[A6 --- Participaci\'{o}n en la planeaci\'{o}n del ciclo siguiente:]{hasta 5
    pts, dependiendo de la frecuencia y magnitud de las aportaciones}
\end{description}
  
  
Ponderaci\'{o}n espec\'{\i}fica (aproximada):

\scalebox{0.9}{
  \begin{tabular}{|c|cccccc|c|}
    \hline
    \cellcolor{tableheader}{{\bf Actividad}}
    & A1
    & A2
    & A3
    & A4
    & A5
    & A6
    & Total \\
    \hline
    \cellcolor{tableheader}{{\bf Ponderaci\'{o}n}}
    & 20\%
    & 30\%
    & 35\%
    & 5\%
    & 5\%
    & 5\%
    & 100\%
    \\ \hline
  \end{tabular}}


\section{Producto integrador de aprendizaje de la unidad:}

\subsection{Producto integrador de Aprendizaje:}

\quad

{\em Portafolio de evidencias.}

La {\em asistencia} se comprueba con captura de pantalla de la
sesi\'{o}n de YouTube; es necesario hacer login para que se vea el
usuario activo en la captura.

La {\em participaci\'{o}n} se evidencia con la captura de pantalla del
chat de YouTube donde se ve la pregunta.

Las {\em contribuciones a los comit\'{e}s de organizaci\'{o}n} se
evidencian con capturas de pantalla de los canales de
comunicaci\'{o}n, capturas de pantalla de las contibuciones al
repositorio compartido.

Los integrantes del comit\'{e} de publicidad adem\'{a}s incluye
capturas de pantalla de las publicaciones en medios sociales, mientras
comit\'{e} de ponentes incluye capturas de pantalla de comunicaciones
con ponentes. 


