\section{Presentaci\'{o}n:}

Simulaci\'{o}n refiere a la reproducci\'{o}n (computacional) de
fen\'{o}menos y procesos del mundo real. T\'{\i}picamente involucra
primero el modelado de dicho proceso o fen\'{o}meno a trav\'{e}s de
experimentos estad\'{\i}sticos. Despu\'{e}s la reproducci\'{o}n
involucra la generaci\'{o}n pseudo-aleatoria para crear los modelos en
el ambiente simulado. Sus aplicaciones son numerosas en temas
cient\'{\i}ficos, industriales, de construcci\'{o}n, de la salud,
evacuaciones, etc\'{e}tera. Las epidemias son particularmente
interesantes de estudiar con simulaciones. El an\'{a}lisis de elemento
finito (FEM) para aspectos estructurales es muy parecido a la
simulaci\'{o}n; se extrapola el comportamiento de una entidad a
trav\'{e}s de un modelo que lo parte a una cantidad discreta de
elementos peque\~{n}os de comportamiento ya modelado. La
simulaci\'{o}n tambi\'{e}n est\'{a} presente en el entretenimiento
(pel\'{\i}culas y videojuegos) en la creaci\'{o}n de ambientes
virtuales y efectos.

En la unidad de aprendizaje se realizan actividades de aprendizaje que
permiten la paralelizaci\'{o}n de algunas tareas fundamentales de
simulaci\'{o}n, desde m\'{a}s sencillos hasta m\'{a}s complejos, para
que el participante pueda en sus trabajos futuros identificar
oportunidades de paralelizaci\'{o}n y dominar las t\'{e}cnicas
b\'{a}sicas de llevarlo a cabo con elegancia y eficiencia. Se
implementan diversas simulaciones computacionales para generar y
analizar informaci\'{o}n de distintos tipos.

\section{Prop\'{o}sito(s):}

Formaci\'{o}n de competencia profesional de nivel posgrado que permita
el participante identificar de fen\'{o}menos que se pueda estudiar
v\'{\i}a la simulaci\'{o}n computacional, proponer modelos
matem\'{a}ticos que capturen dichos fen\'{o}menos a un nivel adecuado,
implementar simulaciones computacionales de estos modelos, adem\'{a}s
de dise\~{n}ar, ejecutar y analizar experimentos computacionales que
adecuadamente capturan la precisi\'{o}n de dichos simulaciones y
permiten conclusiones estad\'{\i}sticamente v\'{a}lidas sobre el
fen\'{o}meno en cuesti\'{o}n. Adem\'{a}s se incluye de forma integral
la preparaci\'{o}n de competencias en la visualizaci\'{o}n de la
informaci\'{o}n cient\'{\i}fica y la redacci\'{o}n cient\'{\i}fica.
