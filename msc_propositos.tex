\documentclass{article}

\usepackage[spanish]{babel}
\usepackage[top=15mm,bottom=25mm,left=20mm,right=20mm]{geometry}

\title{Prop\'{o}sitos de programa}
\author{Maestr\'{\i}a en Ciencias de la Ingenier\'{\i}a con Orientaci\'{o}n en Sistemas}
\date{Plan de estudios, 2017}

\begin{document}

\maketitle

\section{Introducci\'{o}n}

La Maestr\'{\i}a en Ciencias de la Ingenier\'{\i}a con Orientaci\'{o}n en Sistemas
pertenece a la Facultad de Ingenier\'{\i}a Mec\'{a}nica y El\'{e}ctrica (FIME) de
la Universidad Aut\'{o}noma de Nuevo Le\'{o}n (UANL) y ofrece estudios en
ingenier\'{\i}a de sistemas, investigaci\'{o}n de operaciones y ciencias de la
decisi\'{o}n, empleando el m\'{e}todo cientifico y las herramientas
tecnol\'{o}gicas disponibles para resolver problemas de distintos \'{a}mbitos
como son el industrial, el acad\'{e}mico o el gubernamental.

En pa\'{\i}ses desarrollados, la mayor\'{\i}a de las industrias recurren a
expertos en esta rama (sistemas, investigaci\'{o}n de operaciones, ciencias
de decisi\'{o}n, etc.) para dar soluci\'{o}n a los problemas que enfrentan a
diario. En M\'{e}xico y en particular en Nuevo Le\'{o}n, es necesario tener
expertos que formulen, analicen y propongan metodolog\'{\i}as de soluci\'{o}n
que ayuden al proceso de toma de decisiones. Por mencionar un ejemplo,
todas las empresas, en Estados Unidos de Am\'{e}rica, en la industria del
transporte (a\'{e}rea, terrestre, mar\'{\i}tima) cuentan con su propio
departamento interno encargado de dar el soporte t\'{e}cnico y cientifico a
su muy complejo proceso de toma de decisiones. Para este fin, se toman
en cuenta todas las operaciones de asignaci\'{o}n de tripulaciones,
transporte, flete, log\'{\i}stica y satisfacci\'{o}n de demanda. Este soporte
que se brinda tiene un impacto muy fuerte en el aspecto econ\'{o}mico de
dichas empresas. En M\'{e}xico, nuestro posgrado est\'{a} formando
profesionistas e investigadores que son capaces de modelar, analizar y
solucionar este tipo de sistemas.

\section{Prop\'{o}sitos de trabajo institucional}

\begin{itemize}
\item{Consolidar un espacio p\'{u}blico de aprendizaje y ejercicio
    ciudadano.}
\item{PAmpliar y diversificar las oportunidades de acceso a la
    Universidad para un mayor n\'{u}mero de j\'{o}venes con deseos y capacidad
    para la realizaci\'{o}n de estudios}
\item{Asegurar la relevancia y pertinencia social de la oferta
    educativa de la Universidad, en todos sus tipos y
    modalidades.}
\item{Asegurar la igualdad de oportunidades educativas de buena
    calidad para todos.}
\item{Incorporar el Modelo Educativo y los Modelos Acad\'{e}micos de la
    Universidad en todos los programas de los niveles medio superior,
    licenciatura y posgrado que se ofrecen en los sistemas respectivos.}
    \item{Fortalecer y consolidar los programas que contribuyen a la
    permanencia, terminaci\'{o}n oportuna de los estudios y formaci\'{o}n
    integral de los estudiantes.}
\item{Continuar promoviendo la internacionalizaci\'{o}n de la Universidad.}
\item{Consolidar la cultura de la mejora continua de la calidad en
    todos los \'{a}mbitos del quehacer institucional.}
\end{itemize}

\section{Justificaci\'{o}n}

El programa Maestr\'{\i}a en Ciencias de la Ingenier\'{\i}a con Orientaci\'{o}n en
Sistemas e enfoca en modelos matem\'{a}ticos, el an\'{a}lisis y la soluci\'{o}n de
problemas de investigaci\'{o}n de operaciones. La investigaci\'{o}n de
operaciones utiliza t\'{e}cnicas cuantitativas para ayudar en la toma de
decisiones a nivel industrial y gubernamental, asistiendo en la
planeaci\'{o}n, construcci\'{o}n y operaci\'{o}n de sistemas. La especialidad es
interdisciplinaria y el graduado trabaja en una diversidad de campos
incluyendo docencia e investigaci\'{o}n en la academia, consultor\'{\i}a en
administraci\'{o}n, log\'{\i}stica y transporte, planeaci\'{o}n de producci\'{o}n y
comunicaciones, entre otras.  El proceso de globalizaci\'{o}n en el que
M\'{e}xico participa exige de sus profesionistas una preparaci\'{o}n s\'{o}lida,
actualizaci\'{o}n constante y capacidad para realizar investigaci\'{o}n. El
avance tecnol\'{o}gico y el desarrollo cientifico han ocasionado que la
vida en sociedad se organice alrededor de sistemas, cada d\'{\i}a m\'{a}s
complejos. Tanto en la industria como en la pol\'{\i}tica, en el sector
privado o p\'{u}blico, pr\'{a}cticamente en cualquier trabajo hay que
involucrarse con organizaciones y sistemas. Independientemente del
sistema particular que se trate (transporte, el\'{e}ctrico, manufactura,
energ\'{e}tico, computacional, etc.), existen un conjunto de funciones
comunes a los procesos como son: medici\'{o}n, evaluaci\'{o}n, optimizaci\'{o}n y
toma de decisiones. El PE ofrece a los egresados de las diversas
carreras de ingenier\'{\i}a de la regi\'{o}n, la oportunidad de profundizar en
estas funciones y prepararse adecuadamente para realizar en un
ambiente multidisciplinario, un trabajo que logre mejorar la
eficiencia de la organizaci\'{o}n en donde se desempe\~{n}an.

\section{Prop\'{o}sito del proyecto educativo}

La estructura general del PE se fundamenta en las ciencias b\'{a}sicas y
el estudio cientifico con alcance hasta la ingenier\'{\i}a aplicada y el
desarrollo tecnol\'{o}gico. Su objetivo general es la formaci\'{o}n de
recursos humanos de alto nivel. Adem\'{a}s de la selecci\'{o}n cuidadosa del
profesorado, los criterios de admisi\'{o}n al programa son elementos
claves para asegurar que los recursos humanos formados sean de alto
nivel y competitivos a nivel internacional.  El objetivo general de
este programa de posgrado es proveer al estudiante con la base
educacional para el aprendizaje continuo, as\'{\i} como impartir las
habilidades fundamentales necesarias para que logre desempe\~{n}ar de una
manera efectiva su profesi\'{o}n, la ingenier\'{\i}a de sistemas.

\subsection{Objetivos}

\begin{enumerate}
\item{Formar recursos humanos de primer nivel capaces de resolver
    efectivamente problemas de toma de decisiones que surgen en los
    ramos acad\'{e}mico, industrial y gubernamental.}
\item{Efectuar labores de investigaci\'{o}n en las l\'{\i}neas de generaci\'{o}n y
    aplicaci\'{o}n del conocimiento definidas, permaneciendo la maestr\'{\i}a a
    la vanguardia en dichas l\'{\i}neas de investigaci\'{o}n, con la
    participaci\'{o}n de los estudiantes del programa.}
  \item{Establecer lazos de vinculaci\'{o}n con la industria regional y
      nacional, cuyas problem\'{a}ticas existentes involucran problemas de
      toma de decisiones y, por ende, pueden ser significativamente
      beneficiados mediante las herramientas cuantitativas y
      anal\'{\i}ticas disponibles y/o desarrolladas en este programa
      educativo.}
\item{Colaborar con la facultad en la realizaci\'{o}n y permanencia de
    convenios con otras universidades o centros de investigaci\'{o}n a
    nivel nacional e internacional, con problem\'{a}ticas o intereses
    similares, que permitan un beneficio mutuo tanto en materia de
    investigaci\'{o}n como de formaci\'{o}n de estudiantes.}
\end{enumerate}


\end{document}