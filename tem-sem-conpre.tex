\section{Estructuraci\'{o}n en cap\'{\i}tulos, etapas o fases de la unidad de
  aprendizaje:}

\subsection{Desarrollo de las fases de la Unidad de Aprendizaje:}

\quad

Los estudiantes forman tres comit\'{e}s: el primero se encarga del
contacto directo con los ponentes en cuesti\'{o}n organizacional, el
segundo se encarga de la publicidad previa y la organizaci\'{o}n el
d\'{\i}a del evento y el tercero se encarga de redactar rese\~{n}as.
La sesiones son de cuatro horas cada una y son veinte semanas en
total: por lo general son16 sesiones con ponentes m\'{a}s cuatro en
total para planeaci\'{o}n y retroalimentaci\'{o}n; el n\'{u}mero exacto
depende de los asuetos del calendario oficial de la UANL En las
sesiones con una exposici\'{o}n, un ponente o un panel de ponentes
expone un tema, seguido por una sesi\'{o}n de preguntas.

\paragraph{Unidades tem\'{a}ticas:}

\quad

Se desarrollan {\bf en paralelo} las tres unidades a lo largo de las
veinte sesiones. 

\begin{description}[itemsep=-2pt]
\item[U1]{Organizaci\'{o}n de foros cient\'{\i}ficos.}
\item[U2]{Divulgaci\'{o}n y di\'{a}logo cient\'{\i}fico.}
\item[U3]{Presentaci\'{o}n en foros cient\'{\i}ficos.}
\end{description}

\paragraph{Temario semanal:}

\quad

Hay una sesi\'{o}n del seminario por semana durante el semestre. El
tema varia según los ponentes en cada semestre.


\subsubsection{Unidad tem\'{a}tica 1: Organizaci\'{o}n de foros cient\'{\i}ficos.}

\paragraph{Periodo:} en paralelo con UT2, 20 semanas

\paragraph{Elementos de competencia:}

\quad

\begin{tabular}{|p{28mm}|p{30mm}|p{30mm}|p{30mm}|p{30mm}|}
  \hline
  \cellcolor{tableheader}Evidencias de aprendizaje 
  & \cellcolor{tableheader}Criterios de desempe\~{n}o
  & \cellcolor{tableheader}Actividades de aprendizaje
  & \cellcolor{tableheader}Contenidos
  & \cellcolor{tableheader}Recursos \\ \hline


Portafolio de evidencias de aportaciones y participaciones.

&

Claridad del portafolio; frecuencia y calidad de aportaciones.

&

Participaci\'{o}n en los comit\'{e}s.

&

La comunicaci\'{o}n de la ciencia.

&

Material en la p\'{a}gina web de la unidad y la literatura citada;
paquete {\LaTeX}, HTML, CSS, YouTube.

\input{elemtableend.tex}



\newpage

\input{ut2sem.tex}

\subsubsection{Unidad tem\'{a}tica 3: Presentaci\'{o}n en foros
  cient\'{\i}ficos}

\paragraph{Periodo:} aproximadamente tres semanas de preparaciones,
una semana de presentaci\'{o}n y una de autoevaluaci\'{o}n.

\paragraph{Elementos de competencia:}

\quad

\begin{tabular}{|p{28mm}|p{30mm}|p{30mm}|p{30mm}|p{30mm}|}
  \hline
  \cellcolor{tableheader}Evidencias de aprendizaje 
  & \cellcolor{tableheader}Criterios de desempe\~{n}o
  & \cellcolor{tableheader}Actividades de aprendizaje
  & \cellcolor{tableheader}Contenidos
  & \cellcolor{tableheader}Recursos \\ \hline


Diapositivas y grabaci\'{o}n de la presentaci\'{o}n.

&

Claridad de la presentaci\'{o}n.

&

Presentaci\'{o}n propia.

&

La comunicaci\'{o}n de la ciencia.

&

Material en la p\'{a}gina web de la unidad y la literatura citada;
paquete {\LaTeX}, HTML, CSS, YouTube.

\input{elemtableend.tex}


\newpage

\section{Evaluaci\'{o}n integral de procesos y productos:}

Los criterios de calificaci\'{o}n son los siguientes:
\input{crit_exp.tex}
  
Ponderaci\'{o}n espec\'{\i}fica (aproximada):

\scalebox{0.9}{
  \begin{tabular}{|c|ccccccc|c|}
    \hline
    \cellcolor{tableheader}{{\bf Actividad}}
    & A1
    & A2
    & A3
    & A4
    & A5
    & A6
    & A7
    & Total \\
    \hline
    \cellcolor{tableheader}{{\bf Ponderaci\'{o}n}}
    & 20\%
    & 25\%
    & 30\%
    & 5\%
    & 5\%
    & 5\%
    & 10\%      
    & 100\%
    \\ \hline
  \end{tabular}}
 

\section{Producto integrador de aprendizaje de la unidad:}

\subsection{Producto integrador de Aprendizaje:}

{\em Portafolio de evidencias.}

Se incluye la URL en YouTube de la presentaci\'{o}n propia.

La {\em asistencia} se comprueba con captura de pantalla de la
sesi\'{o}n de YouTube; es necesario hacer login para que se vea el
usuario activo en la captura.

La {\em participaci\'{o}n} se evidencia con la captura de pantalla del
chat de YouTube donde se ve la pregunta.

Las {\em contribuciones a los comit\'{e}s de organizaci\'{o}n} se
evidencian con capturas de pantalla de los canales de
comunicaci\'{o}n, capturas de pantalla de las contibuciones al
repositorio compartido.

Los integrantes del comit\'{e} de publicidad adem\'{a}s incluye
capturas de pantalla de las publicaciones en medios sociales, mientras
comit\'{e} de ponentes incluye capturas de pantalla de comunicaciones
con ponentes. 

