\documentclass[10 pt]{article}
\usepackage{wallpaper}
\usepackage[spanish, mexico]{babel}  
\usepackage{color,colortbl}
\usepackage[T1]{fontenc}
\usepackage{fancyhdr} % before geometry
\usepackage[letterpaper,left=18mm,right=18mm,headheight=5mm,headsep=40mm,top=52mm,bottom=22mm]{geometry}
\usepackage{graphicx}
\usepackage{rotating}
\usepackage[latin1]{inputenc}
\usepackage{hyperref}
\usepackage{graphics}
\usepackage{varwidth}
\usepackage{tikz}

\usepackage{tcolorbox}
\definecolor{headerframe}{RGB}{177,178,177} % #b1b2b1
\definecolor{headercontent}{RGB}{239,239,239} % #efefef
\definecolor{tableheader}{RGB}{185,208,238}% #b9d0ee
\definecolor{evidence}{RGB}{238,208,185}
\definecolor{perfil}{RGB}{208,238,185}
\definecolor{unidad}{RGB}{208,185,185} 

\usetikzlibrary{shapes,arrows}
\tikzstyle{elem} = [scale=0.7, draw, rectangle, thick, minimum height=2em,
minimum width=2em, execute at begin node={\begin{varwidth}{12em}},
  execute at end node={\end{varwidth}}]
\tikzstyle{perf} = [draw, rectangle, thick, minimum height=2em,
minimum width=2em, fill=perfil]
\tikzstyle{comp} = [draw, rectangle, thick, minimum height=2em,
minimum width=2em, fill=tableheader]
\tikzstyle{esp} = [scale=0.7, draw, rectangle, thick, minimum height=2em,
minimum width=2em, fill=tableheader]
\tikzstyle{unidad} = [scale=0.9, draw, rectangle, thick, minimum height=2em,
minimum width=2em, fill=evidence]
\tikzstyle{evid} = [scale=0.75, draw, rectangle, thick, minimum height=2em,
minimum width=2em, fill=unidad]
\tikzstyle{header} = [scale=0.8, minimum height=2em, minimum width=2em, execute at begin node={\begin{varwidth}{4em}},
  execute at end node={\end{varwidth}}]


\tikzstyle{line} = [draw, thick, -stealth, shorten >=0pt]
\usepackage{multicol}
\usepackage{wrapfig}
\usepackage{enumitem}
\renewcommand*\familydefault{\sfdefault} 
\renewcommand{\thesection}{\Roman{section}} 
\usepackage{remreset}

\makeatletter
\@removefromreset{subsection}{section}
\renewcommand\thesubsection{\arabic{subsection}}
\makeatother

\makeatletter
\@removefromreset{subsubsection}{section} % no number
\renewcommand\thesubsubsection{}
\makeatother

\usepackage{titlesec}
\titleformat{\subsubsection}[runin]
{\bfseries}{\thesubsubsection}{0em}{} % in boldface

\makeatletter
\@removefromreset{theparagraph}{paragraph} % no number
\renewcommand\theparagraph{}
\makeatother

\usepackage{titlesec}
\titleformat{\paragraph}[runin]
{\itshape}{\theparagraph}{1em}{} % in italics

\ULCornerWallPaper{1}{logos.pdf}
\setlength{\parindent}{1em}
\setlength{\parskip}{2pt}
\tcbset{
  arc = 3mm,
  colframe = headerframe,
  colback = headercontent,
  fonttitle=\sffamily
}

\usepackage{amssymb}
\newcommand{\yes}{\makebox[0pt][l]{$\square$}{\raisebox{0.1\height}{$\times$}}}
\newcommand{\no}{\makebox[0pt][l]{$\square$}{\raisebox{0.1\height}{\phantom{$\times$}}}}
\usepackage{enumitem}
\titleformat{\section}{\normalfont\large\bfseries}{\thesection.}{4pt}{}
\titleformat{\subsection}[runin]{\normalfont\bfseries}{\thesubsection.}{4pt}{}


\newcommand{\UANL}{UNIVERSIDAD AUT\'{O}NOMA DE NUEVO LE\'{O}N}
\newcommand{\uanl}{Universidad Aut\'{o}noma de Nuevo Le\'{o}n}
\newcommand{\fime}{Facultad de Ingenier\'{\i}a Mec\'{a}nica y El\'{e}ctrica}
\newcommand{\maestria}{Maestr\'{\i}a en Ciencias de la Ingenier\'{\i}a con Orientaci\'{o}n en Sistemas}
\newcommand{\doctorado}{Doctorado en Ingenier\'{\i}a de Sistemas}
\newcommand{\PA}{PROGRAMA ANAL\'{I}TICO}

% LGAC
\newcommand{\seys}{Sistemas estoc\'{a}sticos y simulaci\'{o}n}
\newcommand{\mado}{M\'{e}todos avanzados de optimizaci\'{o}n}
\newcommand{\odsi}{Optimizaci\'{o}n de sistemas industriales}

% AREAS CURRICULARES
\newcommand{\fb}{formaci\'{o}n b\'{a}sica} % maestria
\newcommand{\fa}{formaci\'{o}n avanzada} % maestria
\newcommand{\fr}{formaci\'{o}n} % doctorado
\renewcommand{\div}{divulgaci\'{o}n}
\newcommand{\inv}{investigaci\'{o}n}
\renewcommand{\pi}{producto integrador}
\newcommand{\da}{de aplicaci\'{o}n}
\renewcommand{\le}{libre elecci\'{o}n}

% CLAVES DE LAS UNIDADES

\newcommand{\algm}{PM109} % analisis y diseno de algoritmos
\newcommand{\algd}{PD109}
\newcommand{\aprm}{PM134} % aprendizaje automatico
\newcommand{\aprd}{PD134}
\newcommand{\dsm}{PM123} % ciencia de datos
\newcommand{\dsd}{PD123}
\newcommand{\ccm}{PM135} % complejidad computacional
\newcommand{\ccd}{PD135} 
\newcommand{\iam}{PM101} % inteligencia artificial
\newcommand{\iad}{PD101}
\newcommand{\cvm}{PM124} % procesamiento de imagenes y vision computacional
\newcommand{\cvd}{PD124}
\renewcommand{\sim}{PM201} % seminarios
\newcommand{\siim}{PM202}
\newcommand{\sid}{PD201}
\newcommand{\siid}{PD202}
\newcommand{\siiid}{PD203}
\newcommand{\sivd}{PD204}
\newcommand{\svd}{PD205}
\newcommand{\svid}{PD206}
\newcommand{\sviid}{PD207}
\newcommand{\sviiid}{PD208}
\newcommand{\ssm}{PM105} % simulacion de sistemas
\newcommand{\ssd}{PD105}
\newcommand{\tim}{PM501} % tesis
\newcommand{\tiim}{PM502}
\newcommand{\tid}{PD501}
\newcommand{\tiid}{PD502}
\newcommand{\tiiid}{PD503}
\newcommand{\tivd}{PD504}
\newcommand{\tvd}{PD505}
\newcommand{\tvid}{PD506}
\newcommand{\tviid}{PD507}
\newcommand{\tviiid}{PD508} 

\newcommand{\narturo}{095012}
\newcommand{\nelisa}{096633}
\newcommand{\ncesar}{092038}
\newcommand{\nangy}{102662}
\newcommand{\nvincent}{102947}
\newcommand{\nada}{060581}
\newcommand{\niris}{103743}
\newcommand{\nsara}{100546}
\newcommand{\nroger}{090969}
\newcommand{\nigor}{093179}
\newcommand{\nromeo}{100959}
\newcommand{\nferny}{095808}
\newcommand{\arturo}{Dr.\ Jos\'{e} Arturo Berrones Santos}
\newcommand{\elisa}{Dra.\ Satu Elisa Schaeffer}
\newcommand{\cesar}{Dr.\ C\'{e}sar Emilio Villarreal Rodr\'{\i}guez}
\newcommand{\angy}{Dra.\ Mar\'{\i}a Ang\'{e}lica Salazar Aguilar}
\newcommand{\vincent}{Dr.\ Vincent Boyer}
\newcommand{\ada}{Dra.\ Ada Margarita \'{A}lvarez Socarr\'{a}s}
\newcommand{\iris}{Dra.\ Iris Abril Mart\'{\i}nez Salazar}
\newcommand{\sara}{Dra.\ Sara Ver\'{o}nica S\'{a}nchez Rodr\'{\i}guez}
\newcommand{\roger}{Dr.\ Roger Zirahu\'{e}n R\'{\i}os Mercado}
\newcommand{\igor}{Dr.\ Igor Litvinchev}
\newcommand{\romeo}{Dr.\ Romeo S\'{a}nchez Nigenda}
\newcommand{\ferny}{Dr.\ Fernando L\'{o}pez Irarragorri}
\newcommand{\simon}{Dr.\ Sim\'{o}n Mart\'{\i}nez Mart\'{\i}nez} 

\newcommand{\subdirector}{\simon}
\newcommand{\coordinador}{\cesar}



\begin{document}
\begin{tcolorbox}
  \begin{center}
    {\bf UNIVERSIDAD AUT\'{O}NOMA DE NUEVO LE\'{O}N}

    \medskip

    {\bf Facultad de Ingenier\'{\i}a Mec\'{a}nica y El\'{e}ctrica}

    \medskip
    
    {\bf PE} \underline{\bf Doctorado en Ingenier\'{\i}a de Sistemas}

    \medskip

    \underline{PROGRAMA ANAL\'{I}TICO}
  \end{center}
\end{tcolorbox}

\section{Datos de Identificaci\'{o}n de la Unidad de Aprendizaje:}
\subsection{Nombre:} \fbox{Aprendizaje autom\'{a}tico}
\subsection{Frecuencia semanal:} horas de trabajo presencial \fbox{4}
\subsection{Horas de trabajo extra aula por semana:} \fbox{2}
\subsection{Modalidad:} \yes~Escolarizada \no~No escolarizada \no~Mixto
\subsection{Per\'{\i}odo acad\'{e}mico:} \yes~Semestral
\no~Tetramestral \no~Modular
\subsection{LGAC:} \underline{Sistemas estoc\'{a}sticos y simulaci\'{o}n}
\subsection{Ubicaci\'{o}n semestral:} \underline{1 al 8}
\subsection{\'{A}rea curricular:} \underline{Formaci\'{o}n, de libre elecci\'{o}n}
\subsection{Cr\'{e}ditos:} \underline{4}
\subsection{Requisito:} \underline{Ninguno}
\subsection{Fecha de elaboraci\'{o}n:} \underline{20/01/2010}
\subsection{Fecha de la \'{u}ltima actualizaci\'{o}n:} \underline{17/10/2020}
\subsection{Responsable (es) del dise\~{n}o:}
\begin{itemize}[label={}]
\item \underline{\narturo~\arturo}
\item \underline{\nelisa~\elisa}
\end{itemize}

\newpage

\section{Presentaci\'{o}n:}

La teor\'{\i}a del aprendizaje autom\'{a}tico o aprendizaje de m\'{a}quinas (del
ingl\'{e}s, {\em machine learning}) es el subcampo de las ciencias de la
computaci\'{o}n y una rama de la inteligencia artificial cuyo objetivo es
desarrollar t\'{e}cnicas que permitan a las computadoras aprender.



\section{Prop\'{o}sito(s):}

Crear programas capaces de generalizar comportamientos a partir de una
informaci\'{o}n suministrada en forma de ejemplos.


\section{Competencias del perfil de egreso:}

\subsection{Competencias del perfil de egreso}

P1) Realizar investigaci\'{o}n original y resolver problemas en el \'{a}rea de
toma de decisiones en ambientes operativos que pueden ser din\'{a}micos o
inciertos para lograr una asignaci\'{o}n m\'{a}s efectiva de recursos y
decidir el curso de acci\'{o}n \'{o}ptimo para lograr objetivos establecidos.


P2) Resolver problemas concretos en sistemas de la industria, la
academia o el sector p\'{u}blico en base a las herramientas de la toma de
decisiones con bases cient\'{\i}ficas para lograr el mejor dise\~{n}o,
an\'{a}lisis, planeaci\'{o}n o gesti\'{o}n de dichos sistemas.

  
\subsection{Competencias generales a que se vincula la Unidad de
    Aprendizaje:}

  La unidad se vincula con las siguientes competencias generales:

  
  \begin{tabular}{|p{13cm}|p{13mm}|}
    \hline
    {\em Declaraci\'{o}n de la competencia general vinculada a la unidad
    de aprendizaje}
    & {\em Evidencia} \\ \hline
    C2) Utiliza los lenguajes l\'{o}gico, formal, matem\'{a}tico, ic\'{o}nico, verbal
y no verbal de acuerdo a su etapa de vida en el \'{a}rea de las ciencias
para comprender, interpretar y expresar ideas, sentimientos, teor\'{\i}as y
corrientes de pensamiento con un enfoque ecum\'{e}nico.
 & Tareas \\ \hline
    C3) Maneja las tecnolog\'{\i}as de la informaci\'{o}n de acuerdo a los usos del
campo de las ciencias y la comunicaci\'{o}n como herramientas para el
acceso a la informaci\'{o}n y su transformaci\'{o}n en conocimiento, as\'{\i} como
para el aprendizaje y trabajo colaborativo con t\'{e}cnicas de vanguardia
que le permitan su participaci\'{o}n constructiva en la sociedad.
 & Tareas \\ \hline
    C5) Emplea pensamiento l\'{o}gico, cr\'{\i}tico, creativo y propositivo,
siguiendo los modelos de pensamiento cient\'{\i}fico para analizar
fen\'{o}menos naturales y sociales que le permitan tomar decisiones
pertinentes en su \'{a}mbito de influencia con responsabilidad social.
 & Tareas \\ \hline
    C12) Construye propuestas innovadoras basadas en la comprensi\'{o}n
hol\'{\i}stica de la realidad incluyendo los diferentes campos cient\'{\i}ficos
para contribuir a superar los retos del ambiente global
interdependiente.
 & Tareas, proyecto \\ \hline
    C13) Asume el liderazgo que le ha otorgado el dominio de las ciencias,
comprometido con las necesidades sociales y profesionales para
promover el cambio social pertinente.
 & Tareas, proyecto \\ \hline
    \end{tabular}

\newpage
    
\subsection{Competencias espec\'{\i}ficas y nivel de dominio a que se vincula la unidad de aprendizaje:}

  La unidad se vincula con las siguientes competencias espec\'{\i}ficas:

\begin{tabular}{|p{30mm}|p{5mm}|p{5mm}|p{3cm}|p{12mm}|p{30mm}|p{12mm}|p{5mm}|p{12mm}|}
  \hline
  {\em Competencia Espec\'{\i}fica}
  & \rotatebox{90}{Nivel I Inicial}
  & \rotatebox{90}{Evidencia}         
  & \rotatebox{90}{Nivel I II B\'{a}sico}
  & \rotatebox{90}{Evidencia}
  & \rotatebox{90}{Nivel III Aut\'{o}nomo}
  & \rotatebox{90}{Evidencia}
  & \rotatebox{90}{Nivel IV Estrat\'{e}gico\phantom{xxx}}
  & \rotatebox{90}{Evidencia} \\ \hline
  E1) Realizar investigaci\'{o}n original y resolver problemas en el \'{a}rea de
toma de decisiones en ambientes operativos que pueden ser din\'{a}micos o
inciertos para lograr una asignaci\'{o}n m\'{a}s efectiva de recursos y
decidir el curso de acci\'{o}n \'{o}ptimo para lograr objetivos establecidos.

  & & & Resuelve problemas de libro de texto en el \'{a}rea
        de toma de decisione s con bases cient\'{\i}ficas.
  &
    Tareas.
  &
    Encuentra
    soluciones
    para
    la
    consecuci\'{o}n
    de
    objetivos
    establecidos
    en
    un
    sistema
    dado,
    revisando
    literatura
    cient\'{\i}fica
    de
    frontera.
  &
    Tareas.
  &&  \\ \hline
  
\end{tabular}

\section{Representaci\'{o}n gr\'{a}fica:}



\begin{tikzpicture}[scale=1, auto]
  \matrix[row sep=1cm, column sep=7mm]{
      & 
      \node[elem] (ide) {Identificar un problema de inter\'{e}s que se
        busca resolver};
      \\
      \node[elem] (lit) {Identificar y revisar literatura cient\'{\i}fica relacionada relevante};
      &
      \node[elem] (des) {Desarrollar un m\'{e}todo de ...};
      &
      \node[elem] (her) {Revisar herramientas computacionales
        existentes para implementar el m\'{e}todo};
      \\
      \node[elem] (for) {Formular hip\'{o}tesis parala validaci\'{o}n del
        modelo propuesto};
      &
      \node[elem] (imp) {Implementar el m\'{e}todo con las herramientas seleccionadas};
      &
      \node[elem] (sel) {Seleccionar las herramientas seg\'{u}n las caracter\'{\i}sticas del m\'{e}todo};
      \\
      & \node[elem] (dis) {Dise\~{n}ar y realizar experimentos};
      \\
      \node[elem] (ana) {Analizar los resultados de los experimentos y contrastarlas con las hip\'{o}tesis};
      & & 
      \node[elem] (doc) {Documentar el proceso conforme al estilo de
        redacci\'{o}n cient\'{\i}fica};
      \\
      & 
      \node[elem] (apl) {Aplicar visualizaci\'{o}n cient\'{\i}fica a los resultados};
      & 
      \node[elem] (exp) {Discutir entre pares los m\'{e}todos y los
        resultados obtenidos};
      \\
    };
    \draw [line] (ide) -- (lit);
    \draw [line] (ide) -- (des);
    \draw [line] (ide) -- (her);
    \draw [line] (lit) -- (des);
    \draw [line] (lit) -- (for);
    \draw [line] (des) -- (imp);
    \draw [line] (her) -- (sel);
    \draw [line] (sel) -- (imp);
    \draw [line] (for) -- (dis);
    \draw [line] (imp) -- (dis);
    \draw [line] (dis) -- (ana);
    \draw [line] (dis) -- (apl);
    \draw [line] (ana) -- (doc);
    \draw [line] (apl) -- (doc);
    \draw [line] (doc) -- (exp);
  \end{tikzpicture}

\newpage
  
  \section{Estructuraci\'{o}n en cap\'{\i}tulos, etapas o fases de la unidad de
  aprendizaje:}

\subsection{Desarrollo de las fases de la Unidad de Aprendizaje:}

Se cubren los principios te\'{o}ricos del aprendizaje
autom\'{a}tico. Desarrollar habilidades en la resoluci\'{o}n en casos
pr\'{a}cticos concretos. Se necesita contar con un buen entendimiento de varios los conceptos
matem\'{a}ticos, especialmente de matem\'{a}ticas discretas y probabilidad, o
en el caso contrario, estar preparado a estudiarlos seg\'{u}n
necesidad. Tambi\'{e}n se necesita conocimiento de programaci\'{o}n.
La sesiones son de cuatro horas cada una y son veinte semanas en total.
\begin{enumerate}[itemsep=-3pt]
\item{Introducci\'{o}n; selecci\'{o}n de temas de proyecto}
\item{Aprendizaje supervisado}
\item{Regresi\'{o}n lineal}
\item{Clasificaci\'{o}n lineal}
\item{Expansi\'{o}n y regularizaci\'{o}n}
\item{Suavizaci\'{o}n con n\'{u}cleos}
\item{Evaluaci\'{o}n y selecci\'{o}n de modelos}
\item{Inferencia y promediaci\'{o}n}
\item{Modelos aditivos y \'{a}rboles}
\item{Impulso y \'{a}rboles aditivos}
\item{Redes neuronales}
\item{M\'{a}quinas de soporte vectorial}
\item{Vecinos m\'{a}s cercanos}
\item{Aprendizaje no supervisado}
\item{B\'{o}sques aleatorios}
\item{Aprendizaje colectivo}
\item{Modelos gr\'{a}ficos no dirigidos}
\item{Problemas de alta dimensionalidad}
\item{Presentaciones de proyectos}
\item{Revisi\'{o}n de portafolios de evidencia}
\end{enumerate}

{\em Elementos de competencia:}

\begin{tabular}{|p{28mm}|p{30mm}|p{30mm}|p{30mm}|p{30mm}|}
  \hline
  Evidencias de aprendizaje 
  & Criterios de desempe\~{n}o
  & Actividades de aprendizaje
  & Contenidos
  & Recursos \\ \hline

  Reporte escrito y/ c\'{o}digo de la implementaci\'{o}n de un algoritmo de
  aprendizaje autom\'{a}tico.
  & Calidad de la redacci\'{o}n cient\'{\i}fica del reporte; 
    precisi\'{o}n del algoritmo propuesto;
    eficiencia de la implementaci\'{o}n del algoritmo;
    cobertura de la experimentaci\'{o}n.
  & Experimentaci\'{o}n con ejemplos; lectura de material de apoyo;
    modificaci\'{o}n de ejemplos; dise\~{n}o y ejecuci\'{o}n de experimentos;
    an\'{a}lisis y reportaje de resultados obtenidos.
  &
    M\'{e}todos diversos de aprendizaje autom\'{a}tico
  & Material en la p\'{a}gina web de la unidad y la literatura citada;
    lenguaje Python o similar; paquete {\LaTeX} para redacci\'{o}n cient\'{\i}fica;
    repositorios de GitHub. \\ \hline
  
  \end{tabular}

\newpage
  
  \section{Evaluaci\'{o}n integral de procesos y productos:}
  
  Las tareas son individuales; se recomienda estudiar juntos y
  discutir las soluciones, pero no se tolera ning\'{u}n tipo de plagio en
  absoluto, ni de otros estudiantes ni de la red ni de libros --- toda
  referencia bibliogr\'{a}fica tiene que ser apropiadamente citada. La
  entrega se realiza por un repositorio en GitHub que debe reflejar
  todas las fases del trabajo en su log correspondiente. El alumno
  selecciona su lenguaje de programaci\'{o}n para cada tarea.

  Son 17 tareas (A1--A17) que reportan avances semanales de aplicaci\'{o}n
  de la lectura de la semana para el proyecto del estudiante,
  otorgando por m\'{a}ximo 5 puntos por tarea:
  \begin{description}[itemsep=0em]
  \item[NP]{= tarea omitida}
  \item[5]{= excede lo que se esperaba}
  \item[4]{= cumple con lo que se esperaba}
  \item[3]{= d\'{e}bil en alcance y/o calidad}
  \item[2]{= d\'{e}bil en ambos alcance y calidad}
  \item[1]{= sin contribuciones o m\'{e}ritos aunque fue entregada}
  \item[0]{= completamente inadecuado en alzance y calidad}
  \end{description}
  El proyecto final (A18) otorga un m\'{a}ximo de 15 puntos, evaluados en los
  siguientes rubros
  \begin{enumerate}[itemsep=0em]
  \item{Variedad de t\'{e}cnicas de aprendizaje autom\'{a}tico empleadas}
  \item{Cobertura y validez de la experimentaci\'{o}n}
  \item{Claridad y relevancia de los resultados computacionales}    
  \item{Calidad de visualizaci\'{o}n cient\'{\i}fica}
  \item{Calidad de redacci\'{o}n cient\'{\i}fica}
  \end{enumerate}
  con la escala:
  \begin{description}[itemsep=0em]
  \item[3]{= cumple con lo que se esperaba}
  \item[2]{= d\'{e}bil en alcance y/o calidad}
  \item[1]{= d\'{e}bil en ambos alcance y calidad}
  \item[0]{= inadecuado en alzance y calidad}
  \end{description}


  No habr\'{a} examen.


  
  Ponderaci\'{o}n espec\'{\i}fica:

  \scalebox{0.85}{
\begin{tabular}{|c|cccccccccccccccccc|c|}
  \hline

 \rotatebox{90}{{\bf Actividad\phantom{xx}}} & A1 & A2 & A3 & A4 & A5 & A6 & A7 & A8 & A9 & A10 &
                                                                      A11
  & A12 & A13 & A14 & A15 & A16 & A17 & A18 & Total % Poner el nombre de la actividad
  \\
  \hline
  \rotatebox{90}{{\bf Ponderaci\'{o}n\phantom{xx}}}
                                             & 5\% & 5\% & 5\% &
                                                                 5\% &
                                                                       5\% & 5\% & 5\% & 5\%
                                                                                     & 5\%
                                                                                          & 5\%
                                                                                                & 5\%
  & 5\% 
        & 5\% & 5\% & 5\% & 5\% & 5\% & 15\% & 100\%		
	\\ \hline
\end{tabular}}

\newpage

\section{Producto integrador de aprendizaje de la unidad:}

\subsection{Producto integrador de Aprendizaje:} Portafolio en un
repositorio digital p\'{u}blico.

\section{Fuentes de apoyo y consulta:}

\subsection{Fuentes de apoyo y consulta}

\subsubsection{B\'{a}sicas}

 \begin{itemize}[itemsep=0em]

 \item{Hastie, Trevor, Tibshirani, Robert. \& Friedman, Jerome: {\em
       The Elements of Statistical Learning: Data Mining, Inference,
       and Prediction}. Springer; 2nd edition, 2016.}

   \item{Michalski, Ryszard S., Jaime G. Carbonell \& Tom M. Mitchell,
   eds. {\em Machine learning: An artificial intelligence
     approach}. Springer Science \& Business Media, 2013.}

 \item{Witten, Ian H., et al. {\em Data Mining: Practical machine learning
     tools and techniques}. Morgan Kaufmann, 2016.}

 \item{Marsland, Stephen. {\em Machine learning: an algorithmic
       perspective}. CRC press, 2015.}
   
\end{itemize}

\subsubsection{Complementarias}

Art\'{\i}culos cient\'{\i}ficos especializados.

\newpage
 
{\bf Autoriz\'{o}:} \coordinador
  
  \begin{center}
  {\sc Alere Flammam Veritatis}
  
  Ciudad Universitaria, \today

\vspace*{4cm}
  
  \begin{tabular}{p{6cm}cp{7cm}}
    \cline{1-1}
    \cline{3-3}    
    {\bf \coordinador} &
                                                            \phantom{xxx} &{\bf Vo.\ Bo.\ \subdirector} \\
    Coordinador Acad\'{e}mico &  &Subdirector de Estudios de Posgrado \\
    Posgrado en Ingenier\'{\i}a de Sistemas & & Facultad de Ingenier\'{\i}a Mec\'{a}nica y El\'{e}ctrica
                                                                   
  \end{tabular}
\end{center}
  
  
  
\end{document}