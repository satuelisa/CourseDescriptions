\documentclass[10 pt]{article}
\usepackage{wallpaper}
\usepackage[spanish, mexico]{babel}  
\usepackage{color,colortbl}
\usepackage[T1]{fontenc}
\usepackage{fancyhdr} % before geometry
\usepackage[letterpaper,left=18mm,right=18mm,headheight=5mm,headsep=40mm,top=52mm,bottom=22mm]{geometry}
\usepackage{graphicx}
\usepackage{rotating}
\usepackage[latin1]{inputenc}
\usepackage{hyperref}
\usepackage{graphics}
\usepackage{varwidth}
\usepackage{tikz}

\usepackage{tcolorbox}
\definecolor{headerframe}{RGB}{177,178,177} % #b1b2b1
\definecolor{headercontent}{RGB}{239,239,239} % #efefef
\definecolor{tableheader}{RGB}{185,208,238}% #b9d0ee
\definecolor{evidence}{RGB}{238,208,185}
\definecolor{perfil}{RGB}{208,238,185}
\definecolor{unidad}{RGB}{208,185,185} 

\usetikzlibrary{shapes,arrows}
\tikzstyle{elem} = [scale=0.7, draw, rectangle, thick, minimum height=2em,
minimum width=2em, execute at begin node={\begin{varwidth}{12em}},
  execute at end node={\end{varwidth}}]
\tikzstyle{perf} = [draw, rectangle, thick, minimum height=2em,
minimum width=2em, fill=perfil]
\tikzstyle{comp} = [draw, rectangle, thick, minimum height=2em,
minimum width=2em, fill=tableheader]
\tikzstyle{esp} = [scale=0.7, draw, rectangle, thick, minimum height=2em,
minimum width=2em, fill=tableheader]
\tikzstyle{unidad} = [scale=0.9, draw, rectangle, thick, minimum height=2em,
minimum width=2em, fill=evidence]
\tikzstyle{evid} = [scale=0.75, draw, rectangle, thick, minimum height=2em,
minimum width=2em, fill=unidad]
\tikzstyle{header} = [scale=0.8, minimum height=2em, minimum width=2em, execute at begin node={\begin{varwidth}{4em}},
  execute at end node={\end{varwidth}}]


\tikzstyle{line} = [draw, thick, -stealth, shorten >=0pt]
\usepackage{multicol}
\usepackage{wrapfig}
\usepackage{enumitem}
\renewcommand*\familydefault{\sfdefault} 
\renewcommand{\thesection}{\Roman{section}} 
\usepackage{remreset}

\makeatletter
\@removefromreset{subsection}{section}
\renewcommand\thesubsection{\arabic{subsection}}
\makeatother

\makeatletter
\@removefromreset{subsubsection}{section} % no number
\renewcommand\thesubsubsection{}
\makeatother

\usepackage{titlesec}
\titleformat{\subsubsection}[runin]
{\bfseries}{\thesubsubsection}{0em}{} % in boldface

\makeatletter
\@removefromreset{theparagraph}{paragraph} % no number
\renewcommand\theparagraph{}
\makeatother

\usepackage{titlesec}
\titleformat{\paragraph}[runin]
{\itshape}{\theparagraph}{1em}{} % in italics

\ULCornerWallPaper{1}{logos.pdf}
\setlength{\parindent}{1em}
\setlength{\parskip}{2pt}
\tcbset{
  arc = 3mm,
  colframe = headerframe,
  colback = headercontent,
  fonttitle=\sffamily
}

\usepackage{amssymb}
\newcommand{\yes}{\makebox[0pt][l]{$\square$}{\raisebox{0.1\height}{$\times$}}}
\newcommand{\no}{\makebox[0pt][l]{$\square$}{\raisebox{0.1\height}{\phantom{$\times$}}}}
\usepackage{enumitem}
\titleformat{\section}{\normalfont\large\bfseries}{\thesection.}{4pt}{}
\titleformat{\subsection}[runin]{\normalfont\bfseries}{\thesubsection.}{4pt}{}


\newcommand{\UANL}{UNIVERSIDAD AUT\'{O}NOMA DE NUEVO LE\'{O}N}
\newcommand{\uanl}{Universidad Aut\'{o}noma de Nuevo Le\'{o}n}
\newcommand{\fime}{Facultad de Ingenier\'{\i}a Mec\'{a}nica y El\'{e}ctrica}
\newcommand{\maestria}{Maestr\'{\i}a en Ciencias de la Ingenier\'{\i}a con Orientaci\'{o}n en Sistemas}
\newcommand{\doctorado}{Doctorado en Ingenier\'{\i}a de Sistemas}
\newcommand{\PA}{PROGRAMA ANAL\'{I}TICO}

% LGAC
\newcommand{\seys}{Sistemas estoc\'{a}sticos y simulaci\'{o}n}
\newcommand{\mado}{M\'{e}todos avanzados de optimizaci\'{o}n}
\newcommand{\odsi}{Optimizaci\'{o}n de sistemas industriales}

% AREAS CURRICULARES
\newcommand{\fb}{formaci\'{o}n b\'{a}sica} % maestria
\newcommand{\fa}{formaci\'{o}n avanzada} % maestria
\newcommand{\fr}{formaci\'{o}n} % doctorado
\renewcommand{\div}{divulgaci\'{o}n}
\newcommand{\inv}{investigaci\'{o}n}
\renewcommand{\pi}{producto integrador}
\newcommand{\da}{de aplicaci\'{o}n}
\renewcommand{\le}{libre elecci\'{o}n}

% CLAVES DE LAS UNIDADES

\newcommand{\algm}{PM109} % analisis y diseno de algoritmos
\newcommand{\algd}{PD109}
\newcommand{\aprm}{PM134} % aprendizaje automatico
\newcommand{\aprd}{PD134}
\newcommand{\dsm}{PM123} % ciencia de datos
\newcommand{\dsd}{PD123}
\newcommand{\ccm}{PM135} % complejidad computacional
\newcommand{\ccd}{PD135} 
\newcommand{\iam}{PM101} % inteligencia artificial
\newcommand{\iad}{PD101}
\newcommand{\cvm}{PM124} % procesamiento de imagenes y vision computacional
\newcommand{\cvd}{PD124}
\renewcommand{\sim}{PM201} % seminarios
\newcommand{\siim}{PM202}
\newcommand{\sid}{PD201}
\newcommand{\siid}{PD202}
\newcommand{\siiid}{PD203}
\newcommand{\sivd}{PD204}
\newcommand{\svd}{PD205}
\newcommand{\svid}{PD206}
\newcommand{\sviid}{PD207}
\newcommand{\sviiid}{PD208}
\newcommand{\ssm}{PM105} % simulacion de sistemas
\newcommand{\ssd}{PD105}
\newcommand{\tim}{PM501} % tesis
\newcommand{\tiim}{PM502}
\newcommand{\tid}{PD501}
\newcommand{\tiid}{PD502}
\newcommand{\tiiid}{PD503}
\newcommand{\tivd}{PD504}
\newcommand{\tvd}{PD505}
\newcommand{\tvid}{PD506}
\newcommand{\tviid}{PD507}
\newcommand{\tviiid}{PD508} 

\newcommand{\narturo}{095012}
\newcommand{\nelisa}{096633}
\newcommand{\ncesar}{092038}
\newcommand{\nangy}{102662}
\newcommand{\nvincent}{102947}
\newcommand{\nada}{060581}
\newcommand{\niris}{103743}
\newcommand{\nsara}{100546}
\newcommand{\nroger}{090969}
\newcommand{\nigor}{093179}
\newcommand{\nromeo}{100959}
\newcommand{\nferny}{095808}
\newcommand{\arturo}{Dr.\ Jos\'{e} Arturo Berrones Santos}
\newcommand{\elisa}{Dra.\ Satu Elisa Schaeffer}
\newcommand{\cesar}{Dr.\ C\'{e}sar Emilio Villarreal Rodr\'{\i}guez}
\newcommand{\angy}{Dra.\ Mar\'{\i}a Ang\'{e}lica Salazar Aguilar}
\newcommand{\vincent}{Dr.\ Vincent Boyer}
\newcommand{\ada}{Dra.\ Ada Margarita \'{A}lvarez Socarr\'{a}s}
\newcommand{\iris}{Dra.\ Iris Abril Mart\'{\i}nez Salazar}
\newcommand{\sara}{Dra.\ Sara Ver\'{o}nica S\'{a}nchez Rodr\'{\i}guez}
\newcommand{\roger}{Dr.\ Roger Zirahu\'{e}n R\'{\i}os Mercado}
\newcommand{\igor}{Dr.\ Igor Litvinchev}
\newcommand{\romeo}{Dr.\ Romeo S\'{a}nchez Nigenda}
\newcommand{\ferny}{Dr.\ Fernando L\'{o}pez Irarragorri}
\newcommand{\simon}{Dr.\ Sim\'{o}n Mart\'{\i}nez Mart\'{\i}nez} 

\newcommand{\subdirector}{\simon}
\newcommand{\coordinador}{\cesar}


\begin{document}

\pagestyle{fancy}
\renewcommand{\headrulewidth}{0pt}
\fancyhf{}
\fancyhead[L]{}
\fancyhead[C]{}
\fancyhead[R]{IT-8-SPG-02-R03}
\fancyfoot[L]{Revisi\'{o}n: 1 \\
  Vigente a partir del: 01 de agosto del 2016}
\fancyfoot[C]{}
\fancyfoot[R]{P\'{a}gina~\thepage~de~\pageref*{final}}


\fancypagestyle{plain}{%
\fancyhf{}
\fancyhead[L]{}
\fancyhead[C]{}
\fancyhead[R]{IT-8-SPG-02-R03}
\fancyfoot[L]{Revisi\'{o}n: 1 \\
  Vigente a partir del: 01 de agosto del 2016}
\fancyfoot[C]{}
\fancyfoot[R]{}}

\begin{tcolorbox}
  \begin{center}

    {\bf \UANL}

    \medskip

    {\bf \fime}

    \medskip
    
    {\bf PE} \underline{\bf \maestria}

    \medskip

    \underline{\PA}

  \end{center}
\end{tcolorbox}

\section{Datos de Identificaci\'{o}n de la Unidad de Aprendizaje:}
\subsection{Nombre:} \fbox{Seminario 1}
\subsection{Frecuencia semanal:} horas de trabajo presencial \fbox{4}
\subsection{Horas de trabajo extra aula por semana:} \fbox{5}
\subsection{Modalidad:} \yes~Escolarizada \no~No escolarizada \no~Mixto
\subsection{Per\'{\i}odo acad\'{e}mico:} \yes~Semestral
\no~Tetramestral \no~Modular
\subsection{LGAC:} \underline{Optimizaci\'{o}n de sistemas industriales}
\subsection{Ubicaci\'{o}n semestral:} \underline{3}
\subsection{\'{A}rea curricular:} \underline{Formaci\'{o}n, de libre elecci\'{o}n}
\subsection{Cr\'{e}ditos:} \underline{6}
\subsection{Requisito:} \underline{Ninguno}
\subsection{Fecha de elaboraci\'{o}n:} \underline{20/01/2010}
\subsection{Fecha de la \'{u}ltima actualizaci\'{o}n:} \underline{09/06/2020}
\subsection{Responsable (es) del dise\~{n}o:}
\begin{itemize}[label={}]
\item \underline{\ncesar~\cesar}
\item \underline{\nelisa~\elisa}
\end{itemize}
\newpage
\section{Presentaci\'{o}n:}

Se busca exponer al estudiante a la atm\'{o}sfera de presentaci\'{o}n y
discusi\'{o}n de temas de investigaci\'{o}n. En el seminario se presentar\'{a}n
expositores entre los que se incluyen investigadores externos de
reconocida calidad, investigadores de la UANL y estudiantes.


\section{Prop\'{o}sito(s):}

Familiarizar el estudiante con la exposici\'{o}n de trabajos cient\'{\i}ficos.


\section{Competencias del perfil de egreso:}
\subsection{Competencias del perfil de egreso}

P1) Realizar investigaci\'{o}n original y resolver problemas en el \'{a}rea de
toma de decisiones en ambientes operativos que pueden ser din\'{a}micos o
inciertos para lograr una asignaci\'{o}n m\'{a}s efectiva de recursos y
decidir el curso de acci\'{o}n \'{o}ptimo para lograr objetivos establecidos.


P2) Resolver problemas concretos en sistemas de la industria, la
academia o el sector p\'{u}blico en base a las herramientas de la toma de
decisiones con bases cient\'{\i}ficas para lograr el mejor dise\~{n}o,
an\'{a}lisis, planeaci\'{o}n o gesti\'{o}n de dichos sistemas.


P3) Establecer comunicaci\'{o}n con los disGntos sectores de la
sociedad a fin de establecer proyectos estrat\'{e}gicos en las
distintas disciplinas de la ingenier\'{\i}a de sistemas y crear la
cultura de la creaci\'{o}n de riqueza basada en el conocimiento.

  
\subsection{Competencias generales a que se vincula la Unidad de
    Aprendizaje:}

  La unidad se vincula con las siguientes competencias generales:

  \phantom{space}
\begin{tabular}{|p{12cm}|p{45mm}|}
  \hline
  \cellcolor{tableheader}{\em Declaraci\'{o}n de la competencia general vinculada a la unidad
  de aprendizaje}
  & \cellcolor{tableheader}{\em Evidencia} \\ \hline

  C4) Domina su lengua materna en forma oral y escrita con correcci\'{o}n,
relevancia, oportunidad y \'{e}tica adaptando su mensaje a la situaci\'{o}n o
contexto, para la transmisi\'{o}n
 & Asistencia y organizaci\'{o}n del seminario \\ \hline
  C7) Elabora propuestas acad\'{e}micas y profesionales inter, multi y
transdisciplinarias de acuerdo a las mejores pr\'{a}cticas mundiales del
\'{a}rea cient\'{\i}fica en la que trabaja para fomentar y consolidar el
trabajo colaborativo.
 & Asistencia y organizaci\'{o}n del seminario  \\ \hline
  C9) Mantiene una actitud de compromiso y respeto hacia la diversidad
de pr\'{a}cticas sociales y culturales que reafirman el principio de
integraci\'{o}n en el contexto local, nacional e internacional con la
finalidad de promover ambientes de convivencia pac\'{\i}fica sobre todo
trat\'{a}ndose de los adelantos cient\'{\i}ficos.
 & Asistencia y organizaci\'{o}n del seminario  \\ \hline
  \end{tabular}

\newpage
    
\subsection{Competencias espec\'{\i}ficas y nivel de dominio a que se vincula la unidad de aprendizaje:}

  La unidad se vincula con las siguientes competencias espec\'{\i}ficas:

  \phantom{space}
\begin{tabular}{|p{30mm}|p{4mm}|p{4mm}|p{4mm}|p{4mm}|p{30mm}|p{22mm}|p{24mm}|p{22mm}|}
\hline
\cellcolor{tableheader}{{\em Competencia Espec\'{\i}fica}}
& \rotatebox{90}{\cellcolor{tableheader}{Nivel I Inicial}}
& \rotatebox{90}{\cellcolor{tableheader}{Evidencia}}
& \rotatebox{90}{\cellcolor{tableheader}{Nivel II B\'{a}sico}}
& \rotatebox{90}{\cellcolor{tableheader}{Evidencia}}
& \rotatebox{90}{\cellcolor{tableheader}{Nivel III Aut\'{o}nomo}}
& \rotatebox{90}{\cellcolor{tableheader}{Evidencia}}
& \rotatebox{90}{\cellcolor{tableheader}{Nivel IV Estrat\'{e}gico\phantom{xxx}}}
& \rotatebox{90}{\cellcolor{tableheader}{Evidencia}}
\\ \hline


  E2) Resolver problemas concretos en sistemas de la industria, la
academia o el sector p\'{u}blico en base a las herramientas de la toma
de decisiones con bases cient\'{\i}ficas para lograr el mejor dise\~{n}o,
an\'{a}lisis, planeaci\'{o}n o gesti\'{o}n de dichos sistemas.

  & & & & & Encuentra soluciones para la consecuci\'{o}n de objetivos establecidos
para un problema dado, revisando literatura cient\'{\i}fica de
frontera.
 & Participaci\'{o}n en los seminarios y en su organizaci\'{o}n.
  &  Establece junto con distintos sectores en la academia, la industria o
la sociedad en general, proyectos innovadores de car\'{a}cter estrat\'{e}gico.
 & Participaci\'{o}n en los seminarios y en su organizaci\'{o}n.
  \\ \hline
\end{tabular}  

\section{Representaci\'{o}n gr\'{a}fica:}

\begin{center}
\begin{tikzpicture}[scale=1, auto]
  \matrix[row sep=1cm, column sep=7mm]{

& 
\node[elem] (ide) {Identificar un fen\'{o}meno de inter\'{e}s con
  aplicaciones en la ingenier\'{\i}a de sistemas};
\\
\node[elem] (lit) {Identificar y revisar literatura cient\'{\i}fica relacionada relevante};
&
\node[elem] (des) {Identificar cient\'{\i}ficos y desarrolladores
  quienes lo estudian};
\\
\node[elem] (con) {Establecer contacto con estas personas para
  invitarles a exponer en el seminario};
&
\node[elem] (par) {Participar activamente en el seminario,
  haciendo preguntas y tomando notas};
\\
\node[elem] (ref) {Reflexionar sobre las ventajas y desventajas
  de la presentaci\'{o}n};
&  
\node[elem] (doc) {Formar un plan inicial para  una exposici\'{o}n de su propia
  investigaci\'{o}n};
\\
};
\draw [line] (ide) -- (lit);
\draw [line] (lit) -- (des);
\draw [line] (des) -- (con);
\draw [line] (con) -- (par);
\draw [line] (par) -- (ref);
\draw [line] (ref) -- (doc);
\end{tikzpicture}
\end{center}


\newpage
\section{Estructuraci\'{o}n en cap\'{\i}tulos, etapas o fases de la unidad de
  aprendizaje:}
\subsection{Desarrollo de las fases de la Unidad de Aprendizaje:}

Los estudiantes forman tres comit\'{e}s: el primero se encarga del
contacto directo con los ponentes en cuesti\'{o}n organizacional, el
segundo se encarga de la publicidad previa y la organizaci\'{o}n el d\'{\i}a
del evento y el tercero se encarga de redactar rese\~{n}as.  La sesiones
son de cuatro horas cada una y son veinte semanas en total: 16
sesiones con ponentes m\'{a}s cuatro en total para planeaci\'{o}n y
retroalimentaci\'{o}n. En las 16 sesiones con una exposici\'{o}n, un ponente o
un panel de ponentes expone un tema, seguido por una sesi\'{o}n de
preguntas.

{\em Elementos de competencia:}

\paragraph{Elementos de competencia:}

\quad

\begin{tabular}{|p{28mm}|p{30mm}|p{30mm}|p{30mm}|p{30mm}|}
  \hline
  \cellcolor{tableheader}Evidencias de aprendizaje 
  & \cellcolor{tableheader}Criterios de desempe\~{n}o
  & \cellcolor{tableheader}Actividades de aprendizaje
  & \cellcolor{tableheader}Contenidos
  & \cellcolor{tableheader}Recursos \\ \hline

Portafolio de evidencias de aportaciones y participaciones.

& Claridad del portafolio reporte; frecuencia y calidad de aportaciones.

& Participaci\'{o}n en los comit\'{e}s y las presentaciones.

& La comunicaci\'{o}n de la ciencia.

& Material en la p\'{a}gina web de la unidad y la
literatura citada; paquete {\LaTeX}, HTML, CSS, YouTube.
\\ \hline
\end{tabular}

  \section{Evaluaci\'{o}n integral de procesos y productos:}
  

  \begin{description}
  \item[Asistencia a una sesi\'{o}n:]{2 pts (1 pt si se une tarde o si se
      va temprano)}
  \item[Participaci\'{o}n en una sesi\'{o}n:]{hasta 3 pts dependiendo de la
      calidad y cantidad de las preguntas}
  \item[Contribuciones a un comit\'{e} (si aplica):]{hasta 4 pts
      semanales, dependiendo de la frecuencia y magnitud de las
      aportaciones}
  \item[Participaci\'{o}n en preparaciones (fuera del propio
    comit\'{e}):]{hasta 2 pts semanales, dependiendo de la frecuencia y
      magnitud de las aportaciones}
  \item[Participaci\'{o}n en la retroalimentaci\'{o}n del ciclo
    anterior:]{hasta 5 pts, dependiendo de la frecuencia y magnitud de
      las aportaciones}
  \item[Participaci\'{o}n en la planeaci\'{o}n del ciclo siguiente:]{hasta 5
      pts, dependiendo de la frecuencia y magnitud de las
      aportaciones}
  \end{description}

  
  Ponderaci\'{o}n espec\'{\i}fica (aproximada):

    \scalebox{0.9}{
\begin{tabular}{|c|ccccc|c|}
  \hline

\cellcolor{tableheader}{{\bf Actividad}} & A1 & A2 & A3 & A4 & A5 & Total \\
  \hline
\cellcolor{tableheader}{{\bf Ponderaci\'{o}n}}
                                             & 20\% & 30\% & 40\% &
                                                                    5\%
                                      & 5\% & 100\%
        \\ \hline
\end{tabular}}

  
  \newpage

\section{Producto integrador de aprendizaje de la unidad:}
\subsection{Producto integrador de Aprendizaje:} Portafolio de
evidencias.
  La asistencia se comprueba con captura de pantalla de la sesi\'{o}n de
  YouTube; es necesario hacer login para que se vea el usuario activo
  en la captura. La participaci\'{o}n se evidencia con la captura de
  pantalla del chat de YouTube donde se ve la pregunta. Las
  contribuciones se evidencian con capturas de pantalla de los canales
  de comunicaci\'{o}n, capturas de pantalla de los commits en
  GitHub. Comit\'{e} de publicidad adem\'{a}s incluye capturas de pantalla de
  los posts en medios sociales, Comit\'{e} de ponentes incluye capturas de
  pantalla de comunicaciones con ponentes. La hoja portada del
  portafolio incluye un resumen de los puntos evidenciados en ello.


\section{Fuentes de apoyo y consulta:}
\subsection{Fuentes de apoyo y consulta}
\subsubsection{B\'{a}sicas}

 \begin{itemize}[itemsep=0em]

\item{Scott {\sc Morgan}: {\em Speaking about Science: A Manual for
      Creating Clear Presentations}.
    Cambridge University Press; 1st edition (September 1, 2006)}

\end{itemize}

\subsubsection{Complementarias}

Art\'{\i}culos cient\'{\i}ficos especializados.

\label{final} % last page
%\newpage

%\pagestyle{plain}

%\vspace*{3cm}

%{\bf Autoriz\'{o}:} \coordinador

%\vspace*{2cm}

%  \begin{center}
%  {\sc Alere Flammam Veritatis}
  
%  Ciudad Universitaria, \today

%\vspace*{4cm}
  
%  \begin{tabular}{p{6cm}cp{7cm}}
%    \cline{1-1}
%    \cline{3-3}    
%    {\bf \coordinador}
% & \phantom{xxx} &{\bf Vo.\ Bo.\ \subdirector} \\
%    Coordinador Acad\'{e}mico &  &Subdirector de Estudios de Posgrado \\
%    Posgrado en Ingenier\'{\i}a de Sistemas & & Facultad de Ingenier\'{\i}a Mec\'{a}nica y El\'{e}ctrica
                                                                   
%  \end{tabular}
%\end{center}

\end{document}