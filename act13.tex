\section{Evaluaci\'{o}n integral de procesos y productos:}

Las tareas son individuales; se recomienda estudiar juntos y
discutir las soluciones, pero no se tolera ning\'{u}n tipo de plagio
en absoluto, ni de otros estudiantes ni de la red ni de libros ---
toda referencia bibliogr\'{a}fica tiene que ser apropiadamente
citada. La entrega se realiza por un repositorio en GitHub que debe
reflejar todas las fases del trabajo. El
alumno selecciona su lenguaje de programaci\'{o}n para cada
tarea. Son 12 tareas (A1--A12) que reportan avances semanales de
aplicaci\'{o}n de la lectura de la semana para el proyecto del
estudiante, otorgando por m\'{a}ximo 6 puntos por tarea m\'{a}s dos
puntos por dos reto adicionales asociadas a ella:
\begin{description}[itemsep=-2pt]
\item[NP]{= tarea omitida}
\item[6]{= cumple con lo que se esperaba}
\item[5]{= cumple con que se esperaba con errores menores}
\item[4]{= cumple con lo que se esperaba con algunas omisiones}
\item[3]{= d\'{e}bil en alcance y/o calidad}
\item[2]{= d\'{e}bil en ambos alcance y calidad}
\item[1]{= sin contribuciones o m\'{e}ritos aunque fue entregada}
\item[0]{= completamente inadecuado en alzance y calidad}
\end{description}
  
El proyecto final (A13) otorga un m\'{a}ximo de 30 puntos, evaluados en los
siguientes rubros:
\begin{enumerate}[itemsep=-2pt]
\item{Complejidad del problema --- problemas retadores se califican
    m\'{a}s generosamente que problemas sencillos.}
\item{Fidelidad de la
    simulaci\'{o}n --- la incorporaci\'{o}n exitosa de m\'{u}ltiples aspectos del
    fen\'{o}meno estudiado dentro de la simulaci\'{o}n otorga una mayor
    cantidad de puntos que modelos simplificados o poco realistas.}
\item{Visualizaci\'{o}n de la informaci\'{o}n --- el uso de m\'{e}todos diversos e
    informativos para graficar los datos (ambos los originales y los
    generados por la simulaci\'{o}n, igual como datos de desempe\~{n}o de la
    misma simulaci\'{o}n) es premiado en la calificaci\'{o}n.}
\item{An\'{a}lisis estad\'{\i}stico (recomendado)--- aplicaci\'{o}n de
    m\'{e}todos rigurosos de examinaci\'{o}n estad\'{\i}stica, su
    selecci\'{o}n adecuada y su interpretaci\'{o}n correcta otorgan
    m\'{a}s puntos que el mero uso de medidas triviales o la ausencia
    total de an\'{a}lisis estad\'{\i}stico formal.}
\item{Grado y utilidad de paralelismo (opcional) --- la eficiente
    paralelizaci\'{o}n de todo lo que se pueda paralelizar sin
    p\'{e}rdida de eficiencia es altamente deseable; tambi\'{e}n se
    aprecian comparasiones de desempe\~{n}o entre versiones con y sin
    paralelismo igual como el efecto del n\'{u}mero de n\'{u}cleos
    disponibles.}
\item{Claridad y calidad de la redacci\'{o}n --- ortograf\'{\i}a,
    gram\'{a}tica, puntuaci\'{o}n, posicionamiento y formato de
    ecuaciones, cuadros y figuras, estructuraci\'{o}n de frases,
    p\'{a}rrafos, secciones, organizaci\'{o}n de la discusi\'{o}n,
    igual como el formato y la extensi\'{o}n de la bibliograf\'{\i}a
    entran en juego al calificar la calidad de un manuscrito.}
\end{enumerate}

con la escala:
\begin{description}[itemsep=0em]
\item[NP]{= tarea omitida}
\item[5]{= excede lo que se esperaba}
\item[4]{= cumple con lo que se esperaba}
\item[3]{= d\'{e}bil en alcance y/o calidad}
\item[2]{= d\'{e}bil en ambos alcance y calidad}
\item[1]{= sin contribuciones o m\'{e}ritos aunque fue entregada}
\item[0]{= completamente inadecuado en alzance y calidad}
\end{description}


Cada pr\'{a}ctica contiene una tarea que se califica en una escala de
0 a 10. La versi\'{o}n b\'{a}sica de la tarea puede otorgar hasta seis
puntos; hay dos retos opcionales para subir la calificaci\'{o}n hasta
ocho (un reto cumplido --- cumplimiento parcial permite subir hasta
siete) o diez (ambos retos cumplidos --- cumplimiento parcial permite
subir hasta nueve) seg\'{u}n la cantidad de retos cumplidos. En
algunas tareas el segundo reto depende del primero, mientras en otras
se pueden llevar a cabo de manera independiente.  Los puntos obtenidos
de las tareas se suman directamente con los puntos obtenidos del
proyecto (m\'{a}ximo 30) para formar la calificaci\'{o}n final de la
siguiente manera: $C$ es el m\'{\i}nimo entre $T + P$ y 100, donde $T$
es la suma de los puntos de las tareas y $P$ es el puntaje del
proyecto.  Calificaciones $C \geq 80$ son aprobatorias. No se otorgan
fracciones de puntos en ninguna actividad por lo cual no se requiere
ninguna regla de redondeo. No hay ex\'{a}menes.

\paragraph{Ponderaci\'{o}n espec\'{\i}fica}

\quad

\scalebox{0.9}{
  \begin{tabular}{|c|ccccccccccccc|c|}
    \hline
    
    \rotatebox{90}{\cellcolor{tableheader}{\bf Actividad\phantom{xx}}}
    & A1
    & A2
    & A3
    & A4
    & A5
    & A6
    & A7
    & A8
    & A9
    & A10
    & A11
    & A12
    & A13
    & Total 
    \\
    \hline
    \rotatebox{90}{\cellcolor{tableheader}{\bf Ponderaci\'{o}n\phantom{xx}}}

    & 6\%
    & 6\%
    & 6\%
    & 6\%
    & 6\%
    & 6\%
    & 6\%
    & 6\%
    & 7\%
    & 7\%
    & 7\%
    & 7\%
    & 30\%
    & 100\%
    \\ \hline
  \end{tabular}}
