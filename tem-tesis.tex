\section{Estructuraci\'{o}n en cap\'{\i}tulos, etapas o fases de la unidad de
  aprendizaje:}

\subsection{Desarrollo de las fases de la Unidad de Aprendizaje}

\quad

Orientaci\'{o}n al alumno para proseguir el tema de tesis, donde
deber\'{a}n seguirse el problema a tratar, objetivos perseguidos,
justificaci\'{o}n, planteamiento de hip\'{o}tesis y metodolog\'{\i}a a
aplicar.  Presentaci\'{o}n peri\'{o}dica del avance del trabajo para
su revisi\'{o}n, orientaci\'{o}n y validaci\'{o}n del grado de
avance.

La sesiones son de cuatro horas cada una y son veinte semanas en
total.

\paragraph{Unidades tem\'{a}ticas}

\quad

\begin{description}[itemsep=-2pt]
\item[UT1] Planteamiento del problema (2 semanas)
\item[UT2] Contextualizaci\'{o}n del trabajo (4 semanas)
\item[UT3] Descripci\'{o}n de la soluci\'{o}n propuesta (3 semanas)
\item[UT4] Evaluaci\'{o}n de la soluci\'{o}n propuesta (4 semanas)
\item[UT5] Elementos formales de un trabajo de tesis (5 semanas)
\end{description}

\paragraph{Temario semanal}

\quad
