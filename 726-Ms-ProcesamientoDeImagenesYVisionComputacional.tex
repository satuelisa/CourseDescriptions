\documentclass[10 pt]{article}
\usepackage{wallpaper}
\usepackage[spanish, mexico]{babel}  
\usepackage{color,colortbl}
\usepackage[T1]{fontenc}
\usepackage{fancyhdr} % before geometry
\usepackage[letterpaper,left=18mm,right=18mm,headheight=5mm,headsep=40mm,top=52mm,bottom=22mm]{geometry}
\usepackage{graphicx}
\usepackage{rotating}
\usepackage[latin1]{inputenc}
\usepackage{hyperref}
\usepackage{graphics}
\usepackage{varwidth}
\usepackage{tikz}

\usepackage{tcolorbox}
\definecolor{headerframe}{RGB}{177,178,177} % #b1b2b1
\definecolor{headercontent}{RGB}{239,239,239} % #efefef
\definecolor{tableheader}{RGB}{185,208,238}% #b9d0ee
\definecolor{evidence}{RGB}{238,208,185}
\definecolor{perfil}{RGB}{208,238,185}
\definecolor{unidad}{RGB}{208,185,185} 

\usetikzlibrary{shapes,arrows}
\tikzstyle{elem} = [scale=0.7, draw, rectangle, thick, minimum height=2em,
minimum width=2em, execute at begin node={\begin{varwidth}{12em}},
  execute at end node={\end{varwidth}}]
\tikzstyle{perf} = [draw, rectangle, thick, minimum height=2em,
minimum width=2em, fill=perfil]
\tikzstyle{comp} = [draw, rectangle, thick, minimum height=2em,
minimum width=2em, fill=tableheader]
\tikzstyle{esp} = [scale=0.7, draw, rectangle, thick, minimum height=2em,
minimum width=2em, fill=tableheader]
\tikzstyle{unidad} = [scale=0.9, draw, rectangle, thick, minimum height=2em,
minimum width=2em, fill=evidence]
\tikzstyle{evid} = [scale=0.75, draw, rectangle, thick, minimum height=2em,
minimum width=2em, fill=unidad]
\tikzstyle{header} = [scale=0.8, minimum height=2em, minimum width=2em, execute at begin node={\begin{varwidth}{4em}},
  execute at end node={\end{varwidth}}]


\tikzstyle{line} = [draw, thick, -stealth, shorten >=0pt]
\usepackage{multicol}
\usepackage{wrapfig}
\usepackage{enumitem}
\renewcommand*\familydefault{\sfdefault} 
\renewcommand{\thesection}{\Roman{section}} 
\usepackage{remreset}

\makeatletter
\@removefromreset{subsection}{section}
\renewcommand\thesubsection{\arabic{subsection}}
\makeatother

\makeatletter
\@removefromreset{subsubsection}{section} % no number
\renewcommand\thesubsubsection{}
\makeatother

\usepackage{titlesec}
\titleformat{\subsubsection}[runin]
{\bfseries}{\thesubsubsection}{0em}{} % in boldface

\makeatletter
\@removefromreset{theparagraph}{paragraph} % no number
\renewcommand\theparagraph{}
\makeatother

\usepackage{titlesec}
\titleformat{\paragraph}[runin]
{\itshape}{\theparagraph}{1em}{} % in italics

\ULCornerWallPaper{1}{logos.pdf}
\setlength{\parindent}{1em}
\setlength{\parskip}{2pt}
\tcbset{
  arc = 3mm,
  colframe = headerframe,
  colback = headercontent,
  fonttitle=\sffamily
}

\usepackage{amssymb}
\newcommand{\yes}{\makebox[0pt][l]{$\square$}{\raisebox{0.1\height}{$\times$}}}
\newcommand{\no}{\makebox[0pt][l]{$\square$}{\raisebox{0.1\height}{\phantom{$\times$}}}}
\usepackage{enumitem}
\titleformat{\section}{\normalfont\large\bfseries}{\thesection.}{4pt}{}
\titleformat{\subsection}[runin]{\normalfont\bfseries}{\thesubsection.}{4pt}{}


\newcommand{\UANL}{UNIVERSIDAD AUT\'{O}NOMA DE NUEVO LE\'{O}N}
\newcommand{\uanl}{Universidad Aut\'{o}noma de Nuevo Le\'{o}n}
\newcommand{\fime}{Facultad de Ingenier\'{\i}a Mec\'{a}nica y El\'{e}ctrica}
\newcommand{\maestria}{Maestr\'{\i}a en Ciencias de la Ingenier\'{\i}a con Orientaci\'{o}n en Sistemas}
\newcommand{\doctorado}{Doctorado en Ingenier\'{\i}a de Sistemas}
\newcommand{\PA}{PROGRAMA ANAL\'{I}TICO}

% LGAC
\newcommand{\seys}{Sistemas estoc\'{a}sticos y simulaci\'{o}n}
\newcommand{\mado}{M\'{e}todos avanzados de optimizaci\'{o}n}
\newcommand{\odsi}{Optimizaci\'{o}n de sistemas industriales}

% AREAS CURRICULARES
\newcommand{\fb}{formaci\'{o}n b\'{a}sica} % maestria
\newcommand{\fa}{formaci\'{o}n avanzada} % maestria
\newcommand{\fr}{formaci\'{o}n} % doctorado
\renewcommand{\div}{divulgaci\'{o}n}
\newcommand{\inv}{investigaci\'{o}n}
\renewcommand{\pi}{producto integrador}
\newcommand{\da}{de aplicaci\'{o}n}
\renewcommand{\le}{libre elecci\'{o}n}

% CLAVES DE LAS UNIDADES

\newcommand{\algm}{PM109} % analisis y diseno de algoritmos
\newcommand{\algd}{PD109}
\newcommand{\aprm}{PM134} % aprendizaje automatico
\newcommand{\aprd}{PD134}
\newcommand{\dsm}{PM123} % ciencia de datos
\newcommand{\dsd}{PD123}
\newcommand{\ccm}{PM135} % complejidad computacional
\newcommand{\ccd}{PD135} 
\newcommand{\iam}{PM101} % inteligencia artificial
\newcommand{\iad}{PD101}
\newcommand{\cvm}{PM124} % procesamiento de imagenes y vision computacional
\newcommand{\cvd}{PD124}
\renewcommand{\sim}{PM201} % seminarios
\newcommand{\siim}{PM202}
\newcommand{\sid}{PD201}
\newcommand{\siid}{PD202}
\newcommand{\siiid}{PD203}
\newcommand{\sivd}{PD204}
\newcommand{\svd}{PD205}
\newcommand{\svid}{PD206}
\newcommand{\sviid}{PD207}
\newcommand{\sviiid}{PD208}
\newcommand{\ssm}{PM105} % simulacion de sistemas
\newcommand{\ssd}{PD105}
\newcommand{\tim}{PM501} % tesis
\newcommand{\tiim}{PM502}
\newcommand{\tid}{PD501}
\newcommand{\tiid}{PD502}
\newcommand{\tiiid}{PD503}
\newcommand{\tivd}{PD504}
\newcommand{\tvd}{PD505}
\newcommand{\tvid}{PD506}
\newcommand{\tviid}{PD507}
\newcommand{\tviiid}{PD508} 

\newcommand{\narturo}{095012}
\newcommand{\nelisa}{096633}
\newcommand{\ncesar}{092038}
\newcommand{\nangy}{102662}
\newcommand{\nvincent}{102947}
\newcommand{\nada}{060581}
\newcommand{\niris}{103743}
\newcommand{\nsara}{100546}
\newcommand{\nroger}{090969}
\newcommand{\nigor}{093179}
\newcommand{\nromeo}{100959}
\newcommand{\nferny}{095808}
\newcommand{\arturo}{Dr.\ Jos\'{e} Arturo Berrones Santos}
\newcommand{\elisa}{Dra.\ Satu Elisa Schaeffer}
\newcommand{\cesar}{Dr.\ C\'{e}sar Emilio Villarreal Rodr\'{\i}guez}
\newcommand{\angy}{Dra.\ Mar\'{\i}a Ang\'{e}lica Salazar Aguilar}
\newcommand{\vincent}{Dr.\ Vincent Boyer}
\newcommand{\ada}{Dra.\ Ada Margarita \'{A}lvarez Socarr\'{a}s}
\newcommand{\iris}{Dra.\ Iris Abril Mart\'{\i}nez Salazar}
\newcommand{\sara}{Dra.\ Sara Ver\'{o}nica S\'{a}nchez Rodr\'{\i}guez}
\newcommand{\roger}{Dr.\ Roger Zirahu\'{e}n R\'{\i}os Mercado}
\newcommand{\igor}{Dr.\ Igor Litvinchev}
\newcommand{\romeo}{Dr.\ Romeo S\'{a}nchez Nigenda}
\newcommand{\ferny}{Dr.\ Fernando L\'{o}pez Irarragorri}
\newcommand{\simon}{Dr.\ Sim\'{o}n Mart\'{\i}nez Mart\'{\i}nez} 

\newcommand{\subdirector}{\simon}
\newcommand{\coordinador}{\cesar}


\begin{document}

\pagestyle{fancy}
\renewcommand{\headrulewidth}{0pt}
\fancyhf{}
\fancyhead[L]{}
\fancyhead[C]{}
\fancyhead[R]{IT-8-SPG-02-R03}
\fancyfoot[L]{Revisi\'{o}n: 1 \\
  Vigente a partir del: 01 de agosto del 2016}
\fancyfoot[C]{}
\fancyfoot[R]{P\'{a}gina~\thepage~de~\pageref*{final}}


\fancypagestyle{plain}{%
\fancyhf{}
\fancyhead[L]{}
\fancyhead[C]{}
\fancyhead[R]{IT-8-SPG-02-R03}
\fancyfoot[L]{Revisi\'{o}n: 1 \\
  Vigente a partir del: 01 de agosto del 2016}
\fancyfoot[C]{}
\fancyfoot[R]{}}

\begin{tcolorbox}
  \begin{center}

    {\bf \UANL}

    \medskip

    {\bf \fime}

    \medskip
    
    {\bf PE} \underline{\bf \maestria}

    \medskip

    \underline{\PA}

  \end{center}
\end{tcolorbox}

\section{Datos de Identificaci\'{o}n de la Unidad de Aprendizaje:}
\subsection{Nombre:} \fbox{Procesamiento de im\'{a}genes y visi\'{o}n computacional}
\subsection{Frecuencia semanal:} horas de trabajo presencial \fbox{4}
\subsection{Horas de trabajo extra aula por semana:} \fbox{2}
\subsection{Modalidad:} \yes~Escolarizada \no~No escolarizada \no~Mixto
\subsection{Per\'{\i}odo acad\'{e}mico:} \yes~Semestral
\no~Tetramestral \no~Modular
\subsection{LGAC:} \underline{Sistemas estoc\'{a}sticos y simulaci\'{o}n}
\subsection{Ubicaci\'{o}n semestral:} \underline{1 o 2}
\subsection{\'{A}rea curricular:} \underline{Formaci\'{o}n b\'{a}sica,
  formaci\'{o}n avanzada, de aplicaci\'{o}n, libre elecci\'{o}n, investigaci\'{o}n}
\subsection{Cr\'{e}ditos:} \underline{4}
\subsection{Requisito:} \underline{Ninguno}
\subsection{Fecha de elaboraci\'{o}n:} \underline{20/01/2010}
\subsection{Fecha de la \'{u}ltima actualizaci\'{o}n:} \underline{09/06/2021}
\subsection{Responsable (es) del dise\~{n}o:}
\begin{itemize}[label={}]
\item \underline{\narturo~\arturo}
\item \underline{\nelisa~\elisa}
\end{itemize}
\newpage
\section{Presentaci\'{o}n:}

La visi\'{o}n computacional refiere al procesamiento automatizado de
im\'{a}genes para extrar informaci\'{o}n para sistemas de toma de decisiones.
En procesamiento de nivel bajo, se trabaja directamente con las
im\'{a}genes para extraer propiedades como orillas, gradiente,
profundidad, textura, color, etc. Procesamiento de nivel intermedio
consiste generalmente en agrupar los elementos obtenidos en el nivela
bajo, para obtener, por ejemplo, contornos y regiones, generalmente
con el prop\'{o}sito de segmentaci\'{o}n. Procesamiento de alto nivel, por
\'{u}ltimo, consiste en la interpretaci\'{o}n de los entes obtenidos en los
niveles inferiores y se uGlizan modelos y/o conocimiento a priori del
dominio

\section{Prop\'{o}sito(s):}

Introducci\'{o}n a la visi\'{o}n computacional que trata de emular esta
capacidad en las computadoras, de forma que, mediante la
interpretaci\'{o}n de las im\'{a}genes adquiridas, por ejemplo, con una
c\'{a}mara, se puedan reconocer los diversos objetos en el ambiente y su
posici\'{o}n en el espacio.

\section{Competencias del perfil de egreso:}
\subsection{Competencias del perfil de egreso}

P1) Resolver problemas en el \'{a}rea de toma de decisiones en ambientes
operativos que pueden ser din\'{a}micos o inciertos para lograr una
asignaci\'{o}n m\'{a}s efectiva de recursos y decidir el curso de acci\'{o}n
\'{o}ptimo para lograr objetivos establecidos.



P2) Resolver problemas concretos en sistemas de la industria, la
academia o el sector p\'{u}blico en base a las herramientas de la toma de
decisiones con bases cient\'{\i}ficas para lograr el mejor dise\~{n}o,
an\'{a}lisis, planeaci\'{o}n o gesti\'{o}n de dichos sistemas.


P3) Establecer comunicaci\'{o}n con los disGntos sectores de la
sociedad a fin de establecer proyectos estrat\'{e}gicos en las
distintas disciplinas de la ingenier\'{\i}a de sistemas y crear la
cultura de la creaci\'{o}n de riqueza basada en el conocimiento.

  
\subsection{Competencias generales a que se vincula la Unidad de
    Aprendizaje:}

  La unidad se vincula con las siguientes competencias generales:

  \phantom{space}
\begin{tabular}{|p{12cm}|p{45mm}|}
  \hline
  \cellcolor{tableheader}{\em Declaraci\'{o}n de la competencia general vinculada a la unidad
  de aprendizaje}
  & \cellcolor{tableheader}{\em Evidencia} \\ \hline

  C2) Utiliza los lenguajes l\'{o}gico, formal, matem\'{a}tico, ic\'{o}nico, verbal
y no verbal de acuerdo a su etapa de vida en el \'{a}rea de las ciencias
para comprender, interpretar y expresar ideas, sentimientos, teor\'{\i}as y
corrientes de pensamiento con un enfoque ecum\'{e}nico.
 & Tareas \\ \hline
  C3) Maneja las tecnolog\'{\i}as de la informaci\'{o}n de acuerdo a los usos del
campo de las ciencias y la comunicaci\'{o}n como herramientas para el
acceso a la informaci\'{o}n y su transformaci\'{o}n en conocimiento, as\'{\i} como
para el aprendizaje y trabajo colaborativo con t\'{e}cnicas de vanguardia
que le permitan su participaci\'{o}n constructiva en la sociedad.
 & Tareas, proyecto \\ \hline
  C5) Emplea pensamiento l\'{o}gico, cr\'{\i}tico, creativo y propositivo,
siguiendo los modelos de pensamiento cient\'{\i}fico para analizar
fen\'{o}menos naturales y sociales que le permitan tomar decisiones
pertinentes en su \'{a}mbito de influencia con responsabilidad social.
 & Tareas, proyecto \\ \hline
  \end{tabular}

\newpage
    
\subsection{Competencias espec\'{\i}ficas y nivel de dominio a que se vincula la unidad de aprendizaje:}

  La unidad se vincula con las siguientes competencias espec\'{\i}ficas:

  \begin{tabular}{|p{30mm}|p{5mm}|p{5mm}|p{3cm}|p{12mm}|p{30mm}|p{12mm}|p{5mm}|p{12mm}|}
\hline
\cellcolor{tableheader}{{\em Competencia Espec\'{\i}fica}}
& \rotatebox{90}{\cellcolor{tableheader}{Nivel I Inicial}}
& \rotatebox{90}{\cellcolor{tableheader}{Evidencia}}
& \rotatebox{90}{\cellcolor{tableheader}{Nivel II B\'{a}sico}}
& \rotatebox{90}{\cellcolor{tableheader}{Evidencia}}
& \rotatebox{90}{\cellcolor{tableheader}{Nivel III Aut\'{o}nomo}}
& \rotatebox{90}{\cellcolor{tableheader}{Evidencia}}
& \rotatebox{90}{\cellcolor{tableheader}{Nivel IV Estrat\'{e}gico\phantom{xxx}}}
& \rotatebox{90}{\cellcolor{tableheader}{Evidencia}}
\\ \hline


  E2) Resolver problemas concretos en sistemas de la industria, la
academia o el sector p\'{u}blico en base a las herramientas de la toma
de decisiones con bases cient\'{\i}ficas para lograr el mejor dise\~{n}o,
an\'{a}lisis, planeaci\'{o}n o gesti\'{o}n de dichos sistemas.

  & & &
  Identifica los principios de la ingenier\'{\i}a de sistemas necesarios para modelar y resolver un problema aplicado espec\'{\i}fico.
  &
  Tareas.
  &
  Resuelve problemas de libro de texto en el \'{a}rea de toma de decisiones
con bases cient\'{\i}ficas

  &
  Tareas, proyecto.
  &&  \\ \hline
  
\end{tabular}

\section{Representaci\'{o}n gr\'{a}fica:}

\begin{tikzpicture}[scale=1, auto]
  \matrix[row sep=1cm, column sep=7mm]{
      & 
      \node[elem] (ide) {Identificar un problema de inter\'{e}s que se
        busca resolver};
      &
      \node[elem] (inp) {Obtener im\'{a}genes de entrada};
      \\
      \node[elem] (lit) {Identificar y revisar literatura cient\'{\i}fica relacionada relevante};
      &
      \node[elem] (des) {Desarrollar un m\'{e}todo de visi\'{o}n computacional};
      &
      \node[elem] (her) {Revisar herramientas computacionales
        existentes para implementar el m\'{e}todo};
      \\
      \node[elem] (for) {Formular hip\'{o}tesis parala validaci\'{o}n del
        modelo propuesto};
      &
      \node[elem] (imp) {Implementar el m\'{e}todo con las herramientas seleccionadas};
      &
      \node[elem] (sel) {Seleccionar las herramientas seg\'{u}n las caracter\'{\i}sticas del m\'{e}todo};
      \\
      & \node[elem] (dis) {Dise\~{n}ar y realizar experimentos};
      \\
      \node[elem] (ana) {Analizar los resultados de los experimentos y contrastarlas con las hip\'{o}tesis};
      & & 
      \node[elem] (doc) {Documentar el proceso conforme al estilo de
        redacci\'{o}n cient\'{\i}fica};
      \\
      & 
      \node[elem] (apl) {Aplicar visualizaci\'{o}n cient\'{\i}fica a los resultados};
      & 
      \node[elem] (exp) {Discutir entre pares los m\'{e}todos y los
        resultados obtenidos};
      \\
    };
    \draw [line] (ide) -- (inp);
    \draw [line] (inp) -- (dis);
    \draw [line] (ide) -- (lit);
    \draw [line] (ide) -- (des);
    \draw [line] (ide) -- (her);
    \draw [line] (lit) -- (des);
    \draw [line] (lit) -- (for);
    \draw [line] (des) -- (imp);
    \draw [line] (her) -- (sel);
    \draw [line] (sel) -- (imp);
    \draw [line] (for) -- (dis);
    \draw [line] (imp) -- (dis);
    \draw [line] (dis) -- (ana);
    \draw [line] (dis) -- (apl);
    \draw [line] (ana) -- (doc);
    \draw [line] (apl) -- (doc);
    \draw [line] (doc) -- (exp);
  \end{tikzpicture}

\newpage
\section{Estructuraci\'{o}n en cap\'{\i}tulos, etapas o fases de la unidad de
  aprendizaje:}
\subsection{Desarrollo de las fases de la Unidad de Aprendizaje:}

Se cubren los principios te\'{o}ricos de la visi\'{o}n computacional.  Se
busca esarrollar habilidades en la resoluci\'{o}n en casos
pr\'{a}cticos concretos. Se necesita contar con un buen entendimiento
de varios los conceptos matem\'{a}ticos, especialmente de
matem\'{a}ticas discretas y probabilidad, o en el caso contrario,
estar preparado a estudiarlos seg\'{u}n necesidad. Tambi\'{e}n se
necesita conocimiento de programaci\'{o}n.  La sesiones son de cuatro
horas cada una y son veinte semanas en total.
\begin{enumerate}[itemsep=-3pt]
\item{Introducci\'{o}n; selecci\'{o}n de temas de proyecto}
\item{Representaci\'{o}n de colores}
\item{Histogramas y umbrales}
\item{Filtros y m\'{a}scaras}
\item{An\'{a}lisis de bordes}
\item{An\'{a}lisis de formas}
\item{Detecci\'{o}n de l\'{\i}neas}
\item{Detecci\'{o}n de c\'{\i}rculos}
\item{Detecci\'{o}n de elipses}
\item{Detecci\'{o}n de agujeros}
\item{Detecci\'{o}n de pol\'{\i}gonos y esquinas}
\item{Formatos de video}
\item{Detecci\'{o}n de movimiento}
\item{Reconocimiento de objetos}
\item{Decomposici\'{o}n de wavelets}
\item{Reducci\'{o}n de ruido}
\item{Im\'{a}genes h\'{\i}perespectarles}
\item{Procesamiento en tiempo real}
\item{Presentaciones de proyectos}
\item{Revisi\'{o}n de portafolios de evidencia}
\end{enumerate}

{\em Elementos de competencia:}

\paragraph{Elementos de competencia:}

\quad

\begin{tabular}{|p{28mm}|p{30mm}|p{30mm}|p{30mm}|p{30mm}|}
  \hline
  \cellcolor{tableheader}Evidencias de aprendizaje 
  & \cellcolor{tableheader}Criterios de desempe\~{n}o
  & \cellcolor{tableheader}Actividades de aprendizaje
  & \cellcolor{tableheader}Contenidos
  & \cellcolor{tableheader}Recursos \\ \hline

  Reporte escrito y/ c\'{o}digo de la implementaci\'{o}n de un algoritmo de
  visi\'{o}n computacional.
  & Calidad de la redacci\'{o}n cient\'{\i}fica del reporte; 
    precisi\'{o}n del m\'{e}todo propuesto;
    eficiencia de la implementaci\'{o}n del m\'{e}todo;
    cobertura de la experimentaci\'{o}n.
  & Experimentaci\'{o}n con ejemplos; lectura de material de apoyo;
    modificaci\'{o}n de ejemplos; dise\~{n}o y ejecuci\'{o}n de experimentos;
    an\'{a}lisis y reportaje de resultados obtenidos.
  &
    M\'{e}todos diversos de visi\'{o}n computacional
  & Material en la p\'{a}gina web de la unidad y la literatura citada;
    lenguaje Python o similar; paquete {\LaTeX} para redacci\'{o}n cient\'{\i}fica;
    repositorios de GitHub. \\ \hline
  
  \end{tabular}

\newpage
  
  \section{Evaluaci\'{o}n integral de procesos y productos:}
  
  Las tareas son individuales; se recomienda estudiar juntos y
  discutir las soluciones, pero no se tolera ning\'{u}n tipo de plagio en
  absoluto, ni de otros estudiantes ni de la red ni de libros --- toda
  referencia bibliogr\'{a}fica tiene que ser apropiadamente citada. La
  entrega se realiza por un repositorio en GitHub que debe reflejar
  todas las fases del trabajo en su log correspondiente. El alumno
  selecciona su lenguaje de programaci\'{o}n para cada tarea.

  Son 17 tareas (A1--A17) que reportan avances semanales de aplicaci\'{o}n
  de la lectura de la semana para el proyecto del estudiante,
  otorgando por m\'{a}ximo 5 puntos por tarea:
  \begin{description}[itemsep=0em]
  \item[NP]{= tarea omitida}
  \item[5]{= excede lo que se esperaba}
  \item[4]{= cumple con lo que se esperaba}
  \item[3]{= d\'{e}bil en alcance y/o calidad}
  \item[2]{= d\'{e}bil en ambos alcance y calidad}
  \item[1]{= sin contribuciones o m\'{e}ritos aunque fue entregada}
  \item[0]{= completamente inadecuado en alzance y calidad}
  \end{description}
  El proyecto final (A18) otorga un m\'{a}ximo de 15 puntos, evaluados en los
  siguientes rubros
  \begin{enumerate}[itemsep=0em]
  \item{Variedad de t\'{e}cnicas de visi\'{o}n computacional empleadas}
  \item{Cobertura y validez de la experimentaci\'{o}n}
  \item{Claridad y relevancia de los resultados computacionales}    
  \item{Calidad de visualizaci\'{o}n cient\'{\i}fica}
  \item{Calidad de redacci\'{o}n cient\'{\i}fica}
  \end{enumerate}
  con la escala:
  \begin{description}[itemsep=0em]
  \item[3]{= cumple con lo que se esperaba}
  \item[2]{= d\'{e}bil en alcance y/o calidad}
  \item[1]{= d\'{e}bil en ambos alcance y calidad}
  \item[0]{= inadecuado en alzance y calidad}
  \end{description}


  No habr\'{a} examen.


  
  Ponderaci\'{o}n espec\'{\i}fica:

  \scalebox{0.85}{
    \begin{tabular}{|c|cccccccccccccccccc|c|}
      \hline
      
      \rotatebox{90}{\cellcolor{tableheader}{\bf Actividad\phantom{xx}}} & A1 & A2 & A3 & A4 & A5 & A6 & A7 & A8 & A9 & A10 &
                                                                                                                              A11
      & A12 & A13 & A14 & A15 & A16 & A17 & A18 & {\bf Total}
      \\
      \hline
      \rotatebox{90}{\cellcolor{tableheader}{\bf Ponderaci\'{o}n\phantom{xx}}}
                                                                         & 5\% & 5\% & 5\% &
                                                                                             5\% &
                                                                                                   5\% & 5\% & 5\% & 5\%
                                                                                                                 & 5\%
                                                                                                                      & 5\%
                                                                                                                            & 5\%
      & 5\% 
            & 5\% & 5\% & 5\% & 5\% & 5\% & 15\% & 100\%		
      \\ \hline
    \end{tabular}}
  
  \newpage

\section{Producto integrador de aprendizaje de la unidad:}
\subsection{Producto integrador de Aprendizaje:} Portafolio en un repositorio digital p\'{u}blico.

\section{Fuentes de apoyo y consulta:}
\subsection{Fuentes de apoyo y consulta}
\subsubsection{B\'{a}sicas}

 \begin{itemize}[itemsep=0em]

 \item{E.R.\ {\sc Davies}, {\em Machine Vision: Theory, Algorithms,
       Practicalities}, Morgan Kaufmann Publishers Inc., San
     Francisco, CA, USA, 2004. ISBN 0122060938.}
   
\end{itemize}

\subsubsection{Complementarias}

Art\'{\i}culos cient\'{\i}ficos especializados.

\label{final} % last page
%\newpage

%\pagestyle{plain}

%\vspace*{3cm}

%{\bf Autoriz\'{o}:} \coordinador

%\vspace*{2cm}

%  \begin{center}
%  {\sc Alere Flammam Veritatis}
  
%  Ciudad Universitaria, \today

%\vspace*{4cm}
  
%  \begin{tabular}{p{6cm}cp{7cm}}
%    \cline{1-1}
%    \cline{3-3}    
%    {\bf \coordinador}
% & \phantom{xxx} &{\bf Vo.\ Bo.\ \subdirector} \\
%    Coordinador Acad\'{e}mico &  &Subdirector de Estudios de Posgrado \\
%    Posgrado en Ingenier\'{\i}a de Sistemas & & Facultad de Ingenier\'{\i}a Mec\'{a}nica y El\'{e}ctrica
                                                                   
%  \end{tabular}
%\end{center}

\end{document}