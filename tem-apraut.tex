\newpage

\section{Estructuraci\'{o}n en cap\'{\i}tulos, etapas o fases de la unidad de
  aprendizaje:}

\subsection{Desarrollo de las fases de la Unidad de Aprendizaje:}

\quad

Se cubren los principios te\'{o}ricos del aprendizaje
autom\'{a}tico. Desarrollar habilidades en la resoluci\'{o}n en casos
pr\'{a}cticos concretos. Se necesita contar con un buen entendimiento
de varios los conceptos matem\'{a}ticos, especialmente de
matem\'{a}ticas discretas y probabilidad, o en el caso contrario,
estar preparado a estudiarlos seg\'{u}n necesidad. Tambi\'{e}n se
necesita conocimiento de programaci\'{o}n.

\paragraph{Unidades tem\'{a}ticas}

\begin{description}[itemsep=-3pt]
\item[U1]{Modelos lineales de aprendizaje supervisado (7 semanas)}
\item[U2]{Modelos no-lineales de aprendizaje supervisado (5 semanas)}
\item[U3]{Aprendizaje no-supervisado (5 semanas)}  
\end{description}

La sesiones son de cuatro horas cada una y son veinte semanas en
total. Las \'{u}ltimas tres semanas son para el desarrollo del
proyecto integrador que combina elementos de las tres unidades de
aprendizaje.


\paragraph{Temario semanal}

\begin{enumerate}[itemsep=-3pt]
\item{Introducci\'{o}n; selecci\'{o}n de temas de proyecto}
\item{U1: Aprendizaje supervisado}
\item{U1: Regresi\'{o}n lineal}
\item{U1: Clasificaci\'{o}n lineal}
\item{U1: Expansi\'{o}n y regularizaci\'{o}n}
\item{U1: Suavizaci\'{o}n con n\'{u}cleos}
\item{U1: Evaluaci\'{o}n y selecci\'{o}n de modelos}
\item{U1: Inferencia y promediaci\'{o}n}
\item{U2: Modelos aditivos y \'{a}rboles}
\item{U2: Impulso y \'{a}rboles aditivos}
\item{U2: Redes neuronales}
\item{U2: M\'{a}quinas de soporte vectorial}
\item{U2: Vecinos m\'{a}s cercanos}
\item{U3: Aprendizaje no supervisado}
\item{U3: B\'{o}sques aleatorios}
\item{U3: Aprendizaje colectivo}
\item{U3: Modelos gr\'{a}ficos no dirigidos}
\item{U3: Problemas de alta dimensionalidad}
\item{Presentaciones de proyectos}
\item{Revisi\'{o}n de portafolios de evidencia}
\end{enumerate}
