\section{Estructuraci\'{o}n en cap\'{\i}tulos, etapas o fases de la unidad de
  aprendizaje:}

\subsection{Desarrollo de las fases de la Unidad de Aprendizaje:}

\quad

Se cubren los principios te\'{o}ricos de la visi\'{o}n computacional.  Se
busca desarrollar habilidades en la resoluci\'{o}n en casos
pr\'{a}cticos concretos. Se necesita contar con un buen entendimiento
de varios los conceptos matem\'{a}ticos, especialmente de
matem\'{a}ticas discretas y probabilidad, o en el caso contrario,
estar preparado a estudiarlos seg\'{u}n necesidad. Tambi\'{e}n se
necesita conocimiento de programaci\'{o}n.  La sesiones son de cuatro
horas cada una y son veinte semanas en total.

\paragraph{Unidades tem\'{a}ticas}

\begin{description}[itemsep=-3pt]
\item[UT1] Umbrales y filtros (3 semanas)
\item[UT2] Detecci\'{o}n y an\'{a}lisis formas (7 semanas)
\item[UT3] Detecci\'{o}n de movimiento (3 semanas)
\item[UT4] T\'{e}cnicas y aplicaciones (4 semanas)
\end{description}

La sesiones son de cuatro horas cada una y son veinte semanas en
total. La primera semana es introductoria y las \'{u}ltimas dos semanas
combinan elementos de las tres unidades tem\'{a}ticas en el contexto del
proyecto integrador.


\paragraph{Temario semanal}

\begin{enumerate}[itemsep=-3pt]
\item{Introducci\'{o}n; selecci\'{o}n de temas de proyecto}
\item{UT1: Representaci\'{o}n de colores}
\item{UT1: Histogramas y umbrales}
\item{UT1: Filtros y m\'{a}scaras}
\item{UT2: An\'{a}lisis de bordes}
\item{UT2: An\'{a}lisis de formas}
\item{UT2: Detecci\'{o}n de l\'{\i}neas}
\item{UT2: Detecci\'{o}n de c\'{\i}rculos}
\item{UT2: Detecci\'{o}n de elipses}
\item{UT2: Detecci\'{o}n de agujeros}
\item{UT2: Detecci\'{o}n de pol\'{\i}gonos y esquinas}
\item{UT3: Formatos de video}
\item{UT3: Detecci\'{o}n de movimiento}
\item{UT3: Reconocimiento de objetos}
\item{UT4: Decomposici\'{o}n de wavelets}
\item{UT4: Reducci\'{o}n de ruido}
\item{UT4: Im\'{a}genes hiperespectarles}
\item{UT4: Procesamiento en tiempo real}
\item{Presentaciones de proyectos}
\item{Revisi\'{o}n de portafolios de evidencia}
\end{enumerate}

\newpage

\subsubsection{Unidad tem\'{a}tica 1:  Umbrales y filtros}

\paragraph{Periodo:} 3 semanas

\paragraph{Elementos de competencia:}

\quad

\begin{tabular}{|p{28mm}|p{30mm}|p{30mm}|p{30mm}|p{30mm}|}
  \hline
  \cellcolor{tableheader}Evidencias de aprendizaje 
  & \cellcolor{tableheader}Criterios de desempe\~{n}o
  & \cellcolor{tableheader}Actividades de aprendizaje
  & \cellcolor{tableheader}Contenidos
  & \cellcolor{tableheader}Recursos \\ \hline


Tres (3) tareas semanales consistiendo cada una en un reporte escrito
y c\'{o}digo de la implementaci\'{o}n de algoritmos de visi\'{o}n
computacional de aplicaci\'{o}n de umbrales y filtros.

&

Calidad de la redacci\'{o}n cient\'{\i}fica de los reportes;
precisi\'{o}n y eficiencia de la implementaci\'{o}n; cobertura de la
experimentaci\'{o}n.

&

Experimentaci\'{o}n con ejemplos; lectura de material de apoyo;
modificaci\'{o}n de ejemplos; dise\~{n}o y ejecuci\'{o}n de
experimentos; an\'{a}lisis y reportaje de resultados obtenidos.

&

M\'{e}todos diversos de umbralizaci\'{o}n y filtrado.

&

Material en la p\'{a}gina web de la unidad y la literatura citada;
lenguaje Python o similar; paquete {\LaTeX} para redacci\'{o}n
cient\'{\i}fica; repositorios de p\'{u}blicos de c\'{o}digo
fuente.
  \\ \hline

\end{tabular}


\subsubsection{Unidad tem\'{a}tica 2:  Detecci\'{o}n y an\'{a}lisis
  formas}

\paragraph{Periodo:} 7 semanas

\paragraph{Elementos de competencia:}

\quad

\begin{tabular}{|p{28mm}|p{30mm}|p{30mm}|p{30mm}|p{30mm}|}
  \hline
  \cellcolor{tableheader}Evidencias de aprendizaje 
  & \cellcolor{tableheader}Criterios de desempe\~{n}o
  & \cellcolor{tableheader}Actividades de aprendizaje
  & \cellcolor{tableheader}Contenidos
  & \cellcolor{tableheader}Recursos \\ \hline


Siete (7) tareas semanales consistiendo cada una en un reporte escrito
y c\'{o}digo de la implementaci\'{o}n de algoritmos de visi\'{o}n
computacional para la detecci\'{o}n de {\em formas}.

&

Calidad de la redacci\'{o}n cient\'{\i}fica de los reportes;
precisi\'{o}n y eficiencia de la implementaci\'{o}n; cobertura de la
experimentaci\'{o}n.

&

Experimentaci\'{o}n con ejemplos; lectura de material de apoyo;
modificaci\'{o}n de ejemplos; dise\~{n}o y ejecuci\'{o}n de
experimentos; an\'{a}lisis y reportaje de resultados obtenidos.

&

M\'{e}todos diversos de detecci\'{o}n de formas.

&

Material en la p\'{a}gina web de la unidad y la literatura citada;
lenguaje Python o similar; paquete {\LaTeX} para redacci\'{o}n
cient\'{\i}fica; repositorios de p\'{u}blicos de c\'{o}digo
fuente.
  \\ \hline

\end{tabular}

\newpage

\subsubsection{Unidad tem\'{a}tica 3:  Detecci\'{o}n de movimiento}

\paragraph{Periodo:} 3 semanas

\paragraph{Elementos de competencia:}

\quad

\begin{tabular}{|p{28mm}|p{30mm}|p{30mm}|p{30mm}|p{30mm}|}
  \hline
  \cellcolor{tableheader}Evidencias de aprendizaje 
  & \cellcolor{tableheader}Criterios de desempe\~{n}o
  & \cellcolor{tableheader}Actividades de aprendizaje
  & \cellcolor{tableheader}Contenidos
  & \cellcolor{tableheader}Recursos \\ \hline


Tres (3) tareas semanales consistiendo cada una en un reporte escrito
y c\'{o}digo de la implementaci\'{o}n de algoritmos de visi\'{o}n
computacional para la detecci\'{o}n de {\em movimiento}.

&

Calidad de la redacci\'{o}n cient\'{\i}fica de los reportes;
precisi\'{o}n y eficiencia de la implementaci\'{o}n; cobertura de la
experimentaci\'{o}n.

&

Experimentaci\'{o}n con ejemplos; lectura de material de apoyo;
modificaci\'{o}n de ejemplos; dise\~{n}o y ejecuci\'{o}n de
experimentos; an\'{a}lisis y reportaje de resultados obtenidos.

&

M\'{e}todos diversos de detecci\'{o}n de movimiento.

&

Material en la p\'{a}gina web de la unidad y la literatura citada;
lenguaje Python o similar; paquete {\LaTeX} para redacci\'{o}n
cient\'{\i}fica; repositorios de p\'{u}blicos de c\'{o}digo
fuente.
  \\ \hline

\end{tabular}


\subsubsection{Unidad tem\'{a}tica 4:  T\'{e}cnicas y aplicaciones}

\paragraph{Periodo:} 4 semanas

\paragraph{Elementos de competencia:}

\quad

\begin{tabular}{|p{28mm}|p{30mm}|p{30mm}|p{30mm}|p{30mm}|}
  \hline
  \cellcolor{tableheader}Evidencias de aprendizaje 
  & \cellcolor{tableheader}Criterios de desempe\~{n}o
  & \cellcolor{tableheader}Actividades de aprendizaje
  & \cellcolor{tableheader}Contenidos
  & \cellcolor{tableheader}Recursos \\ \hline


Cuatro (4) tareas semanales que son reportes escritos y c\'{o}digo de
aplicaciones de visi\'{o}n computacional.

&

Calidad de la redacci\'{o}n cient\'{\i}fica del reporte; precisi\'{o}n
del m\'{e}todo aplicado; eficiencia de la implementaci\'{o}n del
m\'{e}todo; cobertura de la experimentaci\'{o}n.

&

Experimentaci\'{o}n con ejemplos; lectura de material de apoyo;
modificaci\'{o}n de ejemplos; dise\~{n}o y ejecuci\'{o}n de
experimentos; an\'{a}lisis y reportaje de resultados obtenidos.

&

M\'{e}todos diversos de visi\'{o}n computacional.

&

Material en la p\'{a}gina web de la unidad y la literatura citada;
lenguaje Python o similar; paquete {\LaTeX} para redacci\'{o}n
cient\'{\i}fica; repositorios de p\'{u}blicos de c\'{o}digo
fuente.
  \\ \hline

\end{tabular}

\newpage

\section{Evaluaci\'{o}n integral de procesos y productos:}

Las tareas son individuales; se recomienda estudiar juntos y discutir
las soluciones, pero no se tolera ning\'{u}n tipo de plagio en
absoluto, ni de otros estudiantes ni de la red ni de libros --- toda
referencia bibliogr\'{a}fica tiene que ser apropiadamente citada. La
entrega se realiza por un repositorio p\'{u}blico que debe reflejar todas
las fases del trabajo. 

No habr\'{a} examen.  Son 17 tareas (A1--A17) que reportan avances
semanales de aplicaci\'{o}n de la lectura de la semana para el
proyecto del estudiante, otorgando por m\'{a}ximo 5 puntos por
tarea: \begin{description}[itemsep=-2pt]
\item[NP]{= tarea omitida}
\item[5]{= excede lo que se esperaba}
\item[4]{= cumple con lo que se esperaba}
\item[3]{= d\'{e}bil en alcance y/o calidad}
\item[2]{= d\'{e}bil en ambos alcance y calidad}
\item[1]{= sin contribuciones o m\'{e}ritos aunque fue entregada}
\item[0]{= completamente inadecuado en alzance y calidad}
\end{description}
 El proyecto final (A18) otorga un
m\'{a}ximo de 15 puntos, evaluados en los siguientes
rubros \begin{enumerate}[itemsep=0em]
\item{Variedad de t\'{e}cnicas de empleadas}
\item{Cobertura y validez de la experimentaci\'{o}n}
\item{Claridad y relevancia de los resultados}    
\item{Calidad de visualizaci\'{o}n cient\'{\i}fica}
\item{Calidad de redacci\'{o}n cient\'{\i}fica}
\end{enumerate}
 con la escala: \begin{description}[itemsep=0em]
\item[3]{= cumple con lo que se esperaba}
\item[2]{= d\'{e}bil en alcance y/o calidad}
\item[1]{= d\'{e}bil en ambos alcance y calidad}
\item[0]{= inadecuado en alzance y calidad}
\end{description}


\paragraph{Ponderaci\'{o}n espec\'{\i}fica}

\quad

\scalebox{0.85}{
  \begin{tabular}{|c|cccccccccccccccccc|c|}
    \hline
    \rotatebox{90}{\cellcolor{tableheader}{\bf Actividad\phantom{xx}}}
    & A1
    & A2
    & A3
    & A4
    & A5
    & A6
    & A7
    & A8
    & A9
    & A10
    & A11
    & A12
    & A13
    & A14
    & A15
    & A16
    & A17
    & A18
    & {\bf Total}
    \\
    \hline
    \rotatebox{90}{\cellcolor{tableheader}{\bf Ponderaci\'{o}n\phantom{xx}}}
    & 5\%
    & 5\%
    & 5\%
    & 5\% 
    & 5\%
    & 5\%
    & 5\%
    & 5\%
    & 5\%
    & 5\%
    & 5\%
    & 5\% 
    & 5\%
    & 5\%
    & 5\%
    & 5\%
    & 5\%
    & 15\%
    & 100\%		
    \\ \hline
  \end{tabular}}
  


\section{Producto integrador de aprendizaje de la unidad:}

\subsection{Producto integrador de Aprendizaje:}

\quad

Portafolio en un repositorio digital p\'{u}blico que contiene los
reportes escritos y los c\'{o}digos de la implementaci\'{o}n de todas
las tareas y el proyecto integrador.


\section{Fuentes de apoyo y consulta:}
\subsection{Fuentes de apoyo y consulta}
\subsubsection{B\'{a}sicas}

 \begin{itemize}[itemsep=0em]

 \item{E.R.\ {\sc Davies}, {\em Machine Vision: Theory, Algorithms,
       Practicalities}, Morgan Kaufmann Publishers Inc.\ 2017. Quinta edici\'{o}n.}

 \item{M.\ {\sc Elgendy}, {\em Deep Learning for Vision Systems}, Manning Publications 2020.}

\item{L.\ {\sc Venturi} \& K.\ {\sc Korda}: {\em Hands-On Vision and
    Behavior for Self-Driving Cars: Explore visual perception, lane
    detection, and object classification with Python 3 and OpenCV 4},
  Packt Publishing, 2020.}

\item{R.\ {\sc Klette}: {\em Concise Computer Vision: An Introduction
    Into Theory and Algorithms}, Springer, 2014.}
  
\end{itemize}

\subsubsection{Complementarias}

Art\'{\i}culos cient\'{\i}ficos especializados relacionados a los
temas tratados, de preferencia publicados en revistas internacionales
indizados recientes.

 

\label{final} % last page
\newpage

\pagestyle{plain}

\vspace*{3cm}

{\bf Autoriz\'{o}:} \coordinador

\vspace*{2cm}

  \begin{center}
  {\sc Alere Flammam Veritatis}
  
  Ciudad Universitaria, \today

\vspace*{4cm}
  
  \begin{tabular}{p{6cm}cp{7cm}}
    \cline{1-1}
    \cline{3-3}    
    {\bf \coordinador} &
                                                            \phantom{xxx} &{\bf Vo.\ Bo.\ \subdirector} \\
    Coordinador Acad\'{e}mico &  &Subdirector de Estudios de Posgrado \\
    Posgrado en Ingenier\'{\i}a de Sistemas & & Facultad de Ingenier\'{\i}a Mec\'{a}nica y El\'{e}ctrica
                                                                   
  \end{tabular}
\end{center}

\label{final} % last page
\newpage

\pagestyle{plain}

\vspace*{3cm}

{\bf Autoriz\'{o}:} \coordinador

\vspace*{2cm}

  \begin{center}
  {\sc Alere Flammam Veritatis}
  
  Ciudad Universitaria, \today

\vspace*{4cm}
  
  \begin{tabular}{p{6cm}cp{7cm}}
    \cline{1-1}
    \cline{3-3}    
    {\bf \coordinador} &
                                                            \phantom{xxx} &{\bf Vo.\ Bo.\ \subdirector} \\
    Coordinador Acad\'{e}mico &  &Subdirector de Estudios de Posgrado \\
    Posgrado en Ingenier\'{\i}a de Sistemas & & Facultad de Ingenier\'{\i}a Mec\'{a}nica y El\'{e}ctrica
                                                                   
  \end{tabular}
\end{center}

\end{document}
