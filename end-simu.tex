\section{Estructuraci\'{o}n en cap\'{\i}tulos, etapas o fases de la unidad de
  aprendizaje:}
\subsection{Desarrollo de las fases de la Unidad de Aprendizaje:}

\quad

Se cubren y aplican los principios te\'{o}ricos de la {\em
  simulaci\'{o}n computacional}. Se busca desarrollar habilidades en
la resoluci\'{o}n en casos pr\'{a}cticos concretos. Se necesita contar
con un buen entendimiento de varios conceptos matem\'{a}ticos,
especialmente de matem\'{a}ticas discretas y probabilidad, o en el
caso contrario, estar preparado a estudiarlos seg\'{u}n
necesidad. Tambi\'{e}n es conveniente contar con conocimiento de
programaci\'{o}n.

\paragraph{Unidades tem\'{a}ticas}

\begin{description}[itemsep=-3pt]
\item[UT1] Conceptos b\'{a}sicos para modelado matem\'{a}tico (5 semanas)
\item[UT2] T\'{e}cnicas adicionales de la simulaci\'{o}n (3 semanas)
\item[UT3] Sistemas multivariados y complejos (4 semanas)
\end{description}

La sesiones son de cuatro horas cada una y son veinte semanas en
total. Son dos semanas introductorios al inicio del semestre y al
final se dedican seis semanas para el desarrollo y la revisi\'{o}n del
proyecto integrador.

\newpage

\paragraph{Temario semanal}

\quad


\begin{itemize}[itemsep=-3pt]
\item{Introducci\'{o}n (2 semanas); selecci\'{o}n de temas de
  proyecto}
\item{UT1: Movimiento Browniano (A1)}
\item{UT1: Aut\'{o}mata celular (A2)}
\item{UT1: Teor\'{\i}a de colas (A3)}
\item{UT1: Diagramas de Voronoi (A4)}
\item{UT1: M\'{e}todo Monte-Carlo (A5)}
\item{UT2: Sistema multiagente (A6)}
\item{UT2: B\'{u}squeda local (A7)}
\item{UT2: Modelo de urnas (A8)}
\item{UT3: Interacciones entre part\'{\i}culas (A9)}
\item{UT3: Algoritmo gen\'{e}tico (A10)}
\item{UT3: Frentes de Pareto (A11)}
\item{UT3: Red neuronal (A12)}
\item{Desarrollo de proyectos (4 semanas)}  
\item{Presentaciones de proyectos}
\item{Revisi\'{o}n de portafolios de evidencia}
\end{itemize}


\subsubsection{Unidad tem\'{a}tica 1: Conceptos b\'{a}sicos para modelado matem\'{a}tico}

\paragraph{Periodo:} 5 semanas

\paragraph{Elementos de competencia:}

\quad

\begin{tabular}{|p{28mm}|p{30mm}|p{30mm}|p{30mm}|p{30mm}|}
  \hline
  \cellcolor{tableheader}Evidencias de aprendizaje 
  & \cellcolor{tableheader}Criterios de desempe\~{n}o
  & \cellcolor{tableheader}Actividades de aprendizaje
  & \cellcolor{tableheader}Contenidos
  & \cellcolor{tableheader}Recursos \\ \hline


Cinco (5) tareas semanales, siendo cada uno un reporte escrito y
c\'{o}digo de la implementaci\'{o}n de una simulaci\'{o}n
computacional de un modelo fundamental.

&

Calidad de la redacci\'{o}n cient\'{\i}fica del reporte; 
precisi\'{o}n de la simulaci\'{o}n desarrollada;
eficiencia de la implementaci\'{o}n de la simulaci\'{o}n;
cobertura de la experimentaci\'{o}n.

&

Experimentaci\'{o}n con ejemplos; lectura de material de apoyo;
modificaci\'{o}n de ejemplos; dise\~{n}o y ejecuci\'{o}n de experimentos;
an\'{a}lisis y reportaje de resultados obtenidos.

&

Modelos b\'{a}sicos diversos.

&

Material en la p\'{a}gina web de la unidad y la literatura citada;
lenguaje R o Python; paquete {\LaTeX} para redacci\'{o}n cient\'{\i}fica;
repositorios de GitHub. \\ \hline

\end{tabular}

\newpage

\subsubsection{Unidad tem\'{a}tica 2: T\'{e}cnicas adicionales de la simulaci\'{o}n}

\paragraph{Periodo:} 3 semanas

\paragraph{Elementos de competencia:}

\quad

\begin{tabular}{|p{28mm}|p{30mm}|p{30mm}|p{30mm}|p{30mm}|}
  \hline
  \cellcolor{tableheader}Evidencias de aprendizaje 
  & \cellcolor{tableheader}Criterios de desempe\~{n}o
  & \cellcolor{tableheader}Actividades de aprendizaje
  & \cellcolor{tableheader}Contenidos
  & \cellcolor{tableheader}Recursos \\ \hline


Tres (3) tareas semanales, siendo cada uno un reporte escrito y
c\'{o}digo de la implementaci\'{o}n de una simulaci\'{o}n
computacional con t\'{e}cnicas diversas.

&

Calidad de la redacci\'{o}n cient\'{\i}fica del reporte; 
precisi\'{o}n de la simulaci\'{o}n desarrollada;
eficiencia de la implementaci\'{o}n de la simulaci\'{o}n;
cobertura de la experimentaci\'{o}n.

&

Experimentaci\'{o}n con ejemplos; lectura de material de apoyo;
modificaci\'{o}n de ejemplos; dise\~{n}o y ejecuci\'{o}n de experimentos;
an\'{a}lisis y reportaje de resultados obtenidos.

&

T\'{e}cnicas adicionales de simulaci\'{o}n computacional de sistemas.

&

Material en la p\'{a}gina web de la unidad y la literatura citada;
lenguaje R o Python; paquete {\LaTeX} para redacci\'{o}n cient\'{\i}fica;
repositorios de GitHub. \\ \hline

\end{tabular}


\subsubsection{Unidad tem\'{a}tica 3: Sistemas multivariados y complejos}

\paragraph{Periodo:} 4 semanas

\paragraph{Elementos de competencia:}

\quad

\begin{tabular}{|p{28mm}|p{30mm}|p{30mm}|p{30mm}|p{30mm}|}
  \hline
  \cellcolor{tableheader}Evidencias de aprendizaje 
  & \cellcolor{tableheader}Criterios de desempe\~{n}o
  & \cellcolor{tableheader}Actividades de aprendizaje
  & \cellcolor{tableheader}Contenidos
  & \cellcolor{tableheader}Recursos \\ \hline


Cinco (5) tareas semanales, siendo cada uno un reporte escrito y
c\'{o}digo de la implementaci\'{o}n de una simulaci\'{o}n
computacional de un sistema multivariado y/o complejo.

&

Calidad de la redacci\'{o}n cient\'{\i}fica del reporte; 
precisi\'{o}n de la simulaci\'{o}n desarrollada;
eficiencia de la implementaci\'{o}n de la simulaci\'{o}n;
cobertura de la experimentaci\'{o}n.

&

Experimentaci\'{o}n con ejemplos; lectura de material de apoyo;
modificaci\'{o}n de ejemplos; dise\~{n}o y ejecuci\'{o}n de experimentos;
an\'{a}lisis y reportaje de resultados obtenidos.

&

M\'{e}todos diversos de simulaci\'{o}n computacional de sistemas
multivariados y/o complejos.

&

Material en la p\'{a}gina web de la unidad y la literatura citada;
lenguaje R o Python; paquete {\LaTeX} para redacci\'{o}n cient\'{\i}fica;
repositorios de GitHub. \\ \hline

\end{tabular}

\newpage

\section{Evaluaci\'{o}n integral de procesos y productos:}

Las tareas son individuales; se recomienda estudiar juntos y
discutir las soluciones, pero no se tolera ning\'{u}n tipo de plagio
en absoluto, ni de otros estudiantes ni de la red ni de libros ---
toda referencia bibliogr\'{a}fica tiene que ser apropiadamente
citada. La entrega se realiza por un repositorio en GitHub que debe
reflejar todas las fases del trabajo. El
alumno selecciona su lenguaje de programaci\'{o}n para cada
tarea. Son 12 tareas (A1--A12) que reportan avances semanales de
aplicaci\'{o}n de la lectura de la semana para el proyecto del
estudiante, otorgando por m\'{a}ximo 6 puntos por tarea m\'{a}s dos
puntos por dos reto adicionales asociadas a ella:
\begin{description}[itemsep=-2pt]
\item[NP]{= tarea omitida}
\item[6]{= cumple con lo que se esperaba}
\item[5]{= cumple con que se esperaba con errores menores}
\item[4]{= cumple con lo que se esperaba con algunas omisiones}
\item[3]{= d\'{e}bil en alcance y/o calidad}
\item[2]{= d\'{e}bil en ambos alcance y calidad}
\item[1]{= sin contribuciones o m\'{e}ritos aunque fue entregada}
\item[0]{= completamente inadecuado en alzance y calidad}
\end{description}
  
El proyecto final (A13) otorga un m\'{a}ximo de 30 puntos, evaluados en los
siguientes rubros:
\begin{enumerate}[itemsep=-2pt]
\item{Complejidad del problema --- problemas retadores se califican
    m\'{a}s generosamente que problemas sencillos.}
\item{Fidelidad de la
    simulaci\'{o}n --- la incorporaci\'{o}n exitosa de m\'{u}ltiples aspectos del
    fen\'{o}meno estudiado dentro de la simulaci\'{o}n otorga una mayor
    cantidad de puntos que modelos simplificados o poco realistas.}
\item{Visualizaci\'{o}n de la informaci\'{o}n --- el uso de m\'{e}todos diversos e
    informativos para graficar los datos (ambos los originales y los
    generados por la simulaci\'{o}n, igual como datos de desempe\~{n}o de la
    misma simulaci\'{o}n) es premiado en la calificaci\'{o}n.}
\item{An\'{a}lisis estad\'{\i}stico (recomendado)--- aplicaci\'{o}n de
    m\'{e}todos rigurosos de examinaci\'{o}n estad\'{\i}stica, su
    selecci\'{o}n adecuada y su interpretaci\'{o}n correcta otorgan
    m\'{a}s puntos que el mero uso de medidas triviales o la ausencia
    total de an\'{a}lisis estad\'{\i}stico formal.}
\item{Grado y utilidad de paralelismo (opcional) --- la eficiente
    paralelizaci\'{o}n de todo lo que se pueda paralelizar sin
    p\'{e}rdida de eficiencia es altamente deseable; tambi\'{e}n se
    aprecian comparasiones de desempe\~{n}o entre versiones con y sin
    paralelismo igual como el efecto del n\'{u}mero de n\'{u}cleos
    disponibles.}
\item{Claridad y calidad de la redacci\'{o}n --- ortograf\'{\i}a,
    gram\'{a}tica, puntuaci\'{o}n, posicionamiento y formato de
    ecuaciones, cuadros y figuras, estructuraci\'{o}n de frases,
    p\'{a}rrafos, secciones, organizaci\'{o}n de la discusi\'{o}n,
    igual como el formato y la extensi\'{o}n de la bibliograf\'{\i}a
    entran en juego al calificar la calidad de un manuscrito.}
\end{enumerate}

con la escala:
\begin{description}[itemsep=0em]
\item[NP]{= tarea omitida}
\item[5]{= excede lo que se esperaba}
\item[4]{= cumple con lo que se esperaba}
\item[3]{= d\'{e}bil en alcance y/o calidad}
\item[2]{= d\'{e}bil en ambos alcance y calidad}
\item[1]{= sin contribuciones o m\'{e}ritos aunque fue entregada}
\item[0]{= completamente inadecuado en alzance y calidad}
\end{description}


Cada pr\'{a}ctica contiene una tarea que se califica en una escala de
0 a 10. La versi\'{o}n b\'{a}sica de la tarea puede otorgar hasta seis
puntos; hay dos retos opcionales para subir la calificaci\'{o}n hasta
ocho (un reto cumplido --- cumplimiento parcial permite subir hasta
siete) o diez (ambos retos cumplidos --- cumplimiento parcial permite
subir hasta nueve) seg\'{u}n la cantidad de retos cumplidos. En
algunas tareas el segundo reto depende del primero, mientras en otras
se pueden llevar a cabo de manera independiente.  Los puntos obtenidos
de las tareas se suman directamente con los puntos obtenidos del
proyecto (m\'{a}ximo 30) para formar la calificaci\'{o}n final de la
siguiente manera: $C$ es el m\'{\i}nimo entre $T + P$ y 100, donde $T$
es la suma de los puntos de las tareas y $P$ es el puntaje del
proyecto.  Calificaciones $C \geq 80$ son aprobatorias. No se otorgan
fracciones de puntos en ninguna actividad por lo cual no se requiere
ninguna regla de redondeo. No hay ex\'{a}menes.

\paragraph{Ponderaci\'{o}n espec\'{\i}fica}

\quad

\scalebox{0.9}{
  \begin{tabular}{|c|ccccccccccccc|c|}
    \hline
    
    \rotatebox{90}{\cellcolor{tableheader}{\bf Actividad\phantom{xx}}}
    & A1
    & A2
    & A3
    & A4
    & A5
    & A6
    & A7
    & A8
    & A9
    & A10
    & A11
    & A12
    & A13
    & Total 
    \\
    \hline
    \rotatebox{90}{\cellcolor{tableheader}{\bf Ponderaci\'{o}n\phantom{xx}}}

    & 6\%
    & 6\%
    & 6\%
    & 6\%
    & 6\%
    & 6\%
    & 6\%
    & 6\%
    & 7\%
    & 7\%
    & 7\%
    & 7\%
    & 30\%
    & 100\%
    \\ \hline
  \end{tabular}}


\section{Producto integrador de aprendizaje de la unidad:}

\subsection{Producto integrador de Aprendizaje:}

\quad

Portafolio en un repositorio digital p\'{u}blico que contiene los
reportes escritos y los c\'{o}digos de la implementaci\'{o}n de todas
las tareas y el proyecto integrador.


\section{Fuentes de apoyo y consulta:}
\subsection{Fuentes de apoyo y consulta}
\subsubsection{B\'{a}sicas}

\begin{itemize}[itemsep=0em]
  
\item{M.\ {\sc Templ}: {\em Simulation for Data Science with R},
    2016. Packt Publishing. 978-1785881169.}
  
\item{O.\ {\sc Jones}, R.\ {\sc Maillardet} \& A.\ {\sc
      Robinson}: {\em Introduction to Scientific Programming and
      Simulation Using R}, Chapman \& Hall/CRC, 2nd Edition, 2014,
    978-1466569997.}

\item{{\sc R Project}: {\em Documentation}, \url{https://www.r-project.org/other-docs.html}}
  
\item{R.I.\ {\sc Kabacoff}: {\em Quick-R --- Accessing the power of R}, 2017. \url{http://www.statmethods.net}}

\end{itemize}

\subsubsection{Complementarias}

Art\'{\i}culos cient\'{\i}ficos especializados relacionados a los
temas tratados, de preferencia publicados en revistas internacionales
indizados recientes.


\label{final} % last page
%\newpage

%\pagestyle{plain}

%\vspace*{3cm}

%{\bf Autoriz\'{o}:} \coordinador

%\vspace*{2cm}

%  \begin{center}
%  {\sc Alere Flammam Veritatis}
  
%  Ciudad Universitaria, \today

%\vspace*{4cm}
  
%  \begin{tabular}{p{6cm}cp{7cm}}
%    \cline{1-1}
%    \cline{3-3}    
%    {\bf \coordinador}
% & \phantom{xxx} &{\bf Vo.\ Bo.\ \subdirector} \\
%    Coordinador Acad\'{e}mico &  &Subdirector de Estudios de Posgrado \\
%    Posgrado en Ingenier\'{\i}a de Sistemas & & Facultad de Ingenier\'{\i}a Mec\'{a}nica y El\'{e}ctrica
                                                                   
%  \end{tabular}
%\end{center}

\end{document}
