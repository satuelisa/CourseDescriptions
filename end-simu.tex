\section{Estructuraci\'{o}n en cap\'{\i}tulos, etapas o fases de la unidad de
  aprendizaje:}
\subsection{Desarrollo de las fases de la Unidad de Aprendizaje:}

\quad

Se cubren y aplican los principios te\'{o}ricos de la {\em
  simulaci\'{o}n computacional}. Se busca desarrollar habilidades en
la resoluci\'{o}n en casos pr\'{a}cticos concretos. Se necesita contar
con un buen entendimiento de varios conceptos matem\'{a}ticos,
especialmente de matem\'{a}ticas discretas y probabilidad, o en el
caso contrario, estar preparado a estudiarlos seg\'{u}n
necesidad. Tambi\'{e}n es conveniente contar con conocimiento de
programaci\'{o}n.

\paragraph{Unidades tem\'{a}ticas}

\begin{enumerate}[itemsep=-3pt]
\item Conceptos b\'{a}sicos para modelado matem\'{a}tico
\item T\'{e}cnicas adicionales de la simulaci\'{o}n
\item Sistemas multivariados y complejos
\end{enumerate}

\newpage

\paragraph{Temario semanal}

\quad

La sesiones son de cuatro horas cada una y son veinte semanas en
total. Al inicio hay una fase introductoria breve, seguida por la fase de
tareas en la cual se aprenden las t\'{e}cnicas, y concluyendo al final
con la tercera fase del desarrollo del proyecto individual para
aplicar las t\'{e}cnicas aprendidas a un problema particular.

\begin{itemize}[itemsep=-3pt]
\item{Introducci\'{o}n (2 semanas); selecci\'{o}n de temas de
  proyecto}
\item{UT1: Movimiento Browniano (A1)}
\item{UT1: Aut\'{o}mata celular (A2)}
\item{UT1: Teor\'{\i}a de colas (A3)}
\item{UT1: Diagramas de Voronoi (A4)}
\item{UT1: M\'{e}todo Monte-Carlo (A5)}
\item{UT2: Sistema multiagente (A6)}
\item{UT2: B\'{u}squeda local (A7)}
\item{UT2: Modelo de urnas (A8)}
\item{UT3: Interacciones entre part\'{\i}culas (A9)}
\item{UT3: Algoritmo gen\'{e}tico (A10)}
\item{UT3: Frentes de Pareto (A11)}
\item{UT3: Red neuronal (A12)}
\item{Desarrollo de proyectos (4 semanas)}  
\item{Presentaciones de proyectos}
\item{Revisi\'{o}n de portafolios de evidencia}
\end{itemize}

\paragraph{Elementos de competencia:}

\quad

\paragraph{Elementos de competencia:}

\quad

\begin{tabular}{|p{28mm}|p{30mm}|p{30mm}|p{30mm}|p{30mm}|}
  \hline
  \cellcolor{tableheader}Evidencias de aprendizaje 
  & \cellcolor{tableheader}Criterios de desempe\~{n}o
  & \cellcolor{tableheader}Actividades de aprendizaje
  & \cellcolor{tableheader}Contenidos
  & \cellcolor{tableheader}Recursos \\ \hline


Reporte escrito y c\'{o}digo de la implementaci\'{o}n de una
simulaci\'{o}n computacional.

&

Calidad de la redacci\'{o}n cient\'{\i}fica del reporte; 
precisi\'{o}n de la simulaci\'{o}n desarrollada;
eficiencia de la implementaci\'{o}n de la simulaci\'{o}n;
cobertura de la experimentaci\'{o}n.

&

Experimentaci\'{o}n con ejemplos; lectura de material de apoyo;
modificaci\'{o}n de ejemplos; dise\~{n}o y ejecuci\'{o}n de experimentos;
an\'{a}lisis y reportaje de resultados obtenidos.

&

M\'{e}todos diversos de simulaci\'{o}n computacional de sistemas.

&

Material en la p\'{a}gina web de la unidad y la literatura citada;
lenguaje R o Python; paquete {\LaTeX} para redacci\'{o}n cient\'{\i}fica;
repositorios de GitHub. \\ \hline

\end{tabular}

\newpage

\scalebox{0.9}{
  \begin{tabular}{|c|ccccccccccccc|c|}
    \hline
    
    \rotatebox{90}{\cellcolor{tableheader}{\bf Actividad\phantom{xx}}}
    & A1
    & A2
    & A3
    & A4
    & A5
    & A6
    & A7
    & A8
    & A9
    & A10
    & A11
    & A12
    & A13
    & Total 
    \\
    \hline
    \rotatebox{90}{\cellcolor{tableheader}{\bf Ponderaci\'{o}n\phantom{xx}}}

    & 6\%
    & 6\%
    & 6\%
    & 6\%
    & 6\%
    & 6\%
    & 6\%
    & 6\%
    & 7\%
    & 7\%
    & 7\%
    & 7\%
    & 30\%
    & 100\%
    \\ \hline
  \end{tabular}}


\section{Producto integrador de aprendizaje de la unidad:}

\subsection{Producto integrador de Aprendizaje:}

\quad

Portafolio en un repositorio digital p\'{u}blico que contiene los
reportes escritos y los c\'{o}digos de la implementaci\'{o}n de todas
las tareas y el proyecto integrador.


\section{Fuentes de apoyo y consulta:}
\subsection{Fuentes de apoyo y consulta}
\subsubsection{B\'{a}sicas}

\begin{itemize}[itemsep=0em]
  
\item{M.\ {\sc Templ}: {\em Simulation for Data Science with R},
    2016. Packt Publishing. 978-1785881169.}
  
\item{O.\ {\sc Jones}, R.\ {\sc Maillardet} \& A.\ {\sc
      Robinson}: {\em Introduction to Scientific Programming and
      Simulation Using R}, Chapman \& Hall/CRC, 2nd Edition, 2014,
    978-1466569997.}

\item{{\sc R Project}: {\em Documentation}, \url{https://www.r-project.org/other-docs.html}}
  
\item{R.I.\ {\sc Kabacoff}: {\em Quick-R --- Accessing the power of R}, 2017. \url{http://www.statmethods.net}}

\end{itemize}

\subsubsection{Complementarias}

Art\'{\i}culos cient\'{\i}ficos especializados relacionados a los
temas tratados, de preferencia publicados en revistas internacionales
indizados recientes.


\label{final} % last page
\newpage

\pagestyle{plain}

\vspace*{3cm}

{\bf Autoriz\'{o}:} \coordinador

\vspace*{2cm}

  \begin{center}
  {\sc Alere Flammam Veritatis}
  
  Ciudad Universitaria, \today

\vspace*{4cm}
  
  \begin{tabular}{p{6cm}cp{7cm}}
    \cline{1-1}
    \cline{3-3}    
    {\bf \coordinador} &
                                                            \phantom{xxx} &{\bf Vo.\ Bo.\ \subdirector} \\
    Coordinador Acad\'{e}mico &  &Subdirector de Estudios de Posgrado \\
    Posgrado en Ingenier\'{\i}a de Sistemas & & Facultad de Ingenier\'{\i}a Mec\'{a}nica y El\'{e}ctrica
                                                                   
  \end{tabular}
\end{center}

\label{final} % last page
\newpage

\pagestyle{plain}

\vspace*{3cm}

{\bf Autoriz\'{o}:} \coordinador

\vspace*{2cm}

  \begin{center}
  {\sc Alere Flammam Veritatis}
  
  Ciudad Universitaria, \today

\vspace*{4cm}
  
  \begin{tabular}{p{6cm}cp{7cm}}
    \cline{1-1}
    \cline{3-3}    
    {\bf \coordinador} &
                                                            \phantom{xxx} &{\bf Vo.\ Bo.\ \subdirector} \\
    Coordinador Acad\'{e}mico &  &Subdirector de Estudios de Posgrado \\
    Posgrado en Ingenier\'{\i}a de Sistemas & & Facultad de Ingenier\'{\i}a Mec\'{a}nica y El\'{e}ctrica
                                                                   
  \end{tabular}
\end{center}

\end{document}
