\documentclass[10 pt]{article}
\usepackage{wallpaper}
\usepackage[spanish, mexico]{babel}  
\usepackage{color,colortbl}
\usepackage[T1]{fontenc}
\usepackage{fancyhdr} % before geometry
\usepackage[letterpaper,left=18mm,right=18mm,headheight=5mm,headsep=40mm,top=52mm,bottom=22mm]{geometry}
\usepackage{graphicx}
\usepackage{rotating}
\usepackage[latin1]{inputenc}
\usepackage{hyperref}
\usepackage{graphics}
\usepackage{varwidth}
\usepackage{tikz}

\usepackage{tcolorbox}
\definecolor{headerframe}{RGB}{177,178,177} % #b1b2b1
\definecolor{headercontent}{RGB}{239,239,239} % #efefef
\definecolor{tableheader}{RGB}{185,208,238}% #b9d0ee
\definecolor{evidence}{RGB}{238,208,185}
\definecolor{perfil}{RGB}{208,238,185}
\definecolor{unidad}{RGB}{208,185,185} 

\usetikzlibrary{shapes,arrows}
\tikzstyle{elem} = [scale=0.7, draw, rectangle, thick, minimum height=2em,
minimum width=2em, execute at begin node={\begin{varwidth}{12em}},
  execute at end node={\end{varwidth}}]
\tikzstyle{perf} = [draw, rectangle, thick, minimum height=2em,
minimum width=2em, fill=perfil]
\tikzstyle{comp} = [draw, rectangle, thick, minimum height=2em,
minimum width=2em, fill=tableheader]
\tikzstyle{esp} = [scale=0.7, draw, rectangle, thick, minimum height=2em,
minimum width=2em, fill=tableheader]
\tikzstyle{unidad} = [scale=0.9, draw, rectangle, thick, minimum height=2em,
minimum width=2em, fill=evidence]
\tikzstyle{evid} = [scale=0.75, draw, rectangle, thick, minimum height=2em,
minimum width=2em, fill=unidad]
\tikzstyle{header} = [scale=0.8, minimum height=2em, minimum width=2em, execute at begin node={\begin{varwidth}{4em}},
  execute at end node={\end{varwidth}}]


\tikzstyle{line} = [draw, thick, -stealth, shorten >=0pt]
\usepackage{multicol}
\usepackage{wrapfig}
\usepackage{enumitem}
\renewcommand*\familydefault{\sfdefault} 
\renewcommand{\thesection}{\Roman{section}} 
\usepackage{remreset}

\makeatletter
\@removefromreset{subsection}{section}
\renewcommand\thesubsection{\arabic{subsection}}
\makeatother

\makeatletter
\@removefromreset{subsubsection}{section} % no number
\renewcommand\thesubsubsection{}
\makeatother

\usepackage{titlesec}
\titleformat{\subsubsection}[runin]
{\bfseries}{\thesubsubsection}{0em}{} % in boldface

\makeatletter
\@removefromreset{theparagraph}{paragraph} % no number
\renewcommand\theparagraph{}
\makeatother

\usepackage{titlesec}
\titleformat{\paragraph}[runin]
{\itshape}{\theparagraph}{1em}{} % in italics

\ULCornerWallPaper{1}{logos.pdf}
\setlength{\parindent}{1em}
\setlength{\parskip}{2pt}
\tcbset{
  arc = 3mm,
  colframe = headerframe,
  colback = headercontent,
  fonttitle=\sffamily
}

\usepackage{amssymb}
\newcommand{\yes}{\makebox[0pt][l]{$\square$}{\raisebox{0.1\height}{$\times$}}}
\newcommand{\no}{\makebox[0pt][l]{$\square$}{\raisebox{0.1\height}{\phantom{$\times$}}}}
\usepackage{enumitem}
\titleformat{\section}{\normalfont\large\bfseries}{\thesection.}{4pt}{}
\titleformat{\subsection}[runin]{\normalfont\bfseries}{\thesubsection.}{4pt}{}


\newcommand{\UANL}{UNIVERSIDAD AUT\'{O}NOMA DE NUEVO LE\'{O}N}
\newcommand{\uanl}{Universidad Aut\'{o}noma de Nuevo Le\'{o}n}
\newcommand{\fime}{Facultad de Ingenier\'{\i}a Mec\'{a}nica y El\'{e}ctrica}
\newcommand{\maestria}{Maestr\'{\i}a en Ciencias de la Ingenier\'{\i}a con Orientaci\'{o}n en Sistemas}
\newcommand{\doctorado}{Doctorado en Ingenier\'{\i}a de Sistemas}
\newcommand{\PA}{PROGRAMA ANAL\'{I}TICO}

% LGAC
\newcommand{\seys}{Sistemas estoc\'{a}sticos y simulaci\'{o}n}
\newcommand{\mado}{M\'{e}todos avanzados de optimizaci\'{o}n}
\newcommand{\odsi}{Optimizaci\'{o}n de sistemas industriales}

% AREAS CURRICULARES
\newcommand{\fb}{formaci\'{o}n b\'{a}sica} % maestria
\newcommand{\fa}{formaci\'{o}n avanzada} % maestria
\newcommand{\fr}{formaci\'{o}n} % doctorado
\renewcommand{\div}{divulgaci\'{o}n}
\newcommand{\inv}{investigaci\'{o}n}
\renewcommand{\pi}{producto integrador}
\newcommand{\da}{de aplicaci\'{o}n}
\renewcommand{\le}{libre elecci\'{o}n}

% CLAVES DE LAS UNIDADES

\newcommand{\algm}{PM109} % analisis y diseno de algoritmos
\newcommand{\algd}{PD109}
\newcommand{\aprm}{PM134} % aprendizaje automatico
\newcommand{\aprd}{PD134}
\newcommand{\dsm}{PM123} % ciencia de datos
\newcommand{\dsd}{PD123}
\newcommand{\ccm}{PM135} % complejidad computacional
\newcommand{\ccd}{PD135} 
\newcommand{\iam}{PM101} % inteligencia artificial
\newcommand{\iad}{PD101}
\newcommand{\cvm}{PM124} % procesamiento de imagenes y vision computacional
\newcommand{\cvd}{PD124}
\renewcommand{\sim}{PM201} % seminarios
\newcommand{\siim}{PM202}
\newcommand{\sid}{PD201}
\newcommand{\siid}{PD202}
\newcommand{\siiid}{PD203}
\newcommand{\sivd}{PD204}
\newcommand{\svd}{PD205}
\newcommand{\svid}{PD206}
\newcommand{\sviid}{PD207}
\newcommand{\sviiid}{PD208}
\newcommand{\ssm}{PM105} % simulacion de sistemas
\newcommand{\ssd}{PD105}
\newcommand{\tim}{PM501} % tesis
\newcommand{\tiim}{PM502}
\newcommand{\tid}{PD501}
\newcommand{\tiid}{PD502}
\newcommand{\tiiid}{PD503}
\newcommand{\tivd}{PD504}
\newcommand{\tvd}{PD505}
\newcommand{\tvid}{PD506}
\newcommand{\tviid}{PD507}
\newcommand{\tviiid}{PD508} 

\newcommand{\narturo}{095012}
\newcommand{\nelisa}{096633}
\newcommand{\ncesar}{092038}
\newcommand{\nangy}{102662}
\newcommand{\nvincent}{102947}
\newcommand{\nada}{060581}
\newcommand{\niris}{103743}
\newcommand{\nsara}{100546}
\newcommand{\nroger}{090969}
\newcommand{\nigor}{093179}
\newcommand{\nromeo}{100959}
\newcommand{\nferny}{095808}
\newcommand{\arturo}{Dr.\ Jos\'{e} Arturo Berrones Santos}
\newcommand{\elisa}{Dra.\ Satu Elisa Schaeffer}
\newcommand{\cesar}{Dr.\ C\'{e}sar Emilio Villarreal Rodr\'{\i}guez}
\newcommand{\angy}{Dra.\ Mar\'{\i}a Ang\'{e}lica Salazar Aguilar}
\newcommand{\vincent}{Dr.\ Vincent Boyer}
\newcommand{\ada}{Dra.\ Ada Margarita \'{A}lvarez Socarr\'{a}s}
\newcommand{\iris}{Dra.\ Iris Abril Mart\'{\i}nez Salazar}
\newcommand{\sara}{Dra.\ Sara Ver\'{o}nica S\'{a}nchez Rodr\'{\i}guez}
\newcommand{\roger}{Dr.\ Roger Zirahu\'{e}n R\'{\i}os Mercado}
\newcommand{\igor}{Dr.\ Igor Litvinchev}
\newcommand{\romeo}{Dr.\ Romeo S\'{a}nchez Nigenda}
\newcommand{\ferny}{Dr.\ Fernando L\'{o}pez Irarragorri}
\newcommand{\simon}{Dr.\ Sim\'{o}n Mart\'{\i}nez Mart\'{\i}nez} 

\newcommand{\subdirector}{\simon}
\newcommand{\coordinador}{\cesar}


\begin{document}

\pagestyle{fancy}
\renewcommand{\headrulewidth}{0pt}
\fancyhf{}
\fancyhead[L]{}
\fancyhead[C]{}
\fancyhead[R]{IT-8-SPG-02-R03}
\fancyfoot[L]{Revisi\'{o}n: 1 \\
  Vigente a partir del: 01 de agosto del 2016}
\fancyfoot[C]{}
\fancyfoot[R]{P\'{a}gina~\thepage~de~\pageref*{final}}


\fancypagestyle{plain}{%
\fancyhf{}
\fancyhead[L]{}
\fancyhead[C]{}
\fancyhead[R]{IT-8-SPG-02-R03}
\fancyfoot[L]{Revisi\'{o}n: 1 \\
  Vigente a partir del: 01 de agosto del 2016}
\fancyfoot[C]{}
\fancyfoot[R]{}}

\begin{tcolorbox}
  \begin{center}
    {\bf UNIVERSIDAD AUT\'{O}NOMA DE NUEVO LE\'{O}N}

    \medskip

    {\bf Facultad de Ingenier\'{\i}a Mec\'{a}nica y El\'{e}ctrica}

    \medskip
    
    {\bf PE} \underline{\bf Doctorado en Ingenier\'{\i}a de Sistemas}

    \medskip

    \underline{PROGRAMA ANAL\'{I}TICO}
  \end{center}
\end{tcolorbox}
 % o headerMs.tex si es maestria
\section{Datos de Identificaci\'{o}n de la Unidad de Aprendizaje:}
\subsection{Nombre:} \fbox{Simulaci\'{o}n de sistemas}
\subsection{Frecuencia semanal:} horas de trabajo presencial \fbox{4}
\subsection{Horas de trabajo extra aula por semana:} \fbox{2}
\subsection{Modalidad:} \yes~Escolarizada \no~No escolarizada \no~Mixto
\subsection{Per\'{\i}odo acad\'{e}mico:} \yes~Semestral
\no~Tetramestral \no~Modular
\subsection{LGAC:} \underline{\seys}
\subsection{Ubicaci\'{o}n semestral:} \underline{1--4 \& 6--8}
\subsection{\'{A}rea curricular:} \underline{Formaci\'{o}n, optativa (1--4) o libre
  elecci\'{o}n (1--3)}
\subsection{Cr\'{e}ditos:} \underline{4}
\subsection{Requisito:} \underline{Ninguno}
\subsection{Fecha de elaboraci\'{o}n:} \underline{20/01/2010}
\subsection{Fecha de la \'{u}ltima actualizaci\'{o}n:} \underline{09/06/2021}
\subsection{Responsable(s) del dise\~{n}o:}
\begin{itemize}[label={}]
\item \underline{\nsara~\sara}
\item \underline{\nelisa~\elisa}
\end{itemize}
\newpage
\section{Presentaci\'{o}n:}

Simulaci\'{o}n refiere a la reproducci\'{o}n (computacional) de
fen\'{o}menos y procesos del mundo real. T\'{\i}picamente involucra
primero el modelado de dicho proceso o fen\'{o}meno a trav\'{e}s de
experimentos estad\'{\i}sticos. Despu\'{e}s la reproducci\'{o}n
involucra la generaci\'{o}n pseudo-aleatoria para crear los modelos en
el ambiente simulado. Sus aplicaciones son numerosas en temas
cient\'{\i}ficos, industriales, de construcci\'{o}n, de la salud,
evacuaciones, etc\'{e}tera. Las epidemias son particularmente
interesantes de estudiar con simulaciones. El an\'{a}lisis de elemento
finito (FEM) para aspectos estructurales es muy parecido a la
simulaci\'{o}n; se extrapola el comportamiento de una entidad a
trav\'{e}s de un modelo que lo parte a una cantidad discreta de
elementos peque\~{n}os de comportamiento ya modelado. La
simulaci\'{o}n tambi\'{e}n est\'{a} presente en el entretenimiento
(pel\'{\i}culas y videojuegos) en la creaci\'{o}n de ambientes
virtuales y efectos.

En la unidad de aprendizaje se realizan actividades de aprendizaje que
permiten la paralelizaci\'{o}n de algunas tareas fundamentales de
simulaci\'{o}n, desde m\'{a}s sencillos hasta m\'{a}s complejos, para
que el participante pueda en sus trabajos futuros identificar
oportunidades de paralelizaci\'{o}n y dominar las t\'{e}cnicas
b\'{a}sicas de llevarlo a cabo con elegancia y eficiencia. Se
implementan diversas simulaciones computacionales para generar y
analizar informaci\'{o}n de distintos tipos.

\section{Prop\'{o}sito(s):}

Formaci\'{o}n de competencia profesional de nivel posgrado que permita
el participante identificar de fen\'{o}menos que se pueda estudiar
v\'{\i}a la simulaci\'{o}n computacional, proponer modelos
matem\'{a}ticos que capturen dichos fen\'{o}menos a un nivel adecuado,
implementar simulaciones computacionales de estos modelos, adem\'{a}s
de dise\~{n}ar, ejecutar y analizar experimentos computacionales que
adecuadamente capturan la precisi\'{o}n de dichos simulaciones y
permiten conclusiones estad\'{\i}sticamente v\'{a}lidas sobre el
fen\'{o}meno en cuesti\'{o}n. Adem\'{a}s se incluye de forma integral
la preparaci\'{o}n de competencias en la visualizaci\'{o}n de la
informaci\'{o}n cient\'{\i}fica y la redacci\'{o}n cient\'{\i}fica.

\section{Competencias del perfil de egreso:}
\subsection{Competencias del perfil de egreso}

P1) Realizar investigaci\'{o}n original y resolver problemas en el \'{a}rea de
toma de decisiones en ambientes operativos que pueden ser din\'{a}micos o
inciertos para lograr una asignaci\'{o}n m\'{a}s efectiva de recursos y
decidir el curso de acci\'{o}n \'{o}ptimo para lograr objetivos establecidos.


\subsection{Competencias generales a que se vincula la Unidad de
    Aprendizaje:}

  La unidad se vincula con las siguientes competencias generales:

  \input{gcomptablestart.tex}
  C2) Utiliza los lenguajes l\'{o}gico, formal, matem\'{a}tico, ic\'{o}nico, verbal
y no verbal de acuerdo a su etapa de vida en el \'{a}rea de las ciencias
para comprender, interpretar y expresar ideas, sentimientos, teor\'{\i}as y
corrientes de pensamiento con un enfoque ecum\'{e}nico.
 & Tareas \\ \hline
  C3) Maneja las tecnolog\'{\i}as de la informaci\'{o}n de acuerdo a los usos del
campo de las ciencias y la comunicaci\'{o}n como herramientas para el
acceso a la informaci\'{o}n y su transformaci\'{o}n en conocimiento, as\'{\i} como
para el aprendizaje y trabajo colaborativo con t\'{e}cnicas de vanguardia
que le permitan su participaci\'{o}n constructiva en la sociedad.
 & Tareas \\ \hline
  C5) Emplea pensamiento l\'{o}gico, cr\'{\i}tico, creativo y propositivo,
siguiendo los modelos de pensamiento cient\'{\i}fico para analizar
fen\'{o}menos naturales y sociales que le permitan tomar decisiones
pertinentes en su \'{a}mbito de influencia con responsabilidad social.
 & Tareas \\ \hline
  C10) Practica los valores promovidos por la UANL: verdad, equidad,
honestidad, libertad, solidaridad, respeto a la vida y a los dem\'{a}s,
respeto a la naturaleza, integridad, \'{e}tica profesional, justicia y
responsabilidad, en su \'{a}mbito personal y profesional para contribuir a
construir una sociedad sostenible.
 & Tareas, proyecto \\ \hline
  C15) Logra la adaptabilidad que requieren los ambientes sociales y
profesionales de incertidumbre de nuestra \'{e}poca para crear mejores
condiciones de vida utilizando todos los avances cient\'{\i}ficos a los
cuales ha tenido acceso
 & Tareas, proyecto \\ \hline
  \end{tabular}

\newpage
    
\subsection{Competencias espec\'{\i}ficas y nivel de dominio a que se vincula la unidad de aprendizaje:}

La unidad se vincula con las siguientes competencias espec\'{\i}ficas:

\phantom{space}
\begin{tabular}{|p{32mm}|p{26mm}|p{20mm}|p{2cm}|p{20mm}|p{4mm}|p{4mm}|p{4mm}|p{4mm}|}
\hline
\cellcolor{tableheader}{{\em Competencia Espec\'{\i}fica}}
& \rotatebox{90}{\cellcolor{tableheader}{Nivel I Inicial}}
& \rotatebox{90}{\cellcolor{tableheader}{Evidencia}}
& \rotatebox{90}{\cellcolor{tableheader}{Nivel II B\'{a}sico}}
& \rotatebox{90}{\cellcolor{tableheader}{Evidencia}}
& \rotatebox{90}{\cellcolor{tableheader}{Nivel III Aut\'{o}nomo}}
& \rotatebox{90}{\cellcolor{tableheader}{Evidencia}}
& \rotatebox{90}{\cellcolor{tableheader}{Nivel IV Estrat\'{e}gico\phantom{xxx}}}
& \rotatebox{90}{\cellcolor{tableheader}{Evidencia}}
\\ \hline


E1) Realizar investigaci\'{o}n original y resolver problemas en el \'{a}rea de
toma de decisiones en ambientes operativos que pueden ser din\'{a}micos o
inciertos para lograr una asignaci\'{o}n m\'{a}s efectiva de recursos y
decidir el curso de acci\'{o}n \'{o}ptimo para lograr objetivos establecidos.

& Interpreta y aplica correctamente los principios de la toma de
decisiones con bases cient\'{\i}ficas en sistemas determin\'{\i}sticos
o estoc\'{a}sticos.
 & Tareas.
& Resuelve problemas de libro de texto en el \'{a}rea de toma de decisiones
con bases cient\'{\i}ficas
 & Tareas.
& & & & \\ \hline  
\end{tabular}

\newpage

\section{Representaci\'{o}n gr\'{a}fica:}

\begin{center}
\begin{tikzpicture}[scale=1, auto]
  \matrix[row sep=1cm, column sep=7mm]{

      & 
      \node[elem] (ide) {Identificar un fen\'{o}meno de inter\'{e}s que se
        busca simular};
      \\
      \node[elem] (lit) {Identificar y revisar literatura cient\'{\i}fica relacionada relevante};
      &
      \node[elem] (des) {Desarrollar una simulaci\'{o}n computacional};
      &
      \node[elem] (her) {Revisar herramientas
        existentes para implementar la simulaci\'{o}n};
      \\
      \node[elem] (for) {Formular hip\'{o}tesis para la validaci\'{o}n
        de la simulaci\'{o}n};
      &
      \node[elem] (imp) {Implementar el m\'{e}todo con las herramientas seleccionadas};
      &
      \node[elem] (sel) {Seleccionar las herramientas seg\'{u}n las
        caracter\'{\i}sticas de la simulaci\'{o}n};
      \\
      & \node[elem] (dis) {Dise\~{n}ar y realizar experimentos};
      \\
      \node[elem] (ana) {Analizar los resultados de los experimentos y contrastarlas con las hip\'{o}tesis};
      & & 
      \node[elem] (doc) {Documentar el proceso conforme al estilo de
        redacci\'{o}n cient\'{\i}fica};
      \\
      & 
      \node[elem] (apl) {Aplicar visualizaci\'{o}n cient\'{\i}fica a los resultados};
      & 
      \node[elem] (exp) {Discutir entre pares los m\'{e}todos y los
        resultados obtenidos};
      \\
    };
    \draw [line] (ide) -- (lit);
    \draw [line] (ide) -- (des);
    \draw [line] (ide) -- (her);
    \draw [line] (lit) -- (des);
    \draw [line] (lit) -- (for);
    \draw [line] (des) -- (imp);
    \draw [line] (her) -- (sel);
    \draw [line] (sel) -- (imp);
    \draw [line] (for) -- (dis);
    \draw [line] (imp) -- (dis);
    \draw [line] (dis) -- (ana);
    \draw [line] (dis) -- (apl);
    \draw [line] (ana) -- (doc);
    \draw [line] (apl) -- (doc);
    \draw [line] (doc) -- (exp);
\end{tikzpicture}
\end{center}


\newpage
\section{Estructuraci\'{o}n en cap\'{\i}tulos, etapas o fases de la unidad de
  aprendizaje:}
\subsection{Desarrollo de las fases de la Unidad de Aprendizaje:}

Se cubren los principios te\'{o}ricos de la simulaci'n
computacional. Se busca desarrollar habilidades en la resoluci\'{o}n
en casos pr\'{a}cticos concretos. Se necesita contar con un buen
entendimiento de varios los conceptos matem\'{a}ticos, especialmente
de matem\'{a}ticas discretas y probabilidad, o en el caso contrario,
estar preparado a estudiarlos seg\'{u}n necesidad. Tambi\'{e}n es
convinte contar con conocimiento de programaci\'{o}n.

Al inicio hay una fase introductoria breve, seguida por la fase de
tareas en la cual se aprenden las t\'{e}cnicas, y concluyendo al final con
la tercera fase del desarrollo del proyecto individual para aplicar
las t\'{e}cnicas aprendidas a un problema particular. La sesiones son de cuatro horas cada una y son veinte
semanas en total.
\begin{enumerate}[itemsep=-3pt]
\item{Introducci\'{o}n (2 semanas); selecci\'{o}n de temas de proyecto}
\item{Movimiento Browniano (A1)}
\item{Aut\'{o}mata celular (A2)}
\item{Teor\'{\i}a de colas (A3)}
\item{Diagramas de Voronoi (A4)}
\item{M\'{e}todo Monte-Carlo (A5)}
\item{Sistema multiagente (A6)}
\item{B\'{u}squeda local (A7)}
\item{Modelo de urnas (A8)}
\item{Interacciones entre part\'{\i}culas (A9)}
\item{Algoritmo gen\'{e}tico (A10)}
\item{Frentes de Pareto (A11)}
\item{Red neuronal (A12)}
\item{Desarrollo de proyectos (4 semanas)}  
\item{Presentaciones de proyectos}
\item{Revisi\'{o}n de portafolios de evidencia}
\end{enumerate}

{\em Elementos de competencia:}

\paragraph{Elementos de competencia:}

\quad

\begin{tabular}{|p{28mm}|p{30mm}|p{30mm}|p{30mm}|p{30mm}|}
  \hline
  \cellcolor{tableheader}Evidencias de aprendizaje 
  & \cellcolor{tableheader}Criterios de desempe\~{n}o
  & \cellcolor{tableheader}Actividades de aprendizaje
  & \cellcolor{tableheader}Contenidos
  & \cellcolor{tableheader}Recursos \\ \hline

  Reporte escrito y/ c\'{o}digo de la implementaci\'{o}n de una
  simulaci\'{o}n computacional.
  & Calidad de la redacci\'{o}n cient\'{\i}fica del reporte; 
    precisi\'{o}n de la simulaci\'{o}n desarrollada;
    eficiencia de la implementaci\'{o}n de la simulaci\'{o}n;
    cobertura de la experimentaci\'{o}n.
  & Experimentaci\'{o}n con ejemplos; lectura de material de apoyo;
    modificaci\'{o}n de ejemplos; dise\~{n}o y ejecuci\'{o}n de experimentos;
    an\'{a}lisis y reportaje de resultados obtenidos.
  &
    M\'{e}todos diversos de simulaci\'{o}n computacional de sistemas.
  & Material en la p\'{a}gina web de la unidad y la literatura citada;
    lenguaje R o Python; paquete {\LaTeX} para redacci\'{o}n cient\'{\i}fica;
    repositorios de GitHub. \\ \hline
  
  \end{tabular}

\newpage
  
  \section{Evaluaci\'{o}n integral de procesos y productos:}
  
  Las tareas son individuales; se recomienda estudiar juntos y
  discutir las soluciones, pero no se tolera ning\'{u}n tipo de plagio
  en absoluto, ni de otros estudiantes ni de la red ni de libros ---
  toda referencia bibliogr\'{a}fica tiene que ser apropiadamente
  citada. La entrega se realiza por un repositorio en GitHub que debe
  reflejar todas las fases del trabajo en su log correspondiente. El
  alumno selecciona su lenguaje de programaci\'{o}n para cada
  tarea. Son 12 tareas (A1--A12) que reportan avances semanales de
  aplicaci\'{o}n de la lectura de la semana para el proyecto del
  estudiante, otorgando por m\'{a}ximo 6 puntos por tarea m\'{a}s dos
  puntos por dos reto adicionales asociadas a ella:
  \begin{description}[itemsep=-2pt]
  \item[NP]{= tarea omitida}
  \item[6]{= cumple con lo que se esperaba}
  \item[5]{= cumple con que se esperaba con errores menores}
  \item[4]{= cumple con lo que se esperaba con algunas omisiones}
  \item[3]{= d\'{e}bil en alcance y/o calidad}
  \item[2]{= d\'{e}bil en ambos alcance y calidad}
  \item[1]{= sin contribuciones o m\'{e}ritos aunque fue entregada}
  \item[0]{= completamente inadecuado en alzance y calidad}
  \end{description}
  El proyecto final (A13) otorga un m\'{a}ximo de 30 puntos, evaluados en los
  siguientes rubros
  \begin{enumerate}[itemsep=-1pt]

  \item{Complejidad del problema --- problemas retadores se califican
      m\'{a}s generosamente que problemas sencillos.}
   \item{Fidelidad de la
    simulaci\'{o}n --- la incorporaci\'{o}n exitosa de m\'{u}ltiples aspectos del
    fen\'{o}meno estudiado dentro de la simulaci\'{o}n otorga una mayor
    cantidad de puntos que modelos simplificados o poco realistas.}
  \item{Visualizaci\'{o}n de la informaci\'{o}n --- el uso de m\'{e}todos diversos e
    informativos para graficar los datos (ambos los originales y los
    generados por la simulaci\'{o}n, igual como datos de desempe\~{n}o de la
    misma simulaci\'{o}n) es premiado en la calificaci\'{o}n.}
  \item{An\'{a}lisis
    estad\'{\i}stico --- aplicaci\'{o}n de m\'{e}todos rigurosos de examinaci\'{o}n
    estad\'{\i}stica, su selecci\'{o}n adecuada y su interpretaci\'{o}n correcta
    otorgan m\'{a}s puntos que el mero uso de medidas triviales o la
    ausencia total de an\'{a}lisis estad\'{\i}stico formal.}
  \item{Grado y utilidad
    de paralelismo --- la eficiente paralelizaci\'{o}n de todo lo que se
    pueda paralelizar sin p\'{e}rdida de eficiencia es altamente deseable;
    tambi\'{e}n se aprecian comparasiones de desempe\~{n}o entre versiones con
    y sin paralelismo igual como el efecto del n\'{u}mero de n\'{u}cleos
    disponibles.}
  \item{Claridad y calidad de la redacci\'{o}n --- ortograf\'{\i}a,
    gram\'{a}tica, puntuaci\'{o}n, posicionamiento y formato de ecuaciones,
    cuadros y figuras, estructuraci\'{o}n de frases, p\'{a}rrafos, secciones,
    organizaci\'{o}n de la discusi\'{o}n, igual como el formato y la extensi\'{o}n
    de la bibliograf\'{\i}a entran en juego al calificar la calidad de un
    manuscrito.}
  \end{enumerate}
  con la escala:
  \begin{description}[itemsep=-2pt]
  \item[5]{= cumple con lo que se esperaba}
  \item[4]{= cumple con lo que se esperaba con errores menores}
  \item[3]{= cumple con lo que se esperaba con algunas omisiones menores}    
  \item[2]{= d\'{e}bil en alcance y/o calidad}
  \item[1]{= d\'{e}bil en ambos alcance y calidad}
  \item[0]{= inadecuado en alzance y calidad}
  \end{description}

  Cada pr\'{a}ctica contiene una tarea que se califica en una escala de 0
  a 10. La versi\'{o}n b\'{a}sica de la tarea puede otorgar hasta seis puntos;
  hay dos retos opcionales para subir la calificaci\'{o}n hasta ocho (un
  reto cumplido --- cumplimiento parcial permite subir hasta siete) o
  diez (ambos retos cumplidos --- cumplimiento parcial permite subir
  hasta nueve) seg\'{u}n la cantidad de retos cumplidos. En algunas tareas
  el segundo reto depende del primero, mientras en otras se pueden
  llevar a cabo de manera independiente.  Los puntos obtenidos de las
  tareas se suman directamente con los puntos obtenidos del proyecto
  (m\'{a}ximo 30) para formar la calificaci\'{o}n final de la siguiente
  manera: $C$ es el m\'{\i}nimo entre $T + P$ y 100, donde $T$ es la suma
  de los puntos de las tareas y $P$ es el puntaje del proyecto.
  Calificaciones $C \geq 80$ son aprobatorias. No se otorgan
  fracciones de puntos en ninguna actividad por lo cual no se requiere
  ninguna regla de redondeo. No hay ex\'{a}menes.
  
  Ponderaci\'{o}n espec\'{\i}fica:

  \scalebox{0.9}{
\begin{tabular}{|c|ccccccccccccc|c|}
  \hline

 \rotatebox{90}{\cellcolor{tableheader}{\bf Actividad\phantom{xx}}} & A1 & A2 & A3 & A4 & A5 & A6 & A7 & A8 & A9 & A10 &
                                                                      A11
  & A12 & PI & Total % Poner el nombre de la actividad                                                                                                                    
  \\
  \hline
  \rotatebox{90}{\cellcolor{tableheader}{\bf Ponderaci\'{o}n\phantom{xx}}}
                                             & 6\% & 6\% & 6\% & 6\%
                                                                 & 6\% &
                                                                       6\% & 6\% & 6\% & 7\%
                                                                                     & 7\%
                                                                                          & 7\%
                                                                                                & 7\%
        & 30\% & 100\%
        \\ \hline
\end{tabular}}
  
  \newpage

\section{Producto integrador de aprendizaje de la unidad:}
\subsection{Producto integrador de Aprendizaje:} Portafolio en un
repositorio digital p\'{u}blico que contiene las tareas y un proyecto
sobre la simulaci\'{o}n computacional (documentada) de alg\'{u}n fen\'{o}meno de
inter\'{e}s para la ingenier\'{\i}a de sistemas. Cada participante, una vez
concluida su participaci\'{o}n en las tareas, propone un tema de proyecto
a la profesora. Una vez iniciado el proyecto, ya no se permite
entregar tareas adicionales --- los participantes pueden iniciar el
proyecto cuando consideran que ya no necesitan tareas adicionales ni
para aprendizaje ni para aprobar el curso (se recomienda agendar por
lo menos dos semanas de trabajo para el proyecto y un buen proyecto
puede requerir hasta un mes). No se permite iniciar trabajos sobre el
proyecto antes de haber obtenido el permiso expl\'{\i}cito de la profesora;
es necesario que ella valide que el tema propuesto cubra los aspectos
requeridos del proyecto y que su complejidad no es ni demasiado baja
para un curso de este nivel ni demasiado alta para poder completarse
durante el curso.  El proyecto debe entregarse a la profesora antes de
que termine el periodo de curso, por lo menos tres d\'{\i}as h\'{a}biles antes
de que la captura de las calificaciones (la fecha exacta en cada
semestre depende del departamento escolar de posgrado). El reporte del
proyecto se estructura como un art\'{\i}culo cient\'{\i}fico, contando
con un t\'{\i}tulo, autores, un resumen y secciones para
introducci\'{o}n, antecedentes, trabajos relacionados, modelo
propuesto, implementaci\'{o}n de la simulaci\'{o}n, experimentos
(dise\~{n}o, resultados y discusi\'{o}n), conclusiones y trabajo a
futuro. Es obligatorio incluir una bibliograf\'{\i}a y citar de manera
adecuada a todas las fuentes de consulta.



\section{Fuentes de apoyo y consulta:}
\subsection{Fuentes de apoyo y consulta}
\subsubsection{B\'{a}sicas}

\begin{itemize}[itemsep=0em]
  
\item{Matthias {\sc Templ}: {\em Simulation for Data Science with R},
    2016. Packt Publishing. 978-1785881169.}
  
\item{Owen {\sc Jones}, Robert {\sc Maillardet} y Andrew {\sc
      Robinson}: {\em Introduction to Scientific Programming and
      Simulation Using R}, Chapman \& Hall/CRC, 2nd Edition, 2014,
    978-1466569997.}
   
\end{itemize}

\subsubsection{Complementarias}

\begin{itemize}[itemsep=0em]
  
\item{{\sc R Project}: {\em Documentation}, \url{https://www.r-project.org/other-docs.html}}
  
\item{Robert I.\ {\sc Kabacoff}: {\em Quick-R --- Accessing the power of R}, 2017. \url{http://www.statmethods.net}}
  
\end{itemize}

Art\'{\i}culos cient\'{\i}ficos especializados.

\label{final} % last page
%\newpage

%\pagestyle{plain}

%\vspace*{3cm}

%{\bf Autoriz\'{o}:} \coordinador

%\vspace*{2cm}

%  \begin{center}
%  {\sc Alere Flammam Veritatis}
  
%  Ciudad Universitaria, \today

%\vspace*{4cm}
  
%  \begin{tabular}{p{6cm}cp{7cm}}
%    \cline{1-1}
%    \cline{3-3}    
%    {\bf \coordinador}
% & \phantom{xxx} &{\bf Vo.\ Bo.\ \subdirector} \\
%    Coordinador Acad\'{e}mico &  &Subdirector de Estudios de Posgrado \\
%    Posgrado en Ingenier\'{\i}a de Sistemas & & Facultad de Ingenier\'{\i}a Mec\'{a}nica y El\'{e}ctrica
                                                                   
%  \end{tabular}
%\end{center}

\end{document}