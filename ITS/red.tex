\documentclass{article}
\usepackage[spanish, mexico]{babel}  
\usepackage{color,colortbl}
\usepackage[T1]{fontenc}
\usepackage{fancyhdr} % before geometry
\usepackage[paperheight=10in,paperwidth=25in,left=5mm,right=5mm,top=5mm,bottom=5mm]{geometry}
\usepackage{multicol}
\usepackage{graphicx}
\usepackage{rotating}
\usepackage[hidelinks]{hyperref}
\usepackage{graphics}
\usepackage{varwidth}
\usepackage{tikz}

\definecolor{gral}{HTML}{ffffff}
\definecolor{prog}{HTML}{99ff99}
\definecolor{meth}{HTML}{3399ff}
\definecolor{comp}{HTML}{99ff33}
\definecolor{math}{HTML}{9999ff}
\definecolor{phys}{HTML}{9933ff}
\definecolor{opta}{HTML}{ffcccc}
\definecolor{fina}{HTML}{aaaaaa}


\usetikzlibrary{shapes,arrows,calc}
\tikzstyle{obl} = [draw, rectangle, thick, minimum height=2em, minimum width=2em, fill=gral, execute at begin node={\begin{varwidth}{40mm}},
  execute at end node={\end{varwidth}}]
\tikzstyle{met} = [draw, rectangle, thick, minimum height=2em, minimum width=2em, fill=meth, execute at begin node={\begin{varwidth}{40mm}},
  execute at end node={\end{varwidth}}]  
\tikzstyle{mat} = [draw, rectangle, thick, minimum height=2em, minimum width=2em, fill=math, execute at begin node={\begin{varwidth}{40mm}},
  execute at end node={\end{varwidth}}]
\tikzstyle{phy} = [draw, rectangle, thick, minimum height=2em, minimum width=2em, fill=phys, execute at begin node={\begin{varwidth}{40mm}},
  execute at end node={\end{varwidth}}]
\tikzstyle{pro} = [draw, rectangle, thick, minimum height=2em, minimum width=2em, fill=prog, execute at begin node={\begin{varwidth}{40mm}},
  execute at end node={\end{varwidth}}]
\tikzstyle{com} = [draw, rectangle, thick, minimum height=2em, minimum width=2em, fill=comp, execute at begin node={\begin{varwidth}{40mm}},
  execute at end node={\end{varwidth}}]
\tikzstyle{fin} = [draw, rectangle, thick, minimum height=2em, minimum width=2em, fill=fina, execute at begin node={\begin{varwidth}{40mm}},
  execute at end node={\end{varwidth}}]

\tikzstyle{obl} = [draw, rectangle, thick, minimum height=2em, minimum width=2em, execute at begin node={\begin{varwidth}{40mm}},
  execute at end node={\end{varwidth}}]    

\tikzstyle{opt} = [draw, rounded rectangle, dashed, minimum height=2em, minimum width=2em, execute at begin node={\begin{varwidth}{40mm}},
  execute at end node={\end{varwidth}}]  

\tikzstyle{omet} = [draw, rounded rectangle, dashed, minimum height=2em, minimum width=2em, fill=meth, execute at begin node={\begin{varwidth}{40mm}},
  execute at end node={\end{varwidth}}]  
\tikzstyle{omat} = [draw, rounded rectangle, dashed, minimum height=2em, minimum width=2em, fill=math, execute at begin node={\begin{varwidth}{40mm}},
  execute at end node={\end{varwidth}}]
\tikzstyle{ophy} = [draw, rounded rectangle, dashed, minimum height=2em, minimum width=2em, fill=phys, execute at begin node={\begin{varwidth}{40mm}},
  execute at end node={\end{varwidth}}]
\tikzstyle{opro} = [draw, rounded rectangle, dashed, minimum height=2em, minimum width=2em, fill=prog, execute at begin node={\begin{varwidth}{40mm}},
  execute at end node={\end{varwidth}}]
\tikzstyle{ocom} = [draw, rounded rectangle, dashed, minimum height=2em, minimum width=2em, fill=comp, execute at begin node={\begin{varwidth}{40mm}},
  execute at end node={\end{varwidth}}]
\tikzstyle{ofin} = [draw, rounded rectangle, dashed, minimum height=2em, minimum width=2em, fill=fina, execute at begin node={\begin{varwidth}{40mm}},
  execute at end node={\end{varwidth}}]    


\tikzstyle{txt} = [scale=1, minimum height=1em, minimum width=2em]
\tikzstyle{sem} = [scale=1.3, minimum height=1em, minimum width=2em, execute at begin node={\begin{varwidth}{27mm}}, execute at end node={\end{varwidth}}]  
\tikzstyle{ele} = [scale=0.7, minimum height=1em, minimum width=2em]

\tikzstyle{line} = [draw, thick, -latex', shorten >=0pt]
\tikzstyle{one} = [draw, dotted, -latex', shorten >=0pt]

\tikzset{
    hyperlink node/.style={
        alias=sourcenode,
        append after command={
            let \p1 = (sourcenode.north west),
                \p2=(sourcenode.south east),
                \n1={\x2-\x1},
                \n2={\y1-\y2} in
            node [inner sep=0pt,outer sep=0pt,anchor=north west,at=(\p1)] {\hyperlink{#1}{\XeTeXLinkBox{\phantom{\rule{\n1}{\n2}}}}}
        }
    }
}


\usepackage{enumitem}
\begin{document}

\begin{center}
  \begin{tikzpicture}
  \matrix[row sep=8mm, column sep=1mm] at (0, 2) {
    \node[sem](s1){Primer semestre (27 cr\'{e}ditos)}; \\
    \node[phy](qg){Qu\'{\i}mica general \& lab (4)};\\
    \node[phy](f1){F\'{\i}sica I \& lab (4)};\\
    \node[mat](m1){Matem\'{a}ticas I: C\'{a}lculo diferencial (4)};\\
    \node[mat](api){\'{A}lgebra para ingenier\'{\i}a (3)};\\
    \node[pro](mdp){Metodolog\'{\i}a de programaci\'{o}n (4)};\\
    \node[obl](dpi){Dibujo para ingenier\'{\i}a (4)};\\
    \node[obl](cdlp){Cultura de la paz (2)};\\
    \node[obl](rsyds){Responsabilidad social y desarrollo sustentable (2)};\\
  };
    \matrix[row sep=8mm, column sep=1mm] at (5, 1) {
      \node[sem](s1){Segundo semestre (22 cr\'{e}ditos)}; \\
    \node[ele](e2){Elegir una optativa}; \\          
    \node[phy](f2){F\'{\i}sica II \& lab (4)};\\
    \node[mat](m2){Matem\'{a}ticas II: C\'{a}lculo integral (4)};\\
    \node[mat, hyperlink node=md](md){Matem\'{a}ticas discretas (4)};\\\\
    \node[opro, hyperlink node=lac](lac){Lenguaje C (4)};\\
    \node[opro, hyperlink node=pep](pep){Programaci\'{o}n en Python (4)};\\    
    \node[met](leei){Liderazgo, emprendimiento e innovaci\'{o}n (2)};\\   
    \node[obl](eycdl){\'{E}tica y cultura de legalidad (2)};\\
    \node[obl](cdg){Cultura de g\'{e}nero (2)};\\
  };
    \matrix[row sep=8mm, column sep=1mm] at (11, 2) {
    \node[sem](s1){Tercer semestre (22 cr\'{e}ditos)}; \\
    \node[phy](f3){F\'{\i}sica III \& lab (4)};\\
    \node[mat](m3){Matem\'{a}ticas III (3)};\\\\\\
    \node[com, hyperlink node=sd](sd){Sistemas digitales \& lab (3)};\\\\\\
    \node[pro, hyperlink node=poao](poao){Programaci\'{o}n orientada a objetos \& lab (4)};\\\\\\
    \node[pro, hyperlink node=ayedd](ayedd){Algoritmos y estructuras de datos (4)};\\
    \node[met](ca){Contabilidad administrativa (4)};\\
  };
    \matrix[row sep=8mm, column sep=1mm] at (17, 0) {
    \node[sem](s1){Cuarto semestre (22 cr\'{e}ditos)}; \\
    \node[mat, hyperlink node=pye](pye){Probabilidad y estad\'{\i}stica (3)};\\    
    \node[phy](f4){F\'{\i}sica IV \& lab (4)};\\
    \node[mat](m4){Matem\'{a}ticas IV (3)};\\    
    \node[com, hyperlink node=at](at){Arquitectura tecnol\'{o}gica (3)};\\
    \node[pro, hyperlink node=bdd](bdd){Bases de datos (3)};\\\\\\\\
    \node[com, hyperlink node=tydcc](tycdd){Transmisi\'{o}n y comunicaci\'{o}n de datos (3)};\\\\\\
    \node[met, hyperlink node=mdd](mdd){Metodolog\'{\i}as de desarrollo (3)};\\    
  };
    \matrix[row sep=8mm, column sep=1mm] at (22, 1) {
      \node[sem](s1){Quinto semestre (22 cr\'{e}ditos)}; \\
    \node[ele](e5){Elegir dos optativas}; \\              
    \node[mat, hyperlink node=dde](dde){Dise\~{n}o de experimentos (3)};\\    
    \node[pro, hyperlink node=sa](sa){Sistemas adaptativos (3)};\\
    \node[omat, hyperlink node=mn](mn){M\'{e}todos num\'{e}ricos (3)};\\    
    \node[com, hyperlink node=so](so){Sistemas operativos (4)};\\\\    
    \node[com, hyperlink node=rc](rc){Redes computacionales (3)};\\\\
    \node[opro, hyperlink node=vdi](vdi){Visualizaci\'{o}n de informaci\'{o}n (3)};\\            
    \node[opro, hyperlink node=pf](pf){Programaci\'{o}n funcional (3)};\\            
    \node[omet, hyperlink node=dyo](dyo){Desarrollo y operaciones (3)};\\
    \node[met, hyperlink node=ihc](ihc){Interacci\'{o}n humano-computadora (3)};\\\\\\
  };
    \matrix[row sep=8mm, column sep=1mm] at (27, 1) {
      \node[sem](s1){Sexto semestre (22 cr\'{e}ditos)}; \\
    \node[ele](e6){Elegir tres optativas}; \\                  
    \node[omat, hyperlink node=cdd](cdd){Ciencia de datos (3)};\\\\\\
    \node[ocom, hyperlink node=ayc](ayc){Automatizaci\'{o}n y control (3)};\\
    \node[com, hyperlink node=ci](ci){C\'{o}mputo integrado (3)};\\\\
    \node[mat, hyperlink node=o](o){Optimizaci\'{o}n (3)};\\\\
    \node[omat, hyperlink node=cr](cr){Criptograf\'{\i}a (3)};\\    
    \node[omat, hyperlink node=si](si){Seguridad inform\'{a}tica (3)};\\
    \node[pro, hyperlink node=aw](aw){Aplicaciones web (3)};\\\\
    \node[com, hyperlink node=iddm](iddm){Ingenier\'{\i}a de dispositivos m\'{o}viles \& lab (4)};\\
    \node[omet, hyperlink node=pdne](pdne){Planeaci\'{o}n de negocios emergentes (3)};\\    
    \node[omet, hyperlink node=da](da){Desarrollo \'{a}gil (3)};\\
  };
    \matrix[row sep=8mm, column sep=1mm] at (33, 2) {
    \node[sem](s1){S\'{e}ptimo semestre (22 cr\'{e}ditos)}; \\
    \node[ele](e7){Elegir tres optativas}; \\                  
    \node[omat, hyperlink node=mys](mys){Modelado y simulaci\'{o}n (3)};\\      
    \node[mat, hyperlink node=ia](ia){Inteligencia artificial (3)};\\                
    \node[ocom, hyperlink node=cya](cya){Controladores y actuadores (4)};\\
    \node[omat, hyperlink node=mh](mh){Metaheur\'{\i}sticas (3)};\\
    \node[com, hyperlink node=sdi](sdi){Sistemas distribuidos (3)};\\
    \node[pro, hyperlink node=cp](cp){C\'{o}mputo paralelo (3)};\\
    \node[opro, hyperlink node=ls](ls){Lenguajes script (3)};\\            
    \node[omat, hyperlink node=tdi](tdi){Teor\'{\i}a de informaci\'{o}n (3)};\\\\
    \node[omet, hyperlink node=idc](idc){Ingenier\'{\i}a de calidad (3)};\\    \\
    \node[fin, hyperlink node=pi1](pi1){Proyecto integrador I (4)};\\
  };
    \matrix[row sep=8mm, column sep=1mm] at (40, 2) {
    \node[sem](s1){Octavo semestre (22 cr\'{e}ditos)}; \\
    \node[ele](e8){Elegir cuatro optativas}; \\                      
    \node[com, hyperlink node=te](te){Tecnolog\'{\i}as emergentes (3)};\\
    \node[omat, hyperlink node=am](am){Aprendizaje m\'{a}quina (3)};\\
    \node[ocom, hyperlink node=idsa](idsa){Ingenier\'{\i}a de sistemas aut\'{o}nomos (3)};\\        
    \node[omat, hyperlink node=vf](vf){Verificaci\'{o}n formal (3)};\\
    \node[opro, hyperlink node=celn](celn){C\'{o}mputo en la nube (3)};\\
    \node[omet,hyperlink node=cdc](cdc){Confiabilidad de c\'{e}ntros de c\'{o}mputo (3)};\\
    \node[opro, hyperlink node=gc](gc){Gr\'{a}ficas computacionales (3)};\\    
    \node[pro, hyperlink node=vc](vc){Visi\'{o}n computacional (3)};\\
    \node[omat, hyperlink node=tdc](tdc){Teor\'{\i}a de la computaci\'{o}n (3)};\\        
    \node[omet, hyperlink node=ayp](ayp){Almacenaje y procesamiento de datos grandes (3)};\\        
    \node[fin, hyperlink node=pi2](pi2){Proyecto integrador II (4)};\\                
  };
    \matrix[row sep=8mm, column sep=1mm] at (47, -5) {
      \node[sem](s1){Noveno semestre (22 cr\'{e}ditos)}; \\
    \node[ele](e2){Elegir una optativa}; \\      
    \node[fin](ss){Servicio social (16)};\\
    \node[ofin, hyperlink node=cert](c1){Certificaci\'{o}n I (6)};\\
    \node[ofin](mo1){Movilidad I (6)};\\
    \node[ofin](i1){Investigaci\'{o}n I (6)};\\
    \node[ofin](le1){Libre elecci\'{o}n I (6)};\\
};
    \matrix[row sep=8mm, column sep=1mm] at (54, 0) {

      \node[sem](t0){Leyenda:}; & & & \node[obl](lobl){Unidad obligatoria (cr)};\\
      & & & \node[opt](lopt){Unidad optativa (cr)};\\
      \node[txt](t1){Requisito(s)}; & \node[obl](ej1){...}; & & \node[obl](ej2){...};\\
      \node[txt](t1){Cumplir con uno}; & \node[opt](ej3){...}; & & \node[obl](ej4){...};\\\\\\
      \node[sem](s1){D\'{e}cimo semestre (22 cr\'{e}ditos)}; \\
    \node[ele](e2){Elegir una optativa};\\      
    \node[fin](pp){Pr\'{a}cticas profesionales (16)};\\
    \node[ofin, hyperlink node=cert](c2){Certificaci\'{o}n II (6)};\\
    \node[ofin](mo2){Movilidad II (6)};\\
    \node[ofin](i2){Investigaci\'{o}n II$^\ast$ (6)};\\
    \node[ofin](le2){Libre elecci\'{o}n II (6)};\\    
    };
  \draw [line] (ej1) -- (ej2);
  \draw [one] (ej3) -- (ej4);        
  \draw [line] (api) -- (md);
  \draw [line] (md) -- (pye);
  \draw [line] (m1) -- (m2);
  \draw [line] (m2) -- (m3);
  \draw [line] (m3) -- (m4);
  \draw [line] (f1) -- (f2);
  \draw [line] (f2) -- (f3);
  \draw [line] (f3) -- (f4);
  \draw [line] (mdp) -- (lac);
  \draw [line] (mdp) -- (pep);
  \draw [line] (md) -- (sd);
  \draw [line] (ihc) -- (pdne);
  \draw [line] (mdd) -- (ihc);  
  \draw [line] (sd) -- (at);
  \draw [line] (sd) -- (tycdd);
  \draw [line] (ayedd) -- (tycdd);    
  \draw [line] (at) -- (so);
  \draw [line] (tycdd) -- (rc);
  \draw [line] (tycdd) -- (dyo);    
  \draw [line] (at) -- (rc);
  \draw [line] (so) -- (ci);
  \draw [line] (mdd) -- (ihc);
  \draw [line] (poao) -- (mdd);
  \draw [line] (poao) -- (pf);
  \draw [line] (pye) -- (sa);
  \draw [line] (pye) -- (dde);
  \draw [line] (dde) -- (cdd);
  \draw [line] (mdd) -- (da);
  \draw [line] (rc) -- (aw);
  \draw [line] (mn) -- (ayc);  
  \draw [line] (rc) -- (iddm);
  \draw [line] (rc) -- (si);
  \draw [line] (rc) -- (cr);      
  \draw [one] (lac) -- (ayedd);
  \draw [one] (pep) -- (ayedd);
  \draw [one] (lac) -- (poao);
  \draw [one] (pep) -- (poao);
  \draw [line] (poao) -- (bdd);
  \draw [line] (pi1) -- (pi2);
  \draw [line] (sa) -- (ia);
  \draw [line] (ci) -- (cp);
  \draw [line] (rc) -- (ci);
  \draw [line] (ci) -- (sdi);
  \draw [line] (i1) -- (i2);
  \draw [line] (pye) -- (mn);
  \draw [line] (bdd) -- (vdi);
  \draw [line] (ayc) -- (cya);
  \draw [line] (cya) -- (idsa);
  \draw [line] (aw) -- (pi1);
  \draw [line] (ia) -- (te);
  \draw [line] (m4) -- (o);  
  \draw [line] (cp) -- (vc);
  \draw [line] (cr) -- (tdi);
  \draw [line] (si) -- (idc);  
  \draw [line] (o) -- (mh);
  \draw [line] (cdd) -- (mys);
  \draw [line] (aw) -- (idc);
  \draw [line] (ci) -- (ls);
  \draw [line] (cp) -- (gc);
  \draw [line] (sdi) -- (cdc);
  \draw [line] (ia) -- (am);
  \draw [line] (tdi) -- (tdc);
  \draw [line] (sdi) -- (vf);
  \draw [line] (sdi) -- (celn);
  \draw [line] (pi2) -- (i1);
  \draw [line] (idc) -- (ayp);
  \draw [line] (ls) -- (ayp);  

  
\end{tikzpicture}
\end{center}

\newpage

\begin{multicols}{4}

\section*{Perfil de egreso}

El egresado ser\'{a} un ingeniero capaz de desarrollar software innovador
y de calidad para sistemas inteligentes, embebidos, m\'{o}viles y remotos,
con un fundamento s\'{o}lido te\'{o}rico, empleando metodolog\'{i}as y
herramientas de estado de arte, promoviendo la investigaci\'{o}n y el
desarrollo tecnol\'{o}gico, con la finalidad de proveer soluciones que
satisfagan los requerimientos de los clientes de la Industria del
Software a nivel Nacional e Internacional.

\vfill\null \columnbreak

\hypertarget{md}{\section*{Matem\'{a}ticas discretas}}

Segundo semestre. Tres cr\'{e}ditos. Se utiliza el lenguaje Python en
lugar de una calculadora f\'{i}sica.

\begin{itemize}
\item{Representaci\'{o}n de enteros en varias bases}
\item{L\'{o}gica booleana}
\item{Conjuntos}
\item{Permutaciones}  
\item{Grafos simples no dirigidos; grados, caminos y distancias, \'{a}rboles de expansi\'{o}n}
\item{Aut\'{o}matas}
\item{M\'{a}quinas Turing}
\end{itemize}

\begin{description}
\item[10\%]{Ejercicios de representaci\'{o}n de enteros}
\item[10\%]{Ejercicios de l\'{o}gica booleana}
\item[10\%]{Ejercicios de conjuntos}
\item[10\%]{Ejercicios de permutaciones}
\item[15\%]{Examen de medio curso}
\item[10\%]{Ejercicios de grafos}
\item[10\%]{Ejercicios de aut\'{o}matas}
\item[10\%]{Ejercicios de M\'{a}quinas Turing}        
\item[15\%]{Examen ordinario}
\end{description}

\vfill\null \columnbreak

\hypertarget{sd}{\section*{Sistemas digitales}}

Tercer semestre. Tres cr\'{e}ditos. Incluye laboratorio. Requiere {\em
  Matem\'{a}ticas discretas}. El laboratorio consiste en ejercicios
con hardware y emuladores, mientras la clase cubre conceptos te\'{o}ricos.

\begin{itemize}
\item{Puertas, variables y tablas de verdad}
\item{Axiomas booleanas}
\item{Minimizaci\'{o}n algebr\'{a}ica}
\item{Retrasos y temporizaci\'{o}n}
\item{Lenguages de descripci\'{o}n}
\item{Mapas de Karnaugh}
\item{L\'{o}gica secuencial}
\item{M\'{a}quinas de estado finito}
\end{itemize}

{\em Solicitar al Ing.\ Ju\'{a}n \'{A}ngel la ponderaci\'{o}n y el programa
  anal\'{i}tico ajustado al temario actualizado}

\newpage

\hypertarget{lac}{\section*{Lenguaje C}}

Tres cr\'{e}ditos. Optativa de segundo semestre. Se utiliza el
compilador GCC con un editor de libre elecci\'{o}n.

\begin{itemize}
\item{Variables, tipos, operadores y expresiones}
\item{Condiciones}
\item{Ciclos}
\item{Entrada y salida en consola}
\item{Subrutinas y cabeceras}
\item{Punteros}
\item{Arreglos}
\item{Estructuras}
\item{Entrada y salida con archivos}  
\end{itemize}

\begin{description}
\item[5\%]{Programas con aritm\'{e}tica b\'{a}sica}
\item[5\%]{Programas con condiciones} 
\item[10\%]{Programas con ciclos}
\item[10\%]{Programas con I/O de consola}
\item[10\%]{Examen de medio curso}
\item[10\%]{Programas con subrutinas}
\item[10\%]{Programas con punteros}
\item[10\%]{Programas con arreglos}
\item[10\%]{Programas con estructuras}
\item[10\%]{Programas con I/O de archivos}            
\item[10\%]{Examen ordinario}
\end{description}

\vfill\null \columnbreak

\hypertarget{pep}{\section*{Programaci\'{o}n en Python}}

Cuatro cr\'{e}ditos. Optativa de segundo semestre. Se utiliza el
editor de IDLE.

\begin{itemize}
\item{Variables, tipos, operadores y expresiones}
\item{Condiciones}
\item{Ciclos}
\item{Entrada y salida en consola}  
\item{Subrutinas y librer\'{i}as propias}
\item{Listas y conjuntos}
\item{Diccionarios}  
\item{Entrada y salida con archivos}    
\item{Funciones an\'{o}nimas (lambda)}
\item{Mecanismos de comprensi\'{o}n}
\end{itemize}

\begin{description}
\item[5\%]{Programas con aritm\'{e}tica b\'{a}sica}
\item[5\%]{Programas con condiciones} 
\item[10\%]{Programas con ciclos}
\item[10\%]{Programas con I/O de consola}
\item[10\%]{Examen de medio curso}
\item[10\%]{Programas con subrutinas}
\item[10\%]{Programas con listas y conjuntos}
\item[10\%]{Programas con diccionarios}
\item[10\%]{Programas con I/O de archivos}              
\item[10\%]{Programas con funciones an\'{o}nimas}
\item[10\%]{Examen ordinario}
\end{description}

\vfill\null \columnbreak

\hypertarget{ayedd}{\section*{Algoritmos y estructuras de datos}} 

Tercer semestre. Cuatro cr\'{e}ditos. Requiere la optativa de segundo
semestre (C o Python); los ejemplos en clase se dan en Python.

\begin{itemize}
\item{Problemas de decisi\'{o}n}
\item{Complejidad computacional}
\item{Clases de complejidad P y NP}    
\item{Recursi\'{o}n; t\'{e}cnica dividir-conquistar}
\item{Listas, pilas y colas}
\item{Tablas de dispersi\'{o}n}
\item{\'{A}rboles binarios}
\item{Mont\'{i}culos}
\item{Grafos dirigidos; recorrido y b\'{u}squeda}
\item{Complejidad asint\'{o}tica}
\end{itemize}

\begin{description}
\item[5\%]{Problema de alcance (reachability)}
\item[5\%]{Problema de coloreo (2-coloring \& 3-coloring)}
\item[10\%]{Algoritmo de ordenamiento por fusi\'{o}n (mergesort)}
\item[10\%]{Examen de medio curso}
\item[10\%]{Ejercicios con funciones de dispersi\'{o}n}
\item[10\%]{Ejercicios te\'{o}ricos de construcci\'{o}n y de b\'{u}squeda en \'{a}rboles binarios}
\item[10\%]{Ejercicios te\'{o}ricos de construcci\'{o}n y de b\'{u}squeda en mont\'{i}culos Fibonacci}
\item[10\%]{Ordenamiento topol\'{o}gico}
\item[10\%]{B\'{u}squedas DFS y BFS}
\item[10\%]{Ejercicios te\'{o}ricos de la notaci\'{o}n de peor caso con pseudoc\'{o}digos}
\item[10\%]{Examen ordinario}
\end{description}

\vfill\null \columnbreak

\hypertarget{poao}{\section*{Programaci\'{o}n orientada a objetos}}

Tercer semeste. Cuatro cr\'{e}ditos. Incluye laboratorio. Requiere la
optativa de segundo semestre (C o Python). El laboratorio consiste en
la implementaci\'{o}n de los conceptos en Java, Python y C++, variando el
lenguaje para cubrir los tres lenguajes a nivel b\'{a}sico. En la clase se
discuten la teor\'{i}a y los conceptos igual como las diferencias
conceptuales y gram\'{a}ticas entre los lenguajes de este tipo en general,
sin limitarse solamente a los tres que se aplican en el
laboratorio. La clase y el laboratorio comparten en tema del proyecto
integrador: se dise\~{n}a para la clase y se implementa en el laboratorio.

\begin{itemize}
\item{Compilaci\'{o}n e interpretaci\'{o}n (pr\'{a}ctica en laboratorio)}
\item{Clases y objetos}
\item{Lenguage UML (generaci\'{o}n de diagramas en laboratorio)}
\item{Herencia, polimorfismo e interfaces}
\item{Patrones de dise\~{n}o}
\item{Persistencia y serializaci\'{o}n (pr\'{a}ctica en laboratorio)}
\item{Manejo de eventos, excepciones y errores (pr\'{a}ctica en laboratorio)}
\item{Pruebas unitarias (pr\'{a}ctica en laboratorio)}
\item{Interfaces gr\'{a}ficas (pr\'{a}ctica en laboratorio)}
\end{itemize}

\subsection*{Clase}

\begin{description}
\item[10\%]{Diagrama de clases en UML}
\item[10\%]{Diagrama de secuencia en UML}
\item[20\%]{Examen de medio curso}
\item[5\%]{Patrones de creaci\'{o}n}
\item[5\%]{Patrones de estructura}
\item[5\%]{Patrones de comportamiento}
\item[25\%]{Proyecto integrador: dise\~{n}o de una GUI$^\ast$}      
\item[20\%]{Examen ordinario}
\end{description}

\subsection*{Laboratorio}

\begin{description}
\item[5\%]{Compilaci\'{o}n de programas sencillos en Java}
\item[5\%]{Compilaci\'{o}n de programas sencillos en C++}
\item[5\%]{Interpretaci\'{o}n de programas sencillos en Python}  
\item[10\%]{Generaci\'{o}n autom\'{a}tica de diagramas de UML desde c\'{o}digo}
\item[10\%]{Generaci\'{o}n autom\'{a}tica de c\'{o}digo desde diagramas de UML}
\item[5\%]{Serializaci\'{o}n en Java}
\item[5\%]{Serializaci\'{o}n en C++}
\item[5\%]{Serializaci\'{o}n en Python}
\item[5\%]{Manejo de eventos en Java}
\item[5\%]{Manejo de eventos en C++}
\item[5\%]{Manejo de eventos en Python}
\item[15\%]{Pruebas unitarias en un lenguaje de libre selecci\'{o}n}
\item[20\%]{Proyecto integrador: construcci\'{o}n de GUI en un lenguaje de libre selecci\'{o}n$^\ast$}  
\end{description}

\newpage

\hypertarget{pye}{\section*{Probabilidad y estad\'{\i}stica }}

Cuarto semestre. Tres cr\'{e}ditos. Requiere \hyperlink{md}{\em
  Matem\'{a}ticas discretas}. Se utiliza el lenguaje R en lugar de una
calculadora f\'{i}sica.

\begin{itemize}
\item{Probabilidad b\'{a}sica}
\item{Distribuciones discretas}
\item{Probabilidad condicional}
\item{Distribuciones cont\'{i}nuas}
\item{Momentos}
\item{Funciones generadoras}
\item{Teoremas asint\'{o}ticos}
\item{Regresi\'{o}n lineal}
\item{Procesos Markovianos y de Poisson}    
\item{Modelos de urnas; teor\'{i}a de colas}  
\end{itemize}

\begin{description}
\item[5\%]{Ejercicios de conceptos b\'{a}sicos de probabilidad y conjuntos}
\item[5\%]{Distribuci\'{o}n binomial}
\item[5\%]{Distribuci\'{o}n geom\'{e}trica}  
\item[5\%]{Teorema de Bayes}
\item[5\%]{Distribuci\'{o}n de Poisson}
\item[10\%]{Examen de medio curso}
\item[10\%]{Valores esperados y varianzas de distribuciones}
\item[5\%]{Teorema central de l\'{i}mite}
\item[5\%]{Regresi\'{o}n lineal en R}
\item[5\%]{Distribuci\'{o}n estacionaria te\'{o}rico y experimental}
\item[20\%]{Proyecto integrador: un modelo de llegada de clientes y el
  an\'{a}lisis de tiempos de espera}
\item[20\%]{Examen ordinario}    
\end{description}

\vfill\null \columnbreak

\hypertarget{dde}{\section*{Dise\~{n}o de experimentos}}

Quinto semestre. Tres cr\'{e}ditos. Requiere {\em Probabilidad y estad\'{i}stica}.

\begin{itemize}
\item{Validez y formulaci\'{o}n de hip\'{o}tesis}
\item{Inferencia estad\'{i}stica}
\item{Pruebas estad\'{i}sticas para medias}
\item{Poder de pruebas estad\'{i}sticas}
\item{Pruebas de distribuci\'{o}n libre}
\item{An\'{a}lisis de varianza}
\item{Transformadas}
\item{Dise\~{n}os en bloques}
\item{Dise\~{n}os factoriales}
\item{Efecto y muestreo}
\end{itemize}

\vfill\null \columnbreak

\hypertarget{o}{\section*{Optimizaci\'{o}n}}  

Sexto semestre. Tres cr\'{e}ditos. Requiere {\em Matem\'{a}ticas IV}.

\begin{itemize}
\item{Problemas de optimizaci\'{o}n}
\item{Programaci\'{o}n lineal: variables, restricciones y objetivos}
\item{Algoritmo Simplex}
\item{Flujo en redes}
\item{Acoplamiento}
\item{Programaci\'{o}n din\'{a}mica}
\item{T\'{e}cnica ramificar-acotar}
\item{Frentes de Pareto}
\end{itemize}

\vfill\null \columnbreak

\hypertarget{mh}{\section*{Metaheur\'{\i}sticas}} 

Optativa de s\'{e}ptimo semestre. Tres cr\'{e}ditos. Requiere {\em Optimizaci\'{o}n}.

\begin{itemize}
\item{Ejemplos de problemas NP-duros}  
\item{Construcci\'{o}n de soluciones iniciales}
\item{B\'{u}squeda local}
\item{Escape de \'{o}ptimos locales}
\item{Criterios de desempe\~{n}o}  
\item{M\'{e}todos inspirados en la naturaleza}
\item{Ajuste automatizado de par\'{a}metros}
\item{Hyperheur\'{i}sticos}
\end{itemize}

\newpage

\hypertarget{at}{\section*{Arquitectura tecnol\'{o}gica}} 

Cuarto semestre. Tres cr\'{e}ditos. Requiere \hyperlink{sd}{\em Sistemas digitales}.

\begin{itemize}
\item{Instrucciones y procesadores}
\item{Yerarqu\'{i}a de memoria}
\item{Memoria virtual}
\item{Almacenaje en discos}
\item{Desempe\~{n}o de entrada y salida}
\item{Arquitectura de software}
\item{Requerimientos y atributos de calidad}
\item{Estilos arquitect\'{o}nicos; arquitectura limpia}
\item{Principios de dise\~{n}o}
\item{Software como servicio, microservicios}
\end{itemize}

\vfill\null \columnbreak

\hypertarget{so}{\section*{Sistemas operativos}}  

Quinto semestre. Cuatro cr\'{e}ditos. Requiere {\em Arquitectura tecnol\'{o}gica}.

\begin{itemize}
\item{Procesos e hilos}
\item{Exclusi\'{o}n mutua (deadlock, livelock \& starvation)}
\item{Sem\'{a}foros, candados, variables de condici\'{o}n y monitores}
\item{Calendarizaci\'{o}n de ejecuci\'{o}n}
\item{Manejo de memoria}
\item{Asignaci\'{o}n y reemplazo de p\'{a}ginas}
\item{Sistemas de archivos}
\item{Acceso a redes; sockets}
\item{Seguridad en sistemas operativos}
\end{itemize}

\vfill\null \columnbreak

\hypertarget{bdd}{\section*{Bases de datos}}    

Cuarto semestre. Tres cr\'{e}ditos. Requiere \hyperlink{poao}{\em
  Programaci\'{o}n orientada a objetos}.

\begin{itemize}
\item{Modelos relacionales}
\item{\'{A}lgebra relacional}
\item{Claves y dependencias}
\item{Esquemas}
\item{Modelo entidad-relaci\'{o}n}
\item{Principios de dise\~{n}o}
\item{Consultas}
\item{Vistas e \'{i}ndices}
\item{Lenguaje SQL}
\item{Bases de datos no relacionales}
\item{Representaci\'{o}n XML}
\item{Representaci\'{o}n JSON}  
\end{itemize}

\vfill\null \columnbreak

\hypertarget{vdi}{\section*{Visualizaci\'{o}n de informaci\'{o}n}}

Optativa de quinto semestre. Tres cr\'{e}ditos. Requiere
\hyperlink{bdd}{\em Bases de datos}.

\begin{itemize}
\item{An\'{a}lisis exploratorio de datos}
\item{Esquemas de colores y contrastes}
\item{Selecci\'{o}n de formas y grosores}
\item{Datos cuantitativos}
\item{Datos cualitativos}
\item{Series de tiempo}
\item{Datos georeferenciados}
\item{Visualizaci\'{o}n de texto}
\item{Visualizaci\'{o}n animada}  
\item{Visualizaci\'{o}n interactiva}
\end{itemize}

\newpage

\hypertarget{tydcc}{\section*{Transmisi\'{o}n y comunicaci\'{o}n de datos}}   

Cuarto semestre. Tres cr\'{e}ditos. Requiere \hyperlink{sd}{\em
  Sistemas digitales}.

\begin{itemize}
\item{Modelos de telecomunicaciones}  
\item{Protocolos de telecomunicaciones}
\item{TCP/IP}
\item{Medios de transmisi\'{o}n}
\item{Codificaci\'{o}n de se\~{n}ales}
\item{Protocolos de enlace y multiplex}
\item{Circuitos y paquetes}
\item{Transferencia as\'{i}ncrona}
\item{Retraso, p\'{e}rdida y desempe\~{n}o}
\end{itemize}

\vfill\null \columnbreak

\hypertarget{rc}{\section*{Redes computacionales}}  

Quinto semeste. Tres cr\'{e}ditos. Requiere {\em Transmisi\'{o}n y
  comunicaci\'{o}n de datos}.

\begin{itemize}
\item{Est\'{a}ndares y modelos}
\item{Redes al\'{a}mbricas}
\item{Redes inal\'{a}mbricas}
\item{Calidad de servicio}
\item{Ruteo}
\item{Medidas de desempe\~{n}o de ruteo}  
\item{Ahorro de energ\'{i}a}
\item{Redes de telefon\'{i}a}
\item{Redes satelitales}
\item{Redes ad hoc}
\item{Redes sensoras}    
\item{Simuladores de redes}
\end{itemize}

\vfill\null \columnbreak

\hypertarget{mdd}{\section*{Metodolog\'{\i}as de desarrollo}}

Cuarto semestre. Tres cr\'{e}ditos. Requiere \hyperlink{poao}{\em
  Programaci\'{o}n orientada a objetos}.

\begin{itemize}
\item{Ciclos de vida}
\item{Evaluaci\'{o}n}
\item{Herramientas de planeaci\'{o}n y desarrollo en equipo}
\item{Uso de patrones y antipatrones}
\item{Modelado de madurez de capacidades}
\item{Modelos de contenedores}
\item{Modelos cont\'{i}nuos}
\item{Metodolog\'{i}as \'{a}giles}
\item{Metodolog\'{i}as basados en pruebas}  
\item{Ejemplos de metodolog\'{i}as espec\'{i}ficas}
\end{itemize}

\vfill\null \columnbreak

\hypertarget{dyo}{\section*{Desarrollo y operaciones}} 

Optativa de quinto semestre. Tres cr\'{e}ditos. Requiere
\hyperlink{tycdd}{\em Transmisi\'{o}n y comunicaci\'{o}n de datos}.

\begin{itemize}
\item{Definiciones b\'{a}sicas}
\item{Objetivos principales}
\item{Integraci\'{o}n con metodolog\'{i}as diversas}
\item{Integraci\'{o}n y despliegue}
\item{Contenedores y virtualizaci\'{o}n}
\item{Administraci\'{o}n de configuraciones}
\item{Pruebas}
\item{Herramientas}
\end{itemize}

\newpage

\hypertarget{sa}{\section*{Sistemas adaptativos}} 

Quinto semestre. Tres cr\'{e}ditos. Requiere \hyperlink{pye}{\em
  Probabilidad y estad\'{i}stica}.

\begin{itemize}
\item{Aut\'{o}matas celulares}
\item{Sistemas multiagente}
\item{Algoritmos gen\'{e}ticos}
\item{Perceptrones}
\item{L\'{o}gica difusa}
\item{Teor\'{i}a de juegos}
\item{Teor\'{i}a de colaboraci\'{o}n}
\end{itemize}

\vfill\null \columnbreak

\hypertarget{mn}{\section*{M\'{e}todos num\'{e}ricos}}

Optativa de quinto semestre. Tres cr\'{e}ditos. Requiere \hyperlink{pye}{\em
  Probabilidad y estad\'{i}stica}.

\begin{itemize}
\item{Aritm\'{e}tica de punto flotante}
\item{Errores de redondeo}
\item{Convergencia}
\item{Ecuaciones no lineales}
\item{Interpolaci\'{o}n}
\item{Integraci\'{o}n num\'{e}rica}
\item{Diferenciaci\'{o}n num\'{e}rica}
\item{Valores y vectores propios}
\item{Aproximaci\'{o}n de m\'{i}nimos quadrados}
\end{itemize}

\vfill\null \columnbreak

\hypertarget{pf}{\section*{Programaci\'{o}n funcional}}

Optativa de quinto semestre. Tres cr\'{e}ditos. Require
\hyperlink{poao}{\em Programaci\'{o}n orientada a objetos}.

\begin{itemize}
\item{Evaluaci\'{o}n de expresiones}
\item{Ejemplos de lenguages funcionales}
\item{Estret\'{e}gias de reducci\'{o}n}
\item{Tipos fuertes}
\item{Operaciones con listas}
\item{Recursi\'{o}n e inducci\'{o}n}
\item{Estructuras ramificadas}
\item{An\'{a}lisis de eficiencia}
\item{An\'{a}lisis gram\'{a}tico (parsing) de expresiones aritm\'{e}ticas}
\end{itemize}

\newpage

\hypertarget{ihc}{\section*{Interacci\'{o}n humano-computadora}}   

Quinto semestre. Tres cr\'{e}ditos. Requiere \hyperlink{mdd}{\em
  Metodolog\'{i}as de desarrollo}.

\begin{itemize}
\item{Percepci\'{o}n y procesamiento; formas y colores}
\item{Percepci\'{o}n visual y expectaci\'{o}n; principios Gestalt}
\item{Limitantes cognitivos; atenci\'{o}n, memoria, contexto}
\item{Modelos mentales}
\item{Niveles de expertise}
\item{Pasos habilitadores, met\'{a}foras y affordancia}
\item{Equipo y t\'{e}cnicas de evaluaci\'{o}n en laboratorio}
\item{Dise\~{n}o conceptual y prototipeo}
\item{Evaluaci\'{o}n heur\'{i}stica}
\item{Caminata cognitiva}
\item{Protocolo de pensar en voz alta}
\item{Medidas de desempe\~{n}o en IHC}
\item{C\'{o}mputo ubicuo}
\item{Realidad aumentada}
\item{Realidad virtual}
\end{itemize}

\vfill\null \columnbreak

\hypertarget{cdd}{\section*{Ciencia de datos}}   

Optativa de sexto semestre. Tres cr\'{e}ditos. Requiere
\hyperlink{dde}{\em Dise\~{n}o de experimentos}.

\begin{itemize}
\item{Preparaci\'{o}n de datos}
\item{Lectura y manipulaci\'{o}n de datos}
\item{Estad\'{i}stica descriptiva}
\item{Visualizaci\'{o}n estad\'{i}stica}
\item{Pruebas estad\'{i}sticas}
\item{Regresi\'{o}n m\'{u}ltiple}
\item{An\'{a}lisis de componentes principales}
\item{M\'{a}quinas de vectores de soporte}  
\item{Series de tiempo y pron\'{o}sticos}
\end{itemize}

\vfill\null \columnbreak

\hypertarget{mys}{\section*{Modelado y simulaci\'{o}n}}

Optativa de s\'{e}ptimo semestre. Tres cr\'{e}ditos. Require
\hyperlink{cdd}{\em Ciencia de datos}.

\begin{itemize}
\item{Lenguage R}
\item{Movimiento Browniano}
\item{Interacci\'{o}n entre part\'{i}culas (din\'{a}mica molecular)}  
\item{Diagramas de Voronoi y triangulaci\'{o}n Delaunay}  
\item{Modelos epidemiol\'{o}gicos}
\item{M\'{e}todo Monte-Carlo}
\item{Procesos de nacimiento y muerte}
\item{Sistemas ca\'{o}ticos}
\item{Fractales}
\item{Medici\'{o}n de precisi\'{o}n y desempe\~{n}o}
\end{itemize}

\vfill\null \columnbreak

\hypertarget{ci}{\section*{C\'{o}mputo integrado}}  

Sexto semestre. Tres cr\'{e}ditos. Requiere \hyperlink{so}{\em
  Sistemas operativos}.

\begin{itemize}
\item{Lenguaje ensamblador}
\item{Microcontroladores}
\item{Simuladores y emuladores}
\item{Manejo de memoria}
\item{Perif\'{e}ricos}
\item{Interrupciones}
\item{Sistemas operativos espec\'{i}ficos}
\item{T\'{e}cnicas de optimizaci\'{o}n}
\item{Usabilidad, privacidad y seguridad}
\end{itemize}

\newpage

\hypertarget{cr}{\section*{Criptograf\'{\i}a}}

Optativa de sexto semestre. Tres cr\'{e}ditos. Requiere
\hyperlink{rc}{\em Redes computacionales}.

\begin{itemize}
\item{Aritm\'{e}tica modular}
\item{Cifras}
\item{Protocolos}
\item{Funciones unidireccionales}
\item{Algoritmo RSA}
\item{Firmas digitales}
\item{Cifras de bloque}
\item{Cifras de flujo}
\item{Cifras homomorfas}
\item{Dispersi\'{o}n resistente a colisiones}
\end{itemize}

\vfill\null \columnbreak

\hypertarget{si}{\section*{Seguridad inform\'{a}tica}} 

Optativa de sexto semestre. Tres cr\'{e}ditos. Requiere
\hyperlink{rc}{\em Redes computacionales}.

\begin{itemize}
\item{Ingenier\'{i}a social}
\item{Manejo de riesgos}
\item{Comunicaci\'{o}n y conciencia}
\item{Gobernabilidad y pol\'{i}ticas}
\item{Toma de decisiones}
\item{Seguridad y usabilidad}
\item{Cultura de seguridad}
\item{Cumplimiento}
\item{Pruebas de penetraci\'{o}n}
\item{Privacidad}
\end{itemize}

\vfill\null \columnbreak

\hypertarget{aw}{\section*{Aplicaciones web}}  

Sexto semestre. Tres cr\'{e}ditos. Requiere
\hyperlink{rc}{\em Redes computacionales}.

\begin{itemize}
\item{Sistemas de cliente-servidor}
\item{HTML}
\item{CSS}
\item{Servidores web}
\item{Hosting}
\item{Contenido din\'{a}mico (CGI)}  
\item{Frontend y backend}
\item{Usabilidad web}
\item{Web m\'{o}vil}
\item{Internet de las cosas}  
\end{itemize}

\vfill\null \columnbreak

\hypertarget{iddm}{\section*{Ingenier\'{\i}a de dispositivos m\'{o}viles}} 

Sexto semestre. Cuatro cr\'{e}ditos. Incluye laboratorio. Requiere
\hyperlink{rc}{\em Redes computacionales}.

\begin{itemize}
\item{Hardware m\'{o}vil}
\item{Propiedades de pantallas}  
\item{Acceso a micr\'{o}fonos y bocinas (pr\'{a}ctica en laboratorio)}  
\item{Acceso a sensores y c\'{a}maras (pr\'{a}ctica en laboratorio)}      
\item{Sistemas operativos m\'{o}viles}
\item{Consumo de energia (pr\'{a}ctica en laboratorio)}  
\item{Usabilidad}
\item{Accessibilidad}
\item{Seguridad (pr\'{a}ctica en laboratorio)}  
\item{Privacidad (pr\'{a}ctica en laboratorio)}  
\item{Tecnolog\'{i}as emergentes}
\end{itemize}

\newpage

\hypertarget{pdne}{\section*{Planeaci\'{o}n de negocios emergentes}}

Optativa de sexto semestre. Tres cr\'{e}ditos. Requiere
\hyperlink{ihc}{\em Interacci\'{o}n humano-computadora}.

\begin{itemize}
\item{Estudios de mercado}
\item{An\'{a}lisis competitiva}
\item{Proposici\'{o}n de valor}
\item{Elevator pitch}
\item{Producto m\'{i}nimo viable}
\item{Identificaci\'{o}n de clientes}
\item{Estrat\'{e}gia de negocios}
\item{Medidas de desempe\~{n}o}
\item{Requisitos legales}
\end{itemize}

\vfill\null \columnbreak

\hypertarget{da}{\section*{Desarrollo \'{a}gil}} 

Optativa de sexto semestre. Tres cr\'{e}ditos. Requiere
\hyperlink{mdd}{\em Metodolog\'{i}as de desarrollo}.

\begin{itemize}
\item{Programaci\'{o}n por pares y grupos}
\item{Programaci\'{o}n extrema}
\item{C\'{o}digo limpio}
\item{Refactorizaci\'{o}n}
  \item{Desarrollo basado en pruebas}
\item{Desarrollo basado en comportamiento (behavior)}
\item{Entrega continua}
\item{Scrum}
\item{Administraci\'{o}n de proyectos \'{a}giles}
\end{itemize}

\vfill\null \columnbreak

\hypertarget{idc}{\section*{Ingenier\'{\i}a de calidad}} 

Tres cr\'{e}ditos. Optativa de s\'{e}ptimo semestre. Requiere
\hyperlink{aw}{\em Aplicaciones web}.

\begin{itemize}
\item{Aseguramiento de calidad y pruebas}
\item{Aspectos organizacionales}
\item{Integraci\'{o}n con metodolog\'{i}as de desarrollo}
\item{Monitoreo y control de procesos}
\item{T\'{e}cnicas y herramientas de automatizaci\'{o}n}
\item{Medidas de desempe\~{n}o}
\item{Est\'{a}ndares de calidad de software}
\end{itemize}

\newpage

\hypertarget{ia}{\section*{Inteligencia artificial}}

S\'{e}ptimo semestre. Tres cr\'{e}ditos.

\begin{itemize}
\item{Agentes y entornos}
\item{Estrat\'{e}gias de b\'{u}squeda}
\item{Satisfacci\'{o}n de restricciones}
\item{B\'{u}squeda adversarial}
\item{Razonamiento determinista}
\item{L\'{o}gica proposicional y de primer \'{o}rden; inferencia (prolog)}
\item{Representaci\'{o}n de conocimiento}
\item{Planeaci\'{o}n y actuaci\'{o}n}
\item{Incertidumbre}
\item{Razonamiento probabilista}
\end{itemize}

\vfill\null \columnbreak

\hypertarget{am}{\section*{Aprendizaje m\'{a}quina}} 

Optativa de octavo semestre. Tres cr\'{e}ditos. Requiere {\em Inteligencia artificial}.

\begin{itemize}
\item{Aprendizaje supervisado}
\item{Clasificaci\'{o}n}
\item{Agrupamiento}  
\item{Error, sesgo y varianza}
\item{Criterios de desempe\~{n}o}
\item{Redes neuronales}
\item{Aprendizaje no supervisado}
\item{Aprendizaje reforzado}  
\item{An\'{a}lisis de texto}
\item{Aprendizaje profundo}
\end{itemize}

\vfill\null \columnbreak

\hypertarget{vc}{\section*{Visi\'{o}n computacional}} 

Tres cr\'{e}ditos. Octavo semestre. Requiere \hyperlink{cp}{\em
  C\'{o}mputo paralelo}.

\begin{itemize}
\item{Representaci\'{o}n digital de im\'{a}genes}
\item{Canales y m\'{a}scaras}
\item{Detecci\'{o}n de bordes}
\item{Detecci\'{o}n de entidades}
\item{Detecci\'{o}n de l\'{i}neas rectas}
\item{Detecci\'{o}n de c\'{i}rculos y elipses}
\item{Detecci\'{o}n de movimiento bidimensional}
\item{Detecci\'{o}n de movimiento tridimensional}
\end{itemize}

\vfill\null \columnbreak

\hypertarget{gc}{\section*{Gr\'{a}ficas computacionales}}

Tres cr\'{e}ditos. Optativa de octavo semestre.  Requiere
\hyperlink{cp}{\em C\'{o}mputo paralelo}.

\begin{itemize}
\item{Conceptos y herramientas (OpenGL)}  
\item{Transformadas bidimensionales}
\item{Transformadas tridimensionales}  
\item{Texturas}
\item{Identificaci\'{o}n de superficies visibles}  
\item{Iluminaci\'{o}n}
\item{Manipulaci\'{o}n y almacenaje de im\'{a}genes}
\end{itemize}

\newpage

\hypertarget{ayc}{\section*{Automatizaci\'{o}n y control}} 

Optativa de sexto semestre. Tres cr\'{e}ditos. Requiere {\em M\'{e}todos num\'{e}ricos}.

\begin{itemize}
\item{Sistemas de control}
\item{Diagramas de flujo de se\~{n}ales}
\item{Linealizaci\'{o}n}
\item{An\'{a}lisis en el dominio del tiempo}
\item{An\'{a}lisis en el dominio de la frecuencia}
\item{An\'{a}lisis en el espacio de estados}
\item{Propiedades estructurales}
\end{itemize}

\vfill\null \columnbreak


\hypertarget{cya}{\section*{Controladores y actuadores}} 

Cuatro cr\'{e}ditos. Optativa de s\'{e}ptimo semestre. Require
\hyperlink{ayc}{\em Automatizaci\'{o}n y control}.

\begin{itemize}
\item{Sistemas mecatr\'{o}nicos}
\item{Tipos de sensores}
\item{Tipos de actuadores}
\item{Emuladores y simuladores}  
\item{Modelado mecatr\'{o}nico}
\item{Sistemas de ciclo cerrado}
\item{Posicionamiento}
\item{Coordinaci\'{o}n multiagente}
\end{itemize}

\vfill\null \columnbreak

\hypertarget{idsa}{\section*{Ingenier\'{\i}a de sistemas
    aut\'{o}nomos}} Tres cr\'{e}ditos. Optativa de octavo
semestre. Requiere {\em Controladores y actuadores}.

\begin{itemize}
\item{Observaci\'{o}n de etorno}
\item{Identificaci\'{o}n de objetos}
\item{Ubicaci\'{o}n}
\item{Navegaci\'{o}n}
\item{Agarre y manipulaci\'{o}n}
\item{Coordinaci\'{o}n de parvadas (swarm)}
\item{Robots aut\'{o}nomos}
\item{Veh\'{i}culos aut\'{o}nomos}
\end{itemize}

\newpage

\hypertarget{sdi}{\section*{Sistemas distribuidos}} 

Tres cr\'{e}ditos. S\'{e}ptimo semestre. Requiere \hyperlink{ci}{\em
  C\'{o}mputo integrado}.

\begin{itemize}
\item{Algoritmos distribuidos}
\item{Manejo de memoria}
\item{Sistemas de archivos}
\item{Consistencia y replicaci\'{o}n}
\item{Tolerancia a fallas}
\item{Superc\'{o}mputo}
\item{Algoritmos auto-estabilizadores}
\end{itemize}

\vfill\null \columnbreak

\hypertarget{cp}{\section*{C\'{o}mputo paralelo}} 

Tres cr\'{e}ditos. S\'{e}ptimo semestre. Requiere \hyperlink{ci}{\em
  C\'{o}mputo integrado}.

\begin{itemize}
\item{Algoritmos paralelos}
\item{Procesos y concurrencia}
\item{Memoria compartida y coherencia de cache}
\item{Sistemas de memoria distribuida}
\item{Intercambio de mensajes (MPI)}
\item{Direccionamiento global}
\item{Medici\'{o}n de desempe\~{n}o}
\item{Sincronizaci\'{o}n}
\item{Programaci\'{o}n para GPU}
\end{itemize}

\vfill\null \columnbreak

\hypertarget{ls}{\section*{Lenguajes script}}

Tres cr\'{e}ditos. Optativa de s\'{e}ptimo semestre.
 Requiere \hyperlink{ci}{\em
  C\'{o}mputo integrado}.

\begin{itemize}
\item{Lenguajes de la familia shell}
\item{Argumentos en l\'{i}nea de instrucciones}
\item{C\'{o}digos de error}
\item{Redirecci\'{o}n y mecanismos pipeline}
\item{Lenguaje \texttt{(g)awk}}
\item{Lenguaje \texttt{sed}}
\item{Herramienta \texttt{sort}}  
\item{Herramienta \texttt{tr}}
\item{Herramienta \texttt{grep}}
\item{Herramienta \texttt{screen}}
\item{Herramienta \texttt{crontab}}
\item{Herramientas \texttt{curl} y \texttt{wget}}  
\item{Otras herramientas de l\'{i}nea de instrucciones}
\end{itemize}

\newpage

\hypertarget{tdi}{\section*{Teor\'{\i}a de informaci\'{o}n}}

Tres cr\'{e}ditos. Optativa de s\'{e}ptimo semestre. Requiere
\hyperlink{cr}{\em Criptograf\'{i}a}.

\begin{itemize}
\item{Informaci\'{o}n y entropia}
\item{Formatos de representaci\'{o}n digital}
\item{Codificaci\'{o}n}
\item{C\'{o}digos de bloque}
\item{Detecci\'{o}n de errores}
\item{Recuperaci\'{o}n de errores}
\item{Compresi\'{o}n sin perdida}
\item{Compresi\'{o}n con perdida}
\end{itemize}

\vfill\null \columnbreak

\hypertarget{tdc}{\section*{Teor\'{\i}a de la computaci\'{o}n}}

Tres cr\'{e}ditos. Optativa de octavo semestre. Requiere
{\em Teor\'{i}a de informaci\'{o}n}.

\begin{itemize}
\item{Modelos de c\'{o}mputo}
\item{Lengujes regulares}
\item{Aut\'{o}matas finitos}
\item{Expresiones regulares}
\item{Lenguajes libre de contexto}
\item{Problemas decidibles; problema de detenci\'{o}n}
\item{Reducibilidad}
\item{Clase PSPACE}
\item{Problemas insolubles}
\end{itemize}

\vfill\null \columnbreak

\hypertarget{vf}{\section*{Verificaci\'{o}n formal}} 

Tres cr\'{e}ditos. Optativa de octavo semestre. Requiere
\hyperlink{sdi}{\em Sistemas distribuidos}.

\begin{itemize}
\item{Formas normales de l\'{o}gica proposicional}
\item{Diagramas binarios de decisi\'{o}n}
\item{L\'{o}gica predicativa de primer y segundo orden}
\item{Demostraciones de validez}
\item{Modelado de sistemas concurrentes}
\item{L\'{o}gica temporal lineal}
\item{Modelos l\'{o}gicos de sistemas}
\item{Verificaci\'{o}n de modelos}
\item{Redes Petri}
\end{itemize}


\newpage

\hypertarget{cdc}{\section*{Confiabilidad de c\'{e}ntros de c\'{o}mputo}} 

Tres cr\'{e}ditos. Optativa de octavo semestre. Requiere
\hyperlink{sdi}{\em Sistemas distribuidos}.

\begin{itemize}
\item{Medici\'{o}n de confiabilidad (SRE)}
\item{Expectativas de clientes}
\item{Mecanismos de operaci\'{o}n confiables}
\item{Objetivos de nivel de servicio (SLO)}
\item{Indicadores de nivel de servicio (SLI)}
\item{Acuerdos de nivel de servicio (SLA)}
\item{Monitoreo automatizado}
\item{Selecci\'{o}n de m\'{e}tricas}
\item{Quantificaci\'{o}n de riesgos}
\item{Consecuencias de fallas}
\end{itemize}

\vfill\null \columnbreak

\hypertarget{ayp}{\section*{Almacenaje y procesamiento de datos grandes}}

Tres cr\'{e}ditos.
Optativa de octavo semestre. Requiere
\hyperlink{ls}{\em Lenguajes script}.

\begin{itemize}
\item{Conceptos, paradigmas y plaatformas}
\item{Herramientas de programaci\'{o}n}
\item{Extracci\'{o}n e integraci\'{o}n}
\item{Almacenaje}
\item{Escalabilidad de \'{i}ndices}
\item{Procesamiento de grafos}
\item{Procesamiento de flujos (streams)}
\item{An\'{a}lisis probabilista}
\item{Visualizaci\'{o}n}
\item{Privacidad y anonimidad}
\end{itemize}

\vfill\null \columnbreak

\hypertarget{celn}{\section*{C\'{o}mputo en la nube}} 

Tres cr\'{e}ditos. Optativa de octavo semestre. Requiere
\hyperlink{sdi}{\em Sistemas distribuidos}.

\begin{itemize}
\item{Principios de arquitectura en la nube}
\item{Plataformas de c\'{o}mputo en la nube}
\item{Paralelismo en la nube}
\item{Almacenaje distribuido}
\item{Virtualizaci\'{o}n}
\item{Seguridad}
\item{Sistemas operativos de n\'{u}cleos m\'{u}ltiples}
\item{T\'{e}cnicas map-reduce}
\item{Proveedores de servicio actuales}
\end{itemize}

\end{multicols}
\newpage

\hypertarget{te}{\section*{Tecnolog\'{\i}as emergentes}} 

Tres cr\'{e}ditos. Octavo semestre. Requiere \hyperlink{ia}{\em
  Inteligencia artificial}.

Discusi\'{o}n de t\'{o}picos selectos de inter\'{e}s actual como por ejemplo
procesamiento de lenguage natural, bioinform\'{a}tica, criptomonedas o
blockchain.

\hypertarget{pi1}{\section*{Proyecto integrador I}} 

Cuatro cr\'{e}ditos. Octavo semestre. Requiere \hyperlink{aw}{\em
  Aplicaciones web}.

Se produce un plan de trabajo que detalla un proyecto del \'{a}rea de
tecnolog\'{i}a de software junto con un prototipo inicial en un
repositorio p\'{u}blico. Se debe especificar la metodolog\'{i}a a seguir y
mantener una bit\'{a}cora semanal.

El tama\~{n}o de los grupos de trabajo es de dos a siete personas, a la
par con la complejidad del proyecto propuesto. Cada equipo debe
nombrar un gerente de proyecto y aclarar el papel de cada integrante
al inicio.

\hypertarget{pi2}{\section*{Proyecto integrador II}} 

Cuatro cr\'{e}ditos. Noveno semestre. Requiere {\em Proyecto integrador I}.

Se crea la versi\'{o}n final del producto de software junto con la
documentaci\'{o}n pertinente en el mismo repositorio p\'{u}blico creado en la
primera unidad. Se debe mantener una bit\'{a}cora semanal.

\hypertarget{cert}{\section*{Certificaci\'{o}n I \& II}} 

Cualquier combinaci\'{o}n de certificaciones externas con reconocimiento
internacional que cubra una cantidad de horas por lo menos igual a la
cantidad de cr\'{e}ditos otorgados, como por ejemplo certificaciones de
AWS, MS Azure o Google.



\end{document}




