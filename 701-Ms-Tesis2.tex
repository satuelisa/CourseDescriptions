\documentclass[10 pt]{article}
\usepackage{wallpaper}
\usepackage[spanish, mexico]{babel}  
\usepackage{color,colortbl}
\usepackage[T1]{fontenc}
\usepackage{fancyhdr} % before geometry
\usepackage[letterpaper,left=18mm,right=18mm,headheight=5mm,headsep=40mm,top=52mm,bottom=22mm]{geometry}
\usepackage{graphicx}
\usepackage{rotating}
\usepackage[latin1]{inputenc}
\usepackage{hyperref}
\usepackage{graphics}
\usepackage{varwidth}
\usepackage{tikz}

\usepackage{tcolorbox}
\definecolor{headerframe}{RGB}{177,178,177} % #b1b2b1
\definecolor{headercontent}{RGB}{239,239,239} % #efefef
\definecolor{tableheader}{RGB}{185,208,238}% #b9d0ee
\definecolor{evidence}{RGB}{238,208,185}
\definecolor{perfil}{RGB}{208,238,185}
\definecolor{unidad}{RGB}{208,185,185} 

\usetikzlibrary{shapes,arrows}
\tikzstyle{elem} = [scale=0.7, draw, rectangle, thick, minimum height=2em,
minimum width=2em, execute at begin node={\begin{varwidth}{12em}},
  execute at end node={\end{varwidth}}]
\tikzstyle{perf} = [draw, rectangle, thick, minimum height=2em,
minimum width=2em, fill=perfil]
\tikzstyle{comp} = [draw, rectangle, thick, minimum height=2em,
minimum width=2em, fill=tableheader]
\tikzstyle{esp} = [scale=0.7, draw, rectangle, thick, minimum height=2em,
minimum width=2em, fill=tableheader]
\tikzstyle{unidad} = [scale=0.9, draw, rectangle, thick, minimum height=2em,
minimum width=2em, fill=evidence]
\tikzstyle{evid} = [scale=0.75, draw, rectangle, thick, minimum height=2em,
minimum width=2em, fill=unidad]
\tikzstyle{header} = [scale=0.8, minimum height=2em, minimum width=2em, execute at begin node={\begin{varwidth}{4em}},
  execute at end node={\end{varwidth}}]


\tikzstyle{line} = [draw, thick, -stealth, shorten >=0pt]
\usepackage{multicol}
\usepackage{wrapfig}
\usepackage{enumitem}
\renewcommand*\familydefault{\sfdefault} 
\renewcommand{\thesection}{\Roman{section}} 
\usepackage{remreset}

\makeatletter
\@removefromreset{subsection}{section}
\renewcommand\thesubsection{\arabic{subsection}}
\makeatother

\makeatletter
\@removefromreset{subsubsection}{section} % no number
\renewcommand\thesubsubsection{}
\makeatother

\usepackage{titlesec}
\titleformat{\subsubsection}[runin]
{\bfseries}{\thesubsubsection}{0em}{} % in boldface

\makeatletter
\@removefromreset{theparagraph}{paragraph} % no number
\renewcommand\theparagraph{}
\makeatother

\usepackage{titlesec}
\titleformat{\paragraph}[runin]
{\itshape}{\theparagraph}{1em}{} % in italics

\ULCornerWallPaper{1}{logos.pdf}
\setlength{\parindent}{1em}
\setlength{\parskip}{2pt}
\tcbset{
  arc = 3mm,
  colframe = headerframe,
  colback = headercontent,
  fonttitle=\sffamily
}

\usepackage{amssymb}
\newcommand{\yes}{\makebox[0pt][l]{$\square$}{\raisebox{0.1\height}{$\times$}}}
\newcommand{\no}{\makebox[0pt][l]{$\square$}{\raisebox{0.1\height}{\phantom{$\times$}}}}
\usepackage{enumitem}
\titleformat{\section}{\normalfont\large\bfseries}{\thesection.}{4pt}{}
\titleformat{\subsection}[runin]{\normalfont\bfseries}{\thesubsection.}{4pt}{}


\newcommand{\UANL}{UNIVERSIDAD AUT\'{O}NOMA DE NUEVO LE\'{O}N}
\newcommand{\uanl}{Universidad Aut\'{o}noma de Nuevo Le\'{o}n}
\newcommand{\fime}{Facultad de Ingenier\'{\i}a Mec\'{a}nica y El\'{e}ctrica}
\newcommand{\maestria}{Maestr\'{\i}a en Ciencias de la Ingenier\'{\i}a con Orientaci\'{o}n en Sistemas}
\newcommand{\doctorado}{Doctorado en Ingenier\'{\i}a de Sistemas}
\newcommand{\PA}{PROGRAMA ANAL\'{I}TICO}

% LGAC
\newcommand{\seys}{Sistemas estoc\'{a}sticos y simulaci\'{o}n}
\newcommand{\mado}{M\'{e}todos avanzados de optimizaci\'{o}n}
\newcommand{\odsi}{Optimizaci\'{o}n de sistemas industriales}

% AREAS CURRICULARES
\newcommand{\fb}{formaci\'{o}n b\'{a}sica} % maestria
\newcommand{\fa}{formaci\'{o}n avanzada} % maestria
\newcommand{\fr}{formaci\'{o}n} % doctorado
\renewcommand{\div}{divulgaci\'{o}n}
\newcommand{\inv}{investigaci\'{o}n}
\renewcommand{\pi}{producto integrador}
\newcommand{\da}{de aplicaci\'{o}n}
\renewcommand{\le}{libre elecci\'{o}n}

% CLAVES DE LAS UNIDADES

\newcommand{\algm}{PM109} % analisis y diseno de algoritmos
\newcommand{\algd}{PD109}
\newcommand{\aprm}{PM134} % aprendizaje automatico
\newcommand{\aprd}{PD134}
\newcommand{\dsm}{PM123} % ciencia de datos
\newcommand{\dsd}{PD123}
\newcommand{\ccm}{PM135} % complejidad computacional
\newcommand{\ccd}{PD135} 
\newcommand{\iam}{PM101} % inteligencia artificial
\newcommand{\iad}{PD101}
\newcommand{\cvm}{PM124} % procesamiento de imagenes y vision computacional
\newcommand{\cvd}{PD124}
\renewcommand{\sim}{PM201} % seminarios
\newcommand{\siim}{PM202}
\newcommand{\sid}{PD201}
\newcommand{\siid}{PD202}
\newcommand{\siiid}{PD203}
\newcommand{\sivd}{PD204}
\newcommand{\svd}{PD205}
\newcommand{\svid}{PD206}
\newcommand{\sviid}{PD207}
\newcommand{\sviiid}{PD208}
\newcommand{\ssm}{PM105} % simulacion de sistemas
\newcommand{\ssd}{PD105}
\newcommand{\tim}{PM501} % tesis
\newcommand{\tiim}{PM502}
\newcommand{\tid}{PD501}
\newcommand{\tiid}{PD502}
\newcommand{\tiiid}{PD503}
\newcommand{\tivd}{PD504}
\newcommand{\tvd}{PD505}
\newcommand{\tvid}{PD506}
\newcommand{\tviid}{PD507}
\newcommand{\tviiid}{PD508} 

\newcommand{\narturo}{095012}
\newcommand{\nelisa}{096633}
\newcommand{\ncesar}{092038}
\newcommand{\nangy}{102662}
\newcommand{\nvincent}{102947}
\newcommand{\nada}{060581}
\newcommand{\niris}{103743}
\newcommand{\nsara}{100546}
\newcommand{\nroger}{090969}
\newcommand{\nigor}{093179}
\newcommand{\nromeo}{100959}
\newcommand{\nferny}{095808}
\newcommand{\arturo}{Dr.\ Jos\'{e} Arturo Berrones Santos}
\newcommand{\elisa}{Dra.\ Satu Elisa Schaeffer}
\newcommand{\cesar}{Dr.\ C\'{e}sar Emilio Villarreal Rodr\'{\i}guez}
\newcommand{\angy}{Dra.\ Mar\'{\i}a Ang\'{e}lica Salazar Aguilar}
\newcommand{\vincent}{Dr.\ Vincent Boyer}
\newcommand{\ada}{Dra.\ Ada Margarita \'{A}lvarez Socarr\'{a}s}
\newcommand{\iris}{Dra.\ Iris Abril Mart\'{\i}nez Salazar}
\newcommand{\sara}{Dra.\ Sara Ver\'{o}nica S\'{a}nchez Rodr\'{\i}guez}
\newcommand{\roger}{Dr.\ Roger Zirahu\'{e}n R\'{\i}os Mercado}
\newcommand{\igor}{Dr.\ Igor Litvinchev}
\newcommand{\romeo}{Dr.\ Romeo S\'{a}nchez Nigenda}
\newcommand{\ferny}{Dr.\ Fernando L\'{o}pez Irarragorri}
\newcommand{\simon}{Dr.\ Sim\'{o}n Mart\'{\i}nez Mart\'{\i}nez} 

\newcommand{\subdirector}{\simon}
\newcommand{\coordinador}{\cesar}


\begin{document}

\pagestyle{fancy}
\renewcommand{\headrulewidth}{0pt}
\fancyhf{}
\fancyhead[L]{}
\fancyhead[C]{}
\fancyhead[R]{IT-8-SPG-02-R03}
\fancyfoot[L]{Revisi\'{o}n: 1 \\
  Vigente a partir del: 01 de agosto del 2016}
\fancyfoot[C]{}
\fancyfoot[R]{P\'{a}gina~\thepage~de~\pageref*{final}}


\fancypagestyle{plain}{%
\fancyhf{}
\fancyhead[L]{}
\fancyhead[C]{}
\fancyhead[R]{IT-8-SPG-02-R03}
\fancyfoot[L]{Revisi\'{o}n: 1 \\
  Vigente a partir del: 01 de agosto del 2016}
\fancyfoot[C]{}
\fancyfoot[R]{}}

\begin{tcolorbox}
  \begin{center}

    {\bf \UANL}

    \medskip

    {\bf \fime}

    \medskip
    
    {\bf PE} \underline{\bf \maestria}

    \medskip

    \underline{\PA}

  \end{center}
\end{tcolorbox}

\section{Datos de Identificaci\'{o}n de la Unidad de Aprendizaje:}
\subsection{Nombre:} \fbox{Tesis 2}
\subsection{Frecuencia semanal:} horas de trabajo presencial \fbox{4}
\subsection{Horas de trabajo extra aula por semana:} \fbox{5}
\subsection{Modalidad:} \yes~Escolarizada \no~No escolarizada \no~Mixto
\subsection{Per\'{\i}odo acad\'{e}mico:} \yes~Semestral
\no~Tetramestral \no~Modular
\subsection{LGAC:} \underline{Optimizaci\'{o}n de sistemas industriales}
\subsection{Ubicaci\'{o}n semestral:} \underline{3}
\subsection{\'{A}rea curricular:} \underline{Formaci\'{o}n, investigaci\'{o}n}
\subsection{Cr\'{e}ditos:} \underline{6}
\subsection{Requisito:} \underline{Tesis 1}
\subsection{Fecha de elaboraci\'{o}n:} \underline{20/01/2010}
\subsection{Fecha de la \'{u}ltima actualizaci\'{o}n:} \underline{09/06/2021}
\subsection{Responsable (es) del dise\~{n}o:}
\begin{itemize}[label={}]
\item \underline{\nroger~\roger}
\item \underline{\nelisa~\elisa}
\end{itemize}
\newpage
\section{Presentaci\'{o}n:}

Se apoya al participante en la finalizaci\'{o}n de su trabajo de tesis de
maestr\'{\i}a.

\section{Prop\'{o}sito(s):}

Se brinda apoyo y gu\'{\i}a sobre los aspectos fundamentales del desarrollo
de un trabajo de tesis.

\section{Competencias del perfil de egreso:}
\subsection{Competencias del perfil de egreso}

P1) Resolver problemas en el \'{a}rea de toma de decisiones en ambientes
operativos que pueden ser din\'{a}micos o inciertos para lograr una
asignaci\'{o}n m\'{a}s efectiva de recursos y decidir el curso de acci\'{o}n
\'{o}ptimo para lograr objetivos establecidos.



P2) Resolver problemas concretos en sistemas de la industria, la
academia o el sector p\'{u}blico en base a las herramientas de la toma de
decisiones con bases cient\'{\i}ficas para lograr el mejor dise\~{n}o,
an\'{a}lisis, planeaci\'{o}n o gesti\'{o}n de dichos sistemas.


P3) Establecer comunicaci\'{o}n con los disGntos sectores de la
sociedad a fin de establecer proyectos estrat\'{e}gicos en las
distintas disciplinas de la ingenier\'{\i}a de sistemas y crear la
cultura de la creaci\'{o}n de riqueza basada en el conocimiento.

  
\subsection{Competencias generales a que se vincula la Unidad de Aprendizaje:}
  
La unidad se vincula con las siguientes competencias generales:

\phantom{space}
\begin{tabular}{|p{12cm}|p{45mm}|}
  \hline
  \cellcolor{tableheader}{\em Declaraci\'{o}n de la competencia general vinculada a la unidad
  de aprendizaje}
  & \cellcolor{tableheader}{\em Evidencia} \\ \hline

C11) Practica los valores promovidos por la UANL: verdad, equidad,
honestidad, libertad, solidaridad, respeto a la vida y a los dem\'{a}s,
respeto a la naturaleza, integridad, \'{e}tica profesional, justicia y
responsabilidad, en su \'{a}mbito personal y profesional para contribuir a
construir una sociedad sostenible.
 & Avances en el documento de tesis, bit\'{a}cora semanal \\ \hline
C12) Construye propuestas innovadoras basadas en la comprensi\'{o}n
hol\'{\i}stica de la realidad incluyendo los diferentes campos cient\'{\i}ficos
para contribuir a superar los retos del ambiente global
interdependiente.
 & Avances en el documento de tesis, bit\'{a}cora semanal \\ \hline
C13) Asume el liderazgo que le ha otorgado el dominio de las ciencias,
comprometido con las necesidades sociales y profesionales para
promover el cambio social pertinente.
 & Avances en el documento de tesis, bit\'{a}cora semanal \\ \hline
\end{tabular}

\newpage
    
\subsection{Competencias espec\'{\i}ficas y nivel de dominio a que se vincula la unidad de aprendizaje:}

La unidad se vincula con las siguientes competencias espec\'{\i}ficas:

\phantom{space}
\begin{tabular}{|p{30mm}|p{4mm}|p{4mm}|p{4mm}|p{4mm}|p{30mm}|p{22mm}|p{24mm}|p{22mm}|}
\hline
\cellcolor{tableheader}{{\em Competencia Espec\'{\i}fica}}
& \rotatebox{90}{\cellcolor{tableheader}{Nivel I Inicial}}
& \rotatebox{90}{\cellcolor{tableheader}{Evidencia}}
& \rotatebox{90}{\cellcolor{tableheader}{Nivel II B\'{a}sico}}
& \rotatebox{90}{\cellcolor{tableheader}{Evidencia}}
& \rotatebox{90}{\cellcolor{tableheader}{Nivel III Aut\'{o}nomo}}
& \rotatebox{90}{\cellcolor{tableheader}{Evidencia}}
& \rotatebox{90}{\cellcolor{tableheader}{Nivel IV Estrat\'{e}gico\phantom{xxx}}}
& \rotatebox{90}{\cellcolor{tableheader}{Evidencia}}
\\ \hline


E3) Comunicar efectivamente los resultados obtenidos mediante la
ingenier\'{\i}a de sistemas, tanto con pares de las distintas disciplinas
acad\'{e}micas, as\'{\i} como con los diferentes sectores de la sociedad para
la generaci\'{o}n de bienestar y riqueza en base a la innovaci\'{o}n.

& & & & & 
Comunica efectivamente trabajo original de investigaci\'{o}n en foros o publicaciones cientificas y tecnol\'{o}gicas. & Redacci\'{o}n de la tesis 
& Establece junto con distintos sectores en la academia, la industria o
la sociedad en general, proyectos innovadores de car\'{a}cter estrat\'{e}gico.
 & Redacci\'{o}n de la tesis 
\\ \hline
\end{tabular}

\section{Representaci\'{o}n gr\'{a}fica:}

\begin{center}
\begin{tikzpicture}[scale=1, auto]
  \matrix[row sep=1cm, column sep=7mm]{

& 
\node[elem] (ide) {Diagnosticar el avance actual del trabajo de tesis};
\\
\node[elem] (lit) {Actualizar la revisi\'{o}n de literatura cient\'{\i}fica relacionada relevante};
&
\node[elem] (des) {Finalizar el desarrollo de la soluci\'{o}n propuesta};
&
\node[elem] (her) {Documentar las herramientas computacionales
  utilizadas para implementar el m\'{e}todo};
\\
\node[elem] (for) {Actualizar las hip\'{o}tesis y objetivos};
& &
\node[elem] (sel) {Comparar las caracter\'{\i}sticas del m\'{e}todo
  con el estado de arte};
\\
\node[elem] (dis) {Documentar el dise\~{n}o del m\'{e}todo};
&
\node[elem] (imp) {Finalizar la implementaci\'{o}n del m\'{e}todo};
\\
&
\node[elem] (ana) {Realizar experimentos para
  contrastar sus resultados con las hip\'{o}tesis};
\\ 
& & \node[elem] (doc) {Documentar los resultados conforme al estilo de
  redacci\'{o}n cient\'{\i}fica};
\\ 
\node[elem] (apl) {Aplicar visualizaci\'{o}n cient\'{\i}fica a los resultados};
& &
\node[elem] (exp) {Discutir entre pares los 
  resultados obtenidos};
\\
};
\draw [line] (ide) -- (lit);
\draw [line] (ide) -- (des);
\draw [line] (ide) -- (her);
\draw [line] (lit) -- (des);
\draw [line] (lit) -- (for);
\draw [line] (des) -- (imp);
\draw [line] (her) -- (sel);
\draw [line] (sel) -- (imp);
\draw [line] (for) -- (dis);
\draw [line] (imp) -- (ana);
\draw [line] (dis) -- (ana);
\draw [line] (dis) -- (apl);
\draw [line] (ana) -- (doc);
\draw [line] (apl) -- (doc);
\draw [line] (doc) -- (exp);
\end{tikzpicture}
\end{center}


\newpage
\section{Estructuraci\'{o}n en cap\'{\i}tulos, etapas o fases de la unidad de
  aprendizaje:}
\subsection{Desarrollo de las fases de la Unidad de Aprendizaje:}

Orientaci\'{o}n al alumno para proseguir el tema de tesis, donde deber\'{a}n
seguirse el problema a tratar, objetivos perseguidos, justificaci\'{o}n,
planteamiento de hip\'{o}tesis y metodolog\'{\i}a a aplicar.  Presentaci\'{o}n
peri\'{o}dica del avance del trabajo para su revisi\'{o}n, orientaci\'{o}n y
validaci\'{o}n del grado de avance. La sesiones son de cuatro horas cada
una y son veinte semanas en total.

\begin{enumerate}[itemsep=-2pt]
\item Actualizaci\'{o}n de la calendarizaci\'{o}n de actividades
\item Finalizaci\'{o}n de la redacci\'{o}n de la introducci\'{o}n
\item Finalizaci\'{o}n de la redacci\'{o}n de la hip\'{o}tesis y los objetivos
\item Finalizaci\'{o}n de la redacci\'{o}n de los antecedentes
\item Finalizaci\'{o}n de la redacci\'{o}n de la clasificaci\'{o}n de trabajos relacionados
\item Finalizaci\'{o}n de la redacci\'{o}n del an\'{a}lisis comparativo de trabajos relacionados
\item Finalizaci\'{o}n de la redacci\'{o}n de la identificaci\'{o}n del \'{a}rea de oportunidad
\item Finalizaci\'{o}n de la redacci\'{o}n de la metodolog\'{\i}a
\item Finalizaci\'{o}n de la redacci\'{o}n del dise\~{n}o de la soluci\'{o}n propuesta
\item Finalizaci\'{o}n de la redacci\'{o}n de la implementaci\'{o}n de la soluci\'{o}n propuesta
\item Finalizaci\'{o}n de la redacci\'{o}n del dise\~{n}o experimental
\item Finalizaci\'{o}n de la redacci\'{o}n de reportaje de resultados experimentales
\item Finalizaci\'{o}n de la redacci\'{o}n del an\'{a}lisis de experimentos
\item Finalizaci\'{o}n de la redacci\'{o}n de la discusi\'{o}n de los experimentos
\item Finalizaci\'{o}n de la redacci\'{o}n de las conclusiones
\item Finalizaci\'{o}n de la redacci\'{o}n de trabajo a futuro
\item Finalizaci\'{o}n de la redacci\'{o}n del formato de la bibliograf\'{\i}a
\item Finalizaci\'{o}n de la redacci\'{o}n del resumen
\item Finalizaci\'{o}n de la redacci\'{o}n de los agradecimientos y la autobiograf\'{\i}a
\item Entrega de portafolio de evidencias
\end{enumerate}

{\em Elementos de competencia:}

\paragraph{Elementos de competencia:}

\quad

\begin{tabular}{|p{28mm}|p{30mm}|p{30mm}|p{30mm}|p{30mm}|}
  \hline
  \cellcolor{tableheader}Evidencias de aprendizaje 
  & \cellcolor{tableheader}Criterios de desempe\~{n}o
  & \cellcolor{tableheader}Actividades de aprendizaje
  & \cellcolor{tableheader}Contenidos
  & \cellcolor{tableheader}Recursos \\ \hline

Avance en la bit\'{a}cora. 
  & 
    Precisi\'{o}n del an\'{a}lisis de avance; nivel de detalle de la
    planeaci\'{o}n de actividades pendientes.
    & Redacci\'{o}n de la bit\'{a}cora; actualizaciones en el documento de tesis.
  &
  Estructuraci\'{o}n y estilo de un traajo formal de tesis de maestr\'{\i}a.
  & Material en la p\'{a}gina web de la unidad y la literatura citada;
  paquete {\LaTeX} para redacci\'{o}n cient\'{\i}fica; libros de texto. \\ \hline
  
  \end{tabular}

\newpage
  
  \section{Evaluaci\'{o}n integral de procesos y productos:}
  
  Las bit\'{a}coras son individuales; se recomienda estudiar juntos y
  discutir las soluciones, pero no se tolera ning\'{u}n tipo de plagio
  en absoluto, ni de otros estudiantes ni de la red ni de libros ---
  toda referencia bibliogr\'{a}fica tiene que ser apropiadamente
  citada. No habr\'{a} examen.

  Son 19 avances semanales m\'{a}s la entrega del portafolio, otorgando por m\'{a}ximo 5 puntos por semana
  \begin{description}[itemsep=0em]
  \item[NP]{= entrega omitida}
  \item[5]{= excede lo que se esperaba}
  \item[4]{= cumple con lo que se esperaba}
  \item[3]{= d\'{e}bil en alcance y/o calidad}
  \item[2]{= d\'{e}bil en ambos alcance y calidad}
  \item[1]{= sin contribuciones o m\'{e}ritos aunque fue entregada}
  \item[0]{= completamente inadecuado en alzance y calidad}
  \end{description}

  
  Ponderaci\'{o}n espec\'{\i}fica:

  \scalebox{0.85}{
    \begin{tabular}{|c|cccccccccccccccccccc|c|}
      \hline
      
      \rotatebox{90}{\cellcolor{tableheader}{\bf Actividad\phantom{xx}}} & A1 & A2 & A3 & A4 & A5 & A6 & A7 & A8 & A9 & A10 &
                                                                                                                              A11
      & A12 & A13 & A14 & A15 & A16 & A17 & A18 & A19 & A20 & \rotatebox{90}{\cellcolor{tableheader}{{\bf Total}}}
      \\
      \hline
      \rotatebox{90}{\cellcolor{tableheader}{\bf Ponderaci\'{o}n\phantom{xx}}}
                                                                         & 5\% & 5\% & 5\% &
                                                                                             5\% &
                                                                                                   5\% & 5\% & 5\% & 5\%
                                                                                                                 & 5\%
                                                                                                                      & 5\%
                                                                                                                            & 5\%
      & 5\% 
            & 5\% & 5\% & 5\% & 5\% & 5\% & 5\% & 5\% & 5\% & 100\%		
      \\ \hline
    \end{tabular}}
  
  \newpage

\section{Producto integrador de aprendizaje de la unidad:}
\subsection{Producto integrador de Aprendizaje:} 

Portafolio de actividades. Cada fase semanal otorga por cinco puntos y
el total m\'{a}ximo es de 100 puntos. El contenido espec\'{\i}fico de
cada fase depende del tema de cada estudiante. El estudiante
mantendr\'{a} una bit\'{a}cora semanal de avances, hecha en \LaTeX,
reportando lo discutido y avanzado en cada semana. Siendo cuarto
semestre, no se espera que un alumno concluya a ning\'{u}n
cap\'{\i}tulo en particular --- se visitan de forma sistem\'{a}tica
todos los elementos de un trabajo de tesis de maestr\'{\i}a para analizar
el avance actual y calendarizar de manera estructurada el trabajo
pendiente para semestres posteriores.

En la hoja portada, incluir logos y nombres completos de la
universidad y de la facultad, mencionar el nombre oficial y completo
del programa de posgrado, su nombre completo y matr\'{\i}cula, nombre
completo de su profesor (bien escrito), nombre completo oficial de la
unidad de aprendizaje (Tesis I, en este caso), semestre en que se
cursa (febrero-junio 2021, por ejemplo). En la segunda hoja, incluir
un cuadro de puntos obtenidos por fase, dejando un cuadro en blanco
para la fase 20, indicando el total acumulado de las fases 1--19 (ah\'{\i}
es donde yo agrego esos puntos y pongo mi firma) Despu\'{e}s, incluir los
19 reportes que fueron calificados (o en su estado original, o en el
caso que hicieron una, la versi\'{o}n corregida), usando el paquete
pdfpages Incluir al final del portafolio la tesis completa en su
estado actual, con firma de Visto Bueno (indicando la fecha) en la
portada por su asesor de tesis o todos coasesores en su caso.

\section{Fuentes de apoyo y consulta:}
\subsection{Fuentes de apoyo y consulta}
\subsubsection{B\'{a}sicas}

 \begin{itemize}[itemsep=0em]

   \item{Secci\'{o}n de los Reglamentos de la UANL, FIME y el posgrado que se relacionan con la realizaci\'{o}n del proyecto de tesis.}

    \item{S.\ {\sc Gimbel}, {\em Exploring the Scientific Method:
          Cases and Questions}, University of Chicago Press (abril 15,
        2011), ISBN-10: 0226294838.}

      \item{ H.L.\ {\sc \'{A}vila Baray}: {\em Introducci\'{o}n a la
            metodolog\'{\i}a de la investigaci\'{o}n}, 2006, Edici\'{o}n electr\'{o}nica. ISBN-10: 84-690-1999-6}
   
\end{itemize}

\subsubsection{Complementarias}

Art\'{\i}culos cient\'{\i}ficos especializados relacionados al tema de tesis.

\label{final} % last page
%\newpage

%\pagestyle{plain}

%\vspace*{3cm}

%{\bf Autoriz\'{o}:} \coordinador

%\vspace*{2cm}

%  \begin{center}
%  {\sc Alere Flammam Veritatis}
  
%  Ciudad Universitaria, \today

%\vspace*{4cm}
  
%  \begin{tabular}{p{6cm}cp{7cm}}
%    \cline{1-1}
%    \cline{3-3}    
%    {\bf \coordinador}
% & \phantom{xxx} &{\bf Vo.\ Bo.\ \subdirector} \\
%    Coordinador Acad\'{e}mico &  &Subdirector de Estudios de Posgrado \\
%    Posgrado en Ingenier\'{\i}a de Sistemas & & Facultad de Ingenier\'{\i}a Mec\'{a}nica y El\'{e}ctrica
                                                                   
%  \end{tabular}
%\end{center}

\end{document}