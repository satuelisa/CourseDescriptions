\section{Estructuraci\'{o}n en cap\'{\i}tulos, etapas o fases de la unidad de
  aprendizaje:}

\subsection{Desarrollo de las fases de la Unidad de Aprendizaje:}

Se cubren los principios te\'{o}ricos y pr\'{a}cticos de la
inteligencia artificial. Se busca desarrollar habilidades en la
resoluci\'{o}n en casos pr\'{a}cticos concretos. Se necesita contar
con un buen entendimiento de varios los conceptos matem\'{a}ticos,
especialmente de matem\'{a}ticas discretas y probabilidad, o en el
caso contrario, estar preparado a estudiarlos seg\'{u}n
necesidad. Tambi\'{e}n se necesita un conocimiento s\'{o}lido de
programaci\'{o}n.

La sesiones son de cuatro horas cada una y son veinte semanas en
total. La primera semana es introductoria y las \'{u}ltimas dos semanas
combinan elementos de las tres unidades tem\'{a}ticas en el contexto del
proyecto integrador.


\begin{enumerate}[itemsep=-3pt]
\item[Ini]{Introducci\'{o}n; selecci\'{o}n de temas de proyecto}
\item[UT1]{M\'{e}todos de b\'{u}squeda (4 semanas)}
\item[UT2]{Planeaci\'{o}n (4 semanas)}
\item[UT3]{Algoritmos evolutivos inteligentes (4 semanas)}
\item[UT4]{Aprendizaje m\'{a}quina (5 semanas)}
\item[PI]{Presentaciones de proyectos}
\item[Rev]{Revisi\'{o}n de portafolios de evidencia}
\end{enumerate}

\subsubsection{Unidad tem\'{a}tica 1: M\'{e}todos de b\'{u}squeda}

\paragraph{Periodo:} 4 semanas

\paragraph{Elementos de competencia:}

\quad

\begin{tabular}{|p{28mm}|p{30mm}|p{30mm}|p{30mm}|p{30mm}|}
  \hline
  \cellcolor{tableheader}Evidencias de aprendizaje 
  & \cellcolor{tableheader}Criterios de desempe\~{n}o
  & \cellcolor{tableheader}Actividades de aprendizaje
  & \cellcolor{tableheader}Contenidos
  & \cellcolor{tableheader}Recursos \\ \hline


Cuatro (4) tareas semanales, siendo cada una un reporte escrito y
c\'{o}digo de la implementaci\'{o}n de un m\'{e}todo de b\'{u}squeda de
inteligencia artificial.

&

Calidad de la redacci\'{o}n cient\'{\i}fica del reporte; precisi\'{o}n
y eficiencia del m\'{e}todo; cobertura de la experimentaci\'{o}n.

&

Experimentaci\'{o}n con ejemplos; lectura de material de apoyo;
modificaci\'{o}n de ejemplos; dise\~{n}o y ejecuci\'{o}n de
experimentos; an\'{a}lisis y reportaje de resultados obtenidos.

&

M\'{e}todos diversos de b\'{u}squeda del campo de la inteligencia
artificial.

&

Material en la p\'{a}gina web de la unidad y la literatura citada;
lenguaje Python o similar; paquete {\LaTeX} para redacci\'{o}n
cient\'{\i}fica; repositorios de GitHub. \\ \hline
  
\end{tabular}

\newpage

\subsubsection{Unidad tem\'{a}tica 2: Planeaci\'{o}n}

\paragraph{Periodo:} 4 semanas

\paragraph{Elementos de competencia:}

\quad

\begin{tabular}{|p{28mm}|p{30mm}|p{30mm}|p{30mm}|p{30mm}|}
  \hline
  \cellcolor{tableheader}Evidencias de aprendizaje 
  & \cellcolor{tableheader}Criterios de desempe\~{n}o
  & \cellcolor{tableheader}Actividades de aprendizaje
  & \cellcolor{tableheader}Contenidos
  & \cellcolor{tableheader}Recursos \\ \hline


Cuatro (4) tareas semanales, siendo cada una un reporte escrito y
c\'{o}digo de la implementaci\'{o}n de un m\'{e}todo de planeaci\'{o}n.

&

Calidad de la redacci\'{o}n cient\'{\i}fica del reporte; precisi\'{o}n
del m\'{e}todo; eficiencia de la implementaci\'{o}n del m\'{e}todo;
cobertura de la experimentaci\'{o}n.

&

Experimentaci\'{o}n con ejemplos; lectura de material de apoyo;
modificaci\'{o}n de ejemplos; dise\~{n}o y ejecuci\'{o}n de
experimentos; an\'{a}lisis y reportaje de resultados obtenidos.

&

M\'{e}todos diversos de planeaci\'{o}n.

&

Material en la p\'{a}gina web de la unidad y la literatura citada;
lenguaje Python o similar; paquete {\LaTeX} para redacci\'{o}n
cient\'{\i}fica; repositorios de GitHub. \\ \hline
  
\end{tabular}


\subsubsection{Unidad tem\'{a}tica 3: Algoritmos evolutivos
  inteligentes}

\paragraph{Periodo:} 4 semanas

\paragraph{Elementos de competencia:}

\quad

\begin{tabular}{|p{28mm}|p{30mm}|p{30mm}|p{30mm}|p{30mm}|}
  \hline
  \cellcolor{tableheader}Evidencias de aprendizaje 
  & \cellcolor{tableheader}Criterios de desempe\~{n}o
  & \cellcolor{tableheader}Actividades de aprendizaje
  & \cellcolor{tableheader}Contenidos
  & \cellcolor{tableheader}Recursos \\ \hline


Cuatro (4) tareas semanales, siendo cada una un reporte escrito y
c\'{o}digo de la implementaci\'{o}n de un m\'{e}todo evolutivo.

&

Calidad de la redacci\'{o}n cient\'{\i}fica del reporte; precisi\'{o}n
del m\'{e}todo; eficiencia de la implementaci\'{o}n del
m\'{e}todo; cobertura de la experimentaci\'{o}n.

&

Experimentaci\'{o}n con ejemplos; lectura de material de apoyo;
modificaci\'{o}n de ejemplos; dise\~{n}o y ejecuci\'{o}n de
experimentos; an\'{a}lisis y reportaje de resultados obtenidos.

&

M\'{e}todos evolutivos diversos de la inteligencia artificial.

&

Material en la p\'{a}gina web de la unidad y la literatura citada;
lenguaje Python o similar; paquete {\LaTeX} para redacci\'{o}n
cient\'{\i}fica; repositorios de GitHub. \\ \hline
  
\end{tabular}

\newpage 

\subsubsection{Unidad tem\'{a}tica 4: Aprendizaje m\'{a}quina}

\paragraph{Periodo:} 5 semanas

\paragraph{Elementos de competencia:}

\quad

\begin{tabular}{|p{28mm}|p{30mm}|p{30mm}|p{30mm}|p{30mm}|}
  \hline
  \cellcolor{tableheader}Evidencias de aprendizaje 
  & \cellcolor{tableheader}Criterios de desempe\~{n}o
  & \cellcolor{tableheader}Actividades de aprendizaje
  & \cellcolor{tableheader}Contenidos
  & \cellcolor{tableheader}Recursos \\ \hline


Cinco (5) tareas semanales, siendo cada una un reporte escrito y
c\'{o}digo de la implementaci\'{o}n de un m\'{e}todo de aprendizaje de
m\'{a}quina.

&

Calidad de la redacci\'{o}n cient\'{\i}fica del reporte; precisi\'{o}n
del m\'{e}todo; eficiencia de la implementaci\'{o}n del
m\'{e}todo; cobertura de la experimentaci\'{o}n.

&

Experimentaci\'{o}n con ejemplos; lectura de material de apoyo;
modificaci\'{o}n de ejemplos; dise\~{n}o y ejecuci\'{o}n de
experimentos; an\'{a}lisis y reportaje de resultados obtenidos.

&

M\'{e}todos diversos de aprendizaje m\'{a}quina.

&

Material en la p\'{a}gina web de la unidad y la literatura citada;
lenguaje Python o similar; paquete {\LaTeX} para redacci\'{o}n
cient\'{\i}fica; repositorios de GitHub. \\ \hline
  
\end{tabular}
