\section{Presentaci\'{o}n:}

La {\em visi\'{o}n computacional} refiere al procesamiento
automatizado de im\'{a}genes para extrar informaci\'{o}n para sistemas
de toma de decisiones.  En procesamiento de nivel bajo, se trabaja
directamente con las im\'{a}genes para extraer propiedades como
orillas, gradiente, profundidad, textura, color, etc. Procesamiento de
nivel intermedio consiste generalmente en agrupar los elementos
obtenidos en el nivel bajo, para obtener, por ejemplo, contornos y
regiones, generalmente con el prop\'{o}sito de
segmentaci\'{o}n. Procesamiento de alto nivel, por \'{u}ltimo,
consiste en la interpretaci\'{o}n de los entes obtenidos en los
niveles inferiores y se utilizan modelos y/o conocimiento {\em a
  priori} del dominio

\section{Prop\'{o}sito(s):}

Introducci\'{o}n a la visi\'{o}n computacional que trata de emular esta
capacidad en las computadoras, de forma que, mediante la
interpretaci\'{o}n de las im\'{a}genes adquiridas, por ejemplo, con una
c\'{a}mara, se puedan reconocer los diversos objetos en el ambiente y su
posici\'{o}n en el espacio.
