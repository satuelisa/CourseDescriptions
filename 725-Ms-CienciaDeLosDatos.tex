\documentclass[10 pt]{article}
\usepackage{wallpaper}
\usepackage[spanish, mexico]{babel}  
\usepackage{color,colortbl}
\usepackage[T1]{fontenc}
\usepackage{fancyhdr} % before geometry
\usepackage[letterpaper,left=18mm,right=18mm,headheight=5mm,headsep=40mm,top=52mm,bottom=22mm]{geometry}
\usepackage{graphicx}
\usepackage[latin1]{inputenc}
\usepackage{hyperref}
\usepackage{graphics}
\usepackage{varwidth}
\usepackage{tikz}
\usetikzlibrary{shapes,arrows}
\tikzstyle{elem} = [draw, rectangle, thick, minimum height=2em,
minimum width=2em, execute at begin node={\begin{varwidth}{14em}},
   execute at end node={\end{varwidth}}]
\tikzstyle{line} = [draw, thick, -stealth, shorten >=0pt]
\usepackage{multicol}
\usepackage{wrapfig}
\usepackage{enumitem}
\renewcommand*\familydefault{\sfdefault} 
\renewcommand{\thesection}{\Roman{section}} 
\usepackage{remreset}
\makeatletter
\@removefromreset{subsection}{section}
\renewcommand\thesubsection{\arabic{subsection}}
\makeatother
\usepackage{titlesec}
\ULCornerWallPaper{1}{logos.pdf}
\setlength{\parindent}{1em}
\setlength{\parskip}{2pt}
\usepackage{tcolorbox}
\definecolor{headerframe}{RGB}{177,178,177} % #b1b2b1
\definecolor{headercontent}{RGB}{239,239,239} % #efefef
\definecolor{tableheader}{RGB}{185,208,238}% #b9d0ee
\tcbset{mystyle/.style={
  breakable,
  enhanced,
  roundcorner = 30pt,
  outer arc = 4pt,
  arc = 4pt,
  colframe = headerframe,
  colback = headercontent,
  fonttitle=\sffamily
  }
}
\usepackage{amssymb}
\newcommand{\yes}{\makebox[0pt][l]{$\square$}{\raisebox{0.1\height}{$\times$}}}
\newcommand{\no}{\makebox[0pt][l]{$\square$}{\raisebox{0.1\height}{\phantom{$\times$}}}}
\usepackage{enumitem}
\titleformat{\section}{\normalfont\large\bfseries}{\thesection.}{4pt}{}
\titleformat{\subsection}[runin]{\normalfont\bfseries}{\thesubsection.}{4pt}{}


\newcommand{\narturo}{095012}
\newcommand{\nelisa}{096633}
\newcommand{\ncesar}{092038}
\newcommand{\nangy}{102662}
\newcommand{\nvincent}{102947}
\newcommand{\nada}{060581}
\newcommand{\niris}{103743}
\newcommand{\nsara}{100546}
\newcommand{\nroger}{090969}
\newcommand{\nigor}{093179}
\newcommand{\nromeo}{100959}
\newcommand{\nferny}{095808}
\newcommand{\arturo}{Dr.\ Jos\'{e} Arturo Berrones Santos}
\newcommand{\elisa}{Dra.\ Satu Elisa Schaeffer}
\newcommand{\cesar}{Dr.\ C\'{e}sar Emilio Villarreal Rodr\'{\i}guez}
\newcommand{\angy}{Dra.\ Mar\'{\i}a Ang\'{e}lica Salazar Aguilar}
\newcommand{\vincent}{Dr.\ Vincent Boyer}
\newcommand{\ada}{Dra.\ Ada Margarita \'{A}lvarez Socarr\'{a}s}
\newcommand{\iris}{Dra.\ Iris Abril Mart\'{\i}nez Salazar}
\newcommand{\sara}{Dra.\ Sara Ver\'{o}nica S\'{a}nchez Rodr\'{\i}guez}
\newcommand{\roger}{Dr.\ Roger Zirahu\'{e}n R\'{\i}os Mercado}
\newcommand{\igor}{Dr.\ Igor Litvinchev}
\newcommand{\romeo}{Dr.\ Romeo S\'{a}nchez Nigenda}
\newcommand{\ferny}{Dr.\ Fernando L\'{o}pez Irarragorri}
\newcommand{\simon}{Dr.\ Sim\'{o}n Mart\'{\i}nez Mart\'{\i}nez} 

\newcommand{\subdirector}{\simon}
\newcommand{\coordinador}{\cesar}


\begin{document}
\fancyhf{}
\rhead{IT-8-SPG-02-R03}
\lfoot{Revisión: 1 \\
  Vigente a partir del: 01 de agosto del 2016}

\begin{tcolorbox}
  \begin{center}

    {\bf \UANL}

    \medskip

    {\bf \fime}

    \medskip
    
    {\bf PE} \underline{\bf \maestria}

    \medskip

    \underline{\PA}

  \end{center}
\end{tcolorbox}

\section{Datos de Identificaci\'{o}n de la Unidad de Aprendizaje:}
\subsection{Nombre:} \fbox{Ciencia de los datos}
\subsection{Frecuencia semanal:} horas de trabajo presencial \fbox{4}
\subsection{Horas de trabajo extra aula por semana:} \fbox{2}
\subsection{Modalidad:} \yes~Escolarizada \no~No escolarizada \no~Mixto
\subsection{Per\'{\i}odo acad\'{e}mico:} \yes~Semestral
\no~Tetramestral \no~Modular
\subsection{LGAC:} \underline{\seys}
\subsection{Ubicaci\'{o}n semestral:} \underline{1 o 2}
\subsection{\'{A}rea curricular:} \underline{\fb, \fa, \da, \le, \inv}
\subsection{Cr\'{e}ditos:} \underline{4}
\subsection{Requisito:} \underline{Ninguno}
\subsection{Fecha de elaboraci\'{o}n:} \underline{20/01/2010}
\subsection{Fecha de la \'{u}ltima actualizaci\'{o}n:} \underline{10/06/2020}
\subsection{Responsable(s) del dise\~{n}o:}
\begin{itemize}[label={}]
\item \underline{\nferny~\ferny}
\item \underline{\nelisa~\elisa}
\end{itemize}
\newpage
\section{Presentaci\'{o}n:}

La {\em ciencia de datos} consiste en aplicar herramientas
computacionales, m\'{e}todos estad\'{\i}sticos y modelos
matem\'{a}ticos en general a conjuntos de datos multivariados, de
m\'{u}ltiples tipos de entrada, posiblemente en diversos formatos,
grandes cantidades, potencialmente conteniendo ruido, errores,
omisiones y duplicados con la finalidad de poder obtener respuestas
estad\'{\i}sticamente respaldadas a preguntas de inter\'{e}s.

\section{Prop\'{o}sito(s):}

Que el estudiante se familiarice con los conceptos fundamentales de
anal\'{\i}tica de datos, conozca y sea capaz de usar las tecnolog\'{\i}as m\'{a}s
populares de anal\'{\i}tica para resolver problemas reales.

\section{Competencias del perfil de egreso:}
\subsection{Competencias del perfil de egreso}

\input{p1m.tex}

\input{p2.tex}

P3) Establecer comunicaci\'{o}n con los disGntos sectores de la
sociedad a fin de establecer proyectos estrat\'{e}gicos en las
distintas disciplinas de la ingenier\'{\i}a de sistemas y crear la
cultura de la creaci\'{o}n de riqueza basada en el conocimiento.


\subsection{Competencias generales a que se vincula la Unidad de
  Aprendizaje:}

La unidad se vincula con las siguientes competencias generales:

\phantom{space}
\begin{tabular}{|p{12cm}|p{30mm}|}
  \hline
  \cellcolor{tableheader}{\em Declaraci\'{o}n de la competencia general vinculada a la unidad
  de aprendizaje}
  & \cellcolor{tableheader}{\em Evidencia} \\ \hline
C2) Utiliza los lenguajes l\'{o}gico, formal, matem\'{a}tico, ic\'{o}nico, verbal
y no verbal de acuerdo a su etapa de vida en el \'{a}rea de las ciencias
para comprender, interpretar y expresar ideas, sentimientos, teor\'{\i}as y
corrientes de pensamiento con un enfoque ecum\'{e}nico.
 & Tareas \\ \hline
C3) Maneja las tecnolog\'{\i}as de la informaci\'{o}n de acuerdo a los usos del
campo de las ciencias y la comunicaci\'{o}n como herramientas para el
acceso a la informaci\'{o}n y su transformaci\'{o}n en conocimiento, as\'{\i} como
para el aprendizaje y trabajo colaborativo con t\'{e}cnicas de vanguardia
que le permitan su participaci\'{o}n constructiva en la sociedad.
 & Tareas, proyecto \\ \hline
C5) Emplea pensamiento l\'{o}gico, cr\'{\i}tico, creativo y propositivo,
siguiendo los modelos de pensamiento cient\'{\i}fico para analizar
fen\'{o}menos naturales y sociales que le permitan tomar decisiones
pertinentes en su \'{a}mbito de influencia con responsabilidad social.
 & Tareas, proyecto \\ \hline
\end{tabular}

\newpage

\subsection{Competencias espec\'{\i}ficas y nivel de dominio a que se vincula la unidad de aprendizaje:}

La unidad se vincula con las siguientes competencias espec\'{\i}ficas:

\phantom{space}
\begin{tabular}{|p{30mm}|p{4mm}|p{4mm}|p{25mm}|p{12mm}|p{25mm}|p{12mm}|p{4mm}|p{12mm}|}
\hline
\cellcolor{tableheader}{{\em Competencia Espec\'{\i}fica}}
& \rotatebox{90}{\cellcolor{tableheader}{Nivel I Inicial}}
& \rotatebox{90}{\cellcolor{tableheader}{Evidencia}}
& \rotatebox{90}{\cellcolor{tableheader}{Nivel II B\'{a}sico}}
& \rotatebox{90}{\cellcolor{tableheader}{Evidencia}}
& \rotatebox{90}{\cellcolor{tableheader}{Nivel III Aut\'{o}nomo}}
& \rotatebox{90}{\cellcolor{tableheader}{Evidencia}}
& \rotatebox{90}{\cellcolor{tableheader}{Nivel IV Estrat\'{e}gico\phantom{xxx}}}
& \rotatebox{90}{\cellcolor{tableheader}{Evidencia}}
\\ \hline


\input{e1.tex}
& & & Resuelve problemas de libro de texto en el \'{a}rea de toma de decisiones
con bases cient\'{\i}ficas

& Tareas.
&  Encuentra soluciones para la consecuci\'{o}n de objetivos establecidos
para un problema dado, revisando literatura cient\'{\i}fica de
frontera.

& Tareas.
\input{end_bas_aut.tex}

\section{Representaci\'{o}n gr\'{a}fica:}

\begin{center}
\begin{tikzpicture}[scale=1, auto]
  \matrix[row sep=1cm, column sep=7mm]{

& 
\node[elem] (ide) {Identificar un problema de inter\'{e}s que se
  busca resolver};
\\
\node[elem] (lit) {Identificar y revisar literatura cient\'{\i}fica relacionada relevante};
&
\node[elem] (des) {Obtener los {\bf datos de entrada} por utilizar};
&
\node[elem] (her) {Revisar herramientas computacionales
  existentes para implementar el m\'{e}todo};
\\
\node[elem] (for) {Formular hip\'{o}tesis para explorar en los datos};
&
\node[elem] (imp) {Aplicar m\'{e}todos de ciencia de datos con las herramientas seleccionadas};
\\
& \node[elem] (dis) {Dise\~{n}ar y realizar experimentos para
  determinar el significado estad\'{\i}stico de los hallazgos};
\\
\node[elem] (ana) {Analizar los resultados de los experimentos y contrastarlas con las hip\'{o}tesis};
&  
\node[elem] (doc) {Documentar el proceso conforme al estilo de
  redacci\'{o}n cient\'{\i}fica};
& 
\node[elem] (apl) {Aplicar visualizaci\'{o}n cient\'{\i}fica a los resultados};
\\
& 
& 
\node[elem] (exp) {Discutir entre pares los m\'{e}todos y los
  resultados obtenidos};
\\
};
\draw [line] (ide) -- (lit);
\draw [line] (ide) -- (des);
\draw [line] (ide) -- (her);
\draw [line] (lit) -- (des);
\draw [line] (lit) -- (for);
\draw [line] (her) -- (imp);
\draw [line] (des) -- (imp);
\draw [line] (for) -- (dis);
\draw [line] (imp) -- (dis);
\draw [line] (dis) -- (ana);
\draw [line] (dis) -- (apl);
\draw [line] (ana) -- (doc);
\draw [line] (apl) -- (doc);
\draw [line] (doc) -- (exp);
\input{enddiagram.tex}

\newpage

\section{Estructuraci\'{o}n en cap\'{\i}tulos, etapas o fases de la unidad de
  aprendizaje:}
\subsection{Desarrollo de las fases de la Unidad de Aprendizaje:}

Se busca desarrollar habilidades en la resoluci\'{o}n en casos
pr\'{a}cticos concretos. Se necesita contar con un buen entendimiento
de varios los conceptos matem\'{a}ticos, especialmente de
matem\'{a}ticas discretas y probabilidad, o en el caso contrario,
estar preparado a estudiarlos seg\'{u}n necesidad. Tambi\'{e}n se
necesita conocimiento de programaci\'{o}n.  La sesiones son de cuatro
horas cada una y son veinte semanas en total.
\begin{enumerate}[itemsep=-3pt]
\item{Introducci\'{o}n; selecci\'{o}n de temas de proyecto}
\item{Preparaci\'{o}n de datos}
\item{Lectura y manipulaci\'{o}n de datos}
\item{Estad\'{\i}stica descriptiva b\'{a}sica}
\item{Detecci\'{o}n de anomalias}
\item{Visualizaci\'{o}n est\'{a}tica de informaci\'{o}n}
\item{Visualizaci\'{o}n din\'{a}mica de informaci\'{o}n}
\item{Pruebas estad\'{\i}sticas}
\item{Modelos lineales}
\item{Regresi\'{o}n m\'{u}ltiple}
\item{An\'{a}lisis de varianza y de componentes principales}
\item{Pron\'{o}sticos}
\item{Clasificaci\'{o}n de datos}
\item{Agrupamiento de datos}
\item{An\'{a}lisis de texto}
\item{An\'{a}lisis de im\'{a}genes}
\item{Datos grandes}
\item{Procesamiento en tiempo real}
\item{Presentaciones de proyectos}
\item{Revisi\'{o}n de portafolios de evidencia}
\end{enumerate}

{\em Elementos de competencia:}

\paragraph{Elementos de competencia:}

\quad

\begin{tabular}{|p{28mm}|p{30mm}|p{30mm}|p{30mm}|p{30mm}|}
  \hline
  \cellcolor{tableheader}Evidencias de aprendizaje 
  & \cellcolor{tableheader}Criterios de desempe\~{n}o
  & \cellcolor{tableheader}Actividades de aprendizaje
  & \cellcolor{tableheader}Contenidos
  & \cellcolor{tableheader}Recursos \\ \hline


Reporte escrito y c\'{o}digo de la implementaci\'{o}n de un m\'{e}todo
de ciencia de datos.

&

Calidad de la redacci\'{o}n cient\'{\i}fica del reporte; 
precisi\'{o}n del algoritmo propuesto;
eficiencia de la implementaci\'{o}n del algoritmo;
cobertura de la experimentaci\'{o}n.

&

Experimentaci\'{o}n con ejemplos; lectura de material de apoyo;
modificaci\'{o}n de ejemplos; dise\~{n}o y ejecuci\'{o}n de experimentos;
an\'{a}lisis y reportaje de resultados obtenidos.

&

M\'{e}todos diversos de ciencia de datos

&

Material en la p\'{a}gina web de la unidad y la literatura citada;
lenguaje Python o similar; paquete {\LaTeX} para redacci\'{o}n cient\'{\i}fica;
repositorios de GitHub. \\ \hline

\end{tabular}

\newpage

\section{Evaluaci\'{o}n integral de procesos y productos:}

Las tareas son individuales; se recomienda estudiar juntos y discutir
las soluciones, pero no se tolera ning\'{u}n tipo de plagio en
absoluto, ni de otros estudiantes ni de la red ni de libros --- toda
referencia bibliogr\'{a}fica tiene que ser apropiadamente citada. La
entrega se realiza por un repositorio en GitHub que debe reflejar
todas las fases del trabajo. El alumno
selecciona su lenguaje de programaci\'{o}n para cada tarea.  No
habr\'{a} examen.

Son 17 tareas (A1--A17) que reportan avances semanales de aplicaci\'{o}n
de la lectura de la semana para el proyecto del estudiante,
otorgando por m\'{a}ximo 5 puntos por tarea:
\begin{description}[itemsep=-2pt]
\item[NP]{= tarea omitida}
\item[5]{= excede lo que se esperaba}
\item[4]{= cumple con lo que se esperaba}
\item[3]{= d\'{e}bil en alcance y/o calidad}
\item[2]{= d\'{e}bil en ambos alcance y calidad}
\item[1]{= sin contribuciones o m\'{e}ritos aunque fue entregada}
\item[0]{= completamente inadecuado en alzance y calidad}
\end{description}

El proyecto final (A18) otorga un m\'{a}ximo de 15 puntos, evaluados en los
siguientes rubros
\begin{enumerate}[itemsep=0em]
\item{Variedad de t\'{e}cnicas de empleadas}
\item{Cobertura y validez de la experimentaci\'{o}n}
\item{Claridad y relevancia de los resultados}    
\item{Calidad de visualizaci\'{o}n cient\'{\i}fica}
\item{Calidad de redacci\'{o}n cient\'{\i}fica}
\end{enumerate}

con la escala:
\begin{description}[itemsep=0em]
\item[3]{= cumple con lo que se esperaba}
\item[2]{= d\'{e}bil en alcance y/o calidad}
\item[1]{= d\'{e}bil en ambos alcance y calidad}
\item[0]{= inadecuado en alzance y calidad}
\end{description}


Ponderaci\'{o}n espec\'{\i}fica:

\section{Evaluaci\'{o}n integral de procesos y productos:}

Las tareas son individuales; se recomienda estudiar juntos y discutir
las soluciones, pero no se tolera ning\'{u}n tipo de plagio en
absoluto, ni de otros estudiantes ni de la red ni de libros --- toda
referencia bibliogr\'{a}fica tiene que ser apropiadamente citada. La
entrega se realiza por un repositorio p\'{u}blico que debe reflejar todas
las fases del trabajo. 

No habr\'{a} examen.  Son 17 tareas (A1--A17) que reportan avances
semanales de aplicaci\'{o}n de la lectura de la semana para el
proyecto del estudiante, otorgando por m\'{a}ximo 5 puntos por
tarea: \begin{description}[itemsep=-2pt]
\item[NP]{= tarea omitida}
\item[5]{= excede lo que se esperaba}
\item[4]{= cumple con lo que se esperaba}
\item[3]{= d\'{e}bil en alcance y/o calidad}
\item[2]{= d\'{e}bil en ambos alcance y calidad}
\item[1]{= sin contribuciones o m\'{e}ritos aunque fue entregada}
\item[0]{= completamente inadecuado en alzance y calidad}
\end{description}
 El proyecto final (A18) otorga un
m\'{a}ximo de 15 puntos, evaluados en los siguientes
rubros \begin{enumerate}[itemsep=0em]
\item{Variedad de t\'{e}cnicas de empleadas}
\item{Cobertura y validez de la experimentaci\'{o}n}
\item{Claridad y relevancia de los resultados}    
\item{Calidad de visualizaci\'{o}n cient\'{\i}fica}
\item{Calidad de redacci\'{o}n cient\'{\i}fica}
\end{enumerate}
 con la escala: \begin{description}[itemsep=0em]
\item[3]{= cumple con lo que se esperaba}
\item[2]{= d\'{e}bil en alcance y/o calidad}
\item[1]{= d\'{e}bil en ambos alcance y calidad}
\item[0]{= inadecuado en alzance y calidad}
\end{description}


\paragraph{Ponderaci\'{o}n espec\'{\i}fica}

\quad

\scalebox{0.85}{
  \begin{tabular}{|c|cccccccccccccccccc|c|}
    \hline
    \rotatebox{90}{\cellcolor{tableheader}{\bf Actividad\phantom{xx}}}
    & A1
    & A2
    & A3
    & A4
    & A5
    & A6
    & A7
    & A8
    & A9
    & A10
    & A11
    & A12
    & A13
    & A14
    & A15
    & A16
    & A17
    & A18
    & {\bf Total}
    \\
    \hline
    \rotatebox{90}{\cellcolor{tableheader}{\bf Ponderaci\'{o}n\phantom{xx}}}
    & 5\%
    & 5\%
    & 5\%
    & 5\% 
    & 5\%
    & 5\%
    & 5\%
    & 5\%
    & 5\%
    & 5\%
    & 5\%
    & 5\% 
    & 5\%
    & 5\%
    & 5\%
    & 5\%
    & 5\%
    & 15\%
    & 100\%		
    \\ \hline
  \end{tabular}}
  

  
\newpage

\section{Producto integrador de aprendizaje de la unidad:}
\subsection{Producto integrador de Aprendizaje:} Portafolio en un repositorio digital p\'{u}blico.

\section{Fuentes de apoyo y consulta:}
\subsection{Fuentes de apoyo y consulta}
\subsubsection{B\'{a}sicas}

\begin{itemize}[itemsep=0em]
  
\item{D.T.\ {\sc Larose} \& C.D.\ {\sc Larose}: {\em Data Science Using Python and R}
    Wiley, abril 2019, 256 p\'{a}ginas ISBN-13 978-1119526810}
  
\item{H.\ {\sc Wickham} \& G.\ {\sc  Grolemund}:
    {\em R for Data Science: Import, Tidy, Transform, Visualize, and Model Data},
    O'Reilly Media, enero 2017, 
    520 p\'{a}ginas, ISBN-13 978-1491910399}
  
\item{M.\ {\sc Trovati}, et al., eds.: {\em Big-Data Analy;cs and
      Cloud Compu;ng: Theory, Algorithms and Applications}. Springer,
    2016.}
\end{itemize}

\subsubsection{Complementarias}

Art\'{\i}culos cient\'{\i}ficos especializados.

\label{final} % last page
\newpage

\pagestyle{plain}

\vspace*{3cm}

{\bf Autoriz\'{o}:} \coordinador

\vspace*{2cm}

  \begin{center}
  {\sc Alere Flammam Veritatis}
  
  Ciudad Universitaria, \today

\vspace*{4cm}
  
  \begin{tabular}{p{6cm}cp{7cm}}
    \cline{1-1}
    \cline{3-3}    
    {\bf \coordinador} &
                                                            \phantom{xxx} &{\bf Vo.\ Bo.\ \subdirector} \\
    Coordinador Acad\'{e}mico &  &Subdirector de Estudios de Posgrado \\
    Posgrado en Ingenier\'{\i}a de Sistemas & & Facultad de Ingenier\'{\i}a Mec\'{a}nica y El\'{e}ctrica
                                                                   
  \end{tabular}
\end{center}

\label{final} % last page
\newpage

\pagestyle{plain}

\vspace*{3cm}

{\bf Autoriz\'{o}:} \coordinador

\vspace*{2cm}

  \begin{center}
  {\sc Alere Flammam Veritatis}
  
  Ciudad Universitaria, \today

\vspace*{4cm}
  
  \begin{tabular}{p{6cm}cp{7cm}}
    \cline{1-1}
    \cline{3-3}    
    {\bf \coordinador} &
                                                            \phantom{xxx} &{\bf Vo.\ Bo.\ \subdirector} \\
    Coordinador Acad\'{e}mico &  &Subdirector de Estudios de Posgrado \\
    Posgrado en Ingenier\'{\i}a de Sistemas & & Facultad de Ingenier\'{\i}a Mec\'{a}nica y El\'{e}ctrica
                                                                   
  \end{tabular}
\end{center}

\end{document}