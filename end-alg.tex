\section{Estructuraci\'{o}n en cap\'{\i}tulos, etapas o fases de la unidad de
  aprendizaje:}


\subsection{Desarrollo de las fases de la Unidad de Aprendizaje:}

\quad

Se cubren los principios te\'{o}ricos del an\'{a}lisis y el dise\~{n}o
de algoritmos computacionales. Desarrollar habilidades en el
dise\~{n}o como en el an\'{a}lisis en casos pr\'{a}cticos concretos
basados en algoritmos cl\'{a}sicos. Se necesita contar con un buen
entendimiento de varios los conceptos matem\'{a}ticos, especialmente
de matem\'{a}ticas discretas y probabilidad, o en el caso contrario,
estar preparado a estudiarlos seg\'{u}n necesidad. Tambi\'{e}n se
necesita conocimiento de programaci\'{o}n.

\paragraph{Unidades tem\'{a}ticas}

\begin{description}[itemsep=-2pt]
\item[U1]{Fundamentos de la complejidad computacional (8 semanas)}
\item[U2]{Elementos b\'{a}sicos de algoritmos (7 semanas)}
\item[U3]{Algoritmos no exactos (2 semanas)}
\end{description}

La sesiones son de cuatro horas cada una y son veinte semanas en
total. Las \'{u}ltimas tres semanas son para el desarrollo del
proyecto integrador que combina elementos de las tres unidades de
aprendizaje.


\paragraph{Temario semanal}

\quad

\begin{itemize}[itemsep=-3pt]
\item{Introducci\'{o}n; selecci\'{o}n de temas de proyecto}
\item{U1: Problemas y algoritmos (2 semanas)}
\item{U1: Modelos de computaci\'{o}n (2 semanas)} 
\item{U1: Complejidad computacional de problemas (2 semanas)}
\item{U1: Clases de complejidad (2 semanas)} 
\item{U2: Estructuras de datos (2 semanas)} 
\item{U2: An\'{a}lisis de algoritmos (2 semanas)}
\item{U2: T\'{e}cnicas de dise\~{n}o de algoritmos (2 semanas)} 
\item{U2: Optimizaci\'{o}n combinatoria (1 semana)} 
\item{U3: Algoritmos de aproximaci\'{o}n (1 semana).}
\item{U3: Algoritmos aleatorizados (1 semana)} 
\item{Presentaciones de proyectos}
\item{Revisi\'{o}n de portafolios de evidencia}
\end{itemize}

\newpage

\subsubsection{Unidad tem\'{a}tica 1: Fundamentos de la complejidad computacional}

\paragraph{Periodo:} 8 semanas

\paragraph{Elementos de competencia:}

\quad

\begin{tabular}{|p{28mm}|p{30mm}|p{30mm}|p{30mm}|p{30mm}|}
  \hline
  \cellcolor{tableheader}Evidencias de aprendizaje 
  & \cellcolor{tableheader}Criterios de desempe\~{n}o
  & \cellcolor{tableheader}Actividades de aprendizaje
  & \cellcolor{tableheader}Contenidos
  & \cellcolor{tableheader}Recursos \\ \hline


Ocho (8) tareas semanales que consisten cada una en un reporte
escrito de un aspecto b\'{a}sico de la teor\'{\i}a de la computaci\'{o}n.

&

Calidad de la redacci\'{o}n cient\'{\i}fica del reporte.

&

Experimentaci\'{o}n con ejemplos; lectura de material de apoyo;
modificaci\'{o}n de ejemplos; dise\~{n}o y ejecuci\'{o}n de
experimentos; an\'{a}lisis y reportaje de resultados obtenidos.

&

Conceptos b\'{a}sicos diversos de dise\~{n}o y an\'{a}lisis de algoritmos.

&

Material en la p\'{a}gina web de la unidad y la literatura citada;
lenguaje Python o similar; paquete {\LaTeX} para redacci\'{o}n
cient\'{\i}fica; repositorios de p\'{u}blicos de c\'{o}digo fuente. \\
\hline

\end{tabular}

\subsubsection{Unidad tem\'{a}tica 2: Elementos b\'{a}sicos de algoritmos}

\paragraph{Periodo:} 7 semanas

\paragraph{Elementos de competencia:}

\quad

\begin{tabular}{|p{28mm}|p{30mm}|p{30mm}|p{30mm}|p{30mm}|}
  \hline
  \cellcolor{tableheader}Evidencias de aprendizaje 
  & \cellcolor{tableheader}Criterios de desempe\~{n}o
  & \cellcolor{tableheader}Actividades de aprendizaje
  & \cellcolor{tableheader}Contenidos
  & \cellcolor{tableheader}Recursos \\ \hline


Siete (7) tareas semanales consistiendo en un reporte escrito y
c\'{o}digo de la implementaci\'{o}n de un algoritmo y su an\'{a}lisis.

&

Calidad de la redacci\'{o}n cient\'{\i}fica del reporte; precisi\'{o}n
del algoritmo propuesto; eficiencia de la implementaci\'{o}n del
algoritmo; cobertura de la experimentaci\'{o}n.

&

Experimentaci\'{o}n con ejemplos; lectura de material de apoyo;
modificaci\'{o}n de ejemplos; dise\~{n}o y ejecuci\'{o}n de
experimentos; an\'{a}lisis y reportaje de resultados obtenidos.

&

Diversos algoritmos fundamentales.

&

Material en la p\'{a}gina web de la unidad y la literatura citada;
lenguaje Python o similar; paquete {\LaTeX} para redacci\'{o}n cient\'{\i}fica;
repositorios de p\'{u}blicos de c\'{o}digo fuente. \\ \hline

\end{tabular}

\newpage

\subsubsection{Unidad tem\'{a}tica 3: Algoritmos no exactos}

\paragraph{Periodo:} 2 semanas

\paragraph{Elementos de competencia:}

\quad

\begin{tabular}{|p{28mm}|p{30mm}|p{30mm}|p{30mm}|p{30mm}|}
  \hline
  \cellcolor{tableheader}Evidencias de aprendizaje 
  & \cellcolor{tableheader}Criterios de desempe\~{n}o
  & \cellcolor{tableheader}Actividades de aprendizaje
  & \cellcolor{tableheader}Contenidos
  & \cellcolor{tableheader}Recursos \\ \hline


Dos (2) tareas consistiendo cada una en un reporte escrito y
c\'{o}digo de la implementaci\'{o}n un algoritmo no exacto y su
an\'{a}lisis.

&

Calidad de la redacci\'{o}n cient\'{\i}fica del reporte; precisi\'{o}n
del algoritmo propuesto; eficiencia de la implementaci\'{o}n del
algoritmo; cobertura de la experimentaci\'{o}n.

&

Experimentaci\'{o}n con ejemplos; lectura de material de apoyo;
modificaci\'{o}n de ejemplos; dise\~{n}o y ejecuci\'{o}n de experimentos;
an\'{a}lisis y reportaje de resultados obtenidos.

&

M\'{e}todos diversos de algoritmos no exactos.

&

Material en la p\'{a}gina web de la unidad y la literatura citada;
lenguaje Python o similar; paquete {\LaTeX} para redacci\'{o}n cient\'{\i}fica;
repositorios de p\'{u}blicos de c\'{o}digo fuente. \\ \hline

\end{tabular}

\newpage
  
\section{Evaluaci\'{o}n integral de procesos y productos:}

Las tareas son individuales; se recomienda estudiar juntos y discutir
las soluciones, pero no se tolera ning\'{u}n tipo de plagio en
absoluto, ni de otros estudiantes ni de la red ni de libros --- toda
referencia bibliogr\'{a}fica tiene que ser apropiadamente citada. La
entrega se realiza por un repositorio p\'{u}blico que debe reflejar todas
las fases del trabajo. 

No habr\'{a} examen.  Son 17 tareas (A1--A17) que reportan avances
semanales de aplicaci\'{o}n de la lectura de la semana para el
proyecto del estudiante, otorgando por m\'{a}ximo 5 puntos por
tarea: \begin{description}[itemsep=0em]
\item[NP]{= tarea omitida}
\item[5]{= excede lo que se esperaba}
\item[4]{= cumple con lo que se esperaba}
\item[3]{= d\'{e}bil en alcance y/o calidad}
\item[2]{= d\'{e}bil en ambos alcance y calidad}
\item[1]{= sin contribuciones o m\'{e}ritos aunque fue entregada}
\item[0]{= completamente inadecuado en alzance y calidad}
\end{description}
 El proyecto final (A18) otorga un
m\'{a}ximo de 15 puntos, evaluados en los siguientes
rubros \begin{enumerate}[itemsep=-1pt]
\item{Variedad de t\'{e}cnicas de empleadas}
\item{Cobertura y validez de la experimentaci\'{o}n}
\item{Claridad y relevancia de los resultados}    
\item{Calidad de visualizaci\'{o}n cient\'{\i}fica}
\item{Calidad de redacci\'{o}n cient\'{\i}fica}
\end{enumerate}
 con la escala: \begin{description}[itemsep=0em]
\item[3]{= cumple con lo que se esperaba}
\item[2]{= d\'{e}bil en alcance y/o calidad}
\item[1]{= d\'{e}bil en ambos alcance y calidad}
\item[0]{= inadecuado en alzance y calidad}
\end{description}


\paragraph{Ponderaci\'{o}n espec\'{\i}fica}

\quad

\scalebox{0.85}{
  \begin{tabular}{|c|cccccccccccccccccc|c|}
    \hline
    \rotatebox{90}{\cellcolor{tableheader}{\bf Actividad\phantom{xx}}}
    & A1
    & A2
    & A3
    & A4
    & A5
    & A6
    & A7
    & A8
    & A9
    & A10
    & A11
    & A12
    & A13
    & A14
    & A15
    & A16
    & A17
    & A18
    & {\bf Total}
    \\
    \hline
    \rotatebox{90}{\cellcolor{tableheader}{\bf Ponderaci\'{o}n\phantom{xx}}}
    & 5\%
    & 5\%
    & 5\%
    & 5\% 
    & 5\%
    & 5\%
    & 5\%
    & 5\%
    & 5\%
    & 5\%
    & 5\%
    & 5\% 
    & 5\%
    & 5\%
    & 5\%
    & 5\%
    & 5\%
    & 15\%
    & 100\%		
    \\ \hline
  \end{tabular}}
  


\newpage

\section{Producto integrador de aprendizaje de la unidad:}

\subsection{Producto integrador de Aprendizaje:}

\quad

Portafolio en un repositorio digital p\'{u}blico que contiene los
reportes escritos y los c\'{o}digos de la implementaci\'{o}n de todas
las tareas y el proyecto integrador.


\section{Fuentes de apoyo y consulta:}
\subsection{Fuentes de apoyo y consulta}
\subsubsection{B\'{a}sicas}

\begin{itemize}[itemsep=-1pt]
  
\item R.\ {\sc Sedgewick} \& P. {\sc Flajolet}: {\em An                                                                                                          
    Introduction to the Analysis of Algorithms}. Addison Wesley, 512
  p\'{a}ginas, 1995. ISBN-13 978-0201400090.
  
\item C.H.\ {\sc Papadimitriou}: {\em Computational                                                                                                               
     Complexity}. Addison Wesley, 500 p\'{a}ginas, 1993. ISBN-13
   978-0201530827.

 \item D.L.\ {\sc Kreher} \& Douglas R.\ {\sc Stinton}: {\em                                                                                                          
     Combinatorial Algorithms --- Generation, Enumeration, and                                                                                                            
     Search}. CRC Press, 344 p\'{a}ginas, 1998. ISBN-13 978-0849339882.

 \item M.R.\ {\sc Garey} \& D.S.\ {\sc Johnson}: {\em                                                                                                            
     Computers and Intractability: A Guide to the Theory of                                                                                                               
     NP-Completeness}. W.\ H.\ Freeman, 340 p\'{a}ginas, 1979. ISBN-13:
   978-0716710455.   
\end{itemize}

\subsubsection{Complementarias}

\begin{itemize}[itemsep=-1pt]
  
\item T.H.\ {\sc Cormen}, C.E.\ {\sc Leiserson}, R.L.\ Rivest \&
  C.\ {\sc Stein}: {\em Introduction to Algorithms}. MIT Press,
  1184 p\'{a}ginas, segunda edici\'{o}n, 2001. ISBN-13
  978-0262032933.
  
\item R.\ {\sc Diestel}: {\em Graph Theory}. Graduate Texts in
  Mathematics, Volume 173. Springer-Verlag, 431 p\'{a}ginas,
  2005. ISBN 3-540-26183-4. Tercera edici\'{o}n.
  
\item M.\ {\sc Mitzenmacher} y Eli {\sc Upfal}: {\em
    Probability and Computing: Randomized Algorithms and
    Probabilistic Analysis.}  Cambridge University Press, 368
  p\'{a}ginas, 2005. ISBN-13 978-0521835404.
  
\item D.E.\ {\sc Knuth}: {\em The Art of Computer
    Programming}. Vol\'{u}menes 1--3. Addison Wesley, 896
  p\'{a}ginas, segunda edici\'{o}n, 1998. ISBN-13
  978-0201485417. Volumen 4: Generating All Trees--History of
  Combinatorial Generation. Addison Wesley, 128 p\'{a}ginas,
  2006. ISBN-13 978-0321335708.
  
\item D.\ {\sc Jungnickel}: {\em Graphs, Networks and
    Algorithms}. Springer, 611 p\'{a}ginas, segunda edici\'{o}n,
  2004. ISBN-13 978-3540219057.
  
\item R.L.\ {\sc Graham}, D.E.\ {\sc Knuth} \& O.\ {\sc
    Patashnik}: {\em Concrete Mathematics: A Foundation for
    Computer
    Science.} Addison Wesley, 672 p\'{a}ginas, segunda edici\'{o}n,
  1994. ISBN-13 978-0201558029.
  
\item N.\ {\sc Alon} \& J.H.\ {\sc Spencer}: {\em The
    Probabilistic Method}. Wiley Intersecience, 328 p\'{a}ginas,
  2000. ISBN-13 978-0471370468.
  
\item E.\ {\sc Aarts} \& J.K.\ {\sc Lenstra}: {\em Local
    Search
    in Combinatorial Optimization}. Princeton University
  Press, 536
  p\'{a}ginas, 2003. ISBN-13 978-0691115221.
  
\item A.V.\ {\sc Aho}, et al: {\em Compilers ---
    Principles,
    Techniques \& Tools}. Addison Wesley, 1040
  p\'{a}ginas, 2006. (Segunda
  edici\'{o}n.) ISBN-13 978-0321486813.
  
\end{itemize}
 
Art\'{\i}culos cient\'{\i}ficos especializados.

\label{final} % last page
%\newpage

%\pagestyle{plain}

%\vspace*{3cm}

%{\bf Autoriz\'{o}:} \coordinador

%\vspace*{2cm}

%  \begin{center}
%  {\sc Alere Flammam Veritatis}
  
%  Ciudad Universitaria, \today

%\vspace*{4cm}
  
%  \begin{tabular}{p{6cm}cp{7cm}}
%    \cline{1-1}
%    \cline{3-3}    
%    {\bf \coordinador}
% & \phantom{xxx} &{\bf Vo.\ Bo.\ \subdirector} \\
%    Coordinador Acad\'{e}mico &  &Subdirector de Estudios de Posgrado \\
%    Posgrado en Ingenier\'{\i}a de Sistemas & & Facultad de Ingenier\'{\i}a Mec\'{a}nica y El\'{e}ctrica
                                                                   
%  \end{tabular}
%\end{center}

\end{document}
