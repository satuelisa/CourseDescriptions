\documentclass[10 pt]{article}
\usepackage{wallpaper}
\usepackage[spanish, mexico]{babel}  
\usepackage{color,colortbl}
\usepackage[T1]{fontenc}
\usepackage{fancyhdr} % before geometry
\usepackage[letterpaper,left=18mm,right=18mm,headheight=5mm,headsep=40mm,top=52mm,bottom=22mm]{geometry}
\usepackage{graphicx}
\usepackage[latin1]{inputenc}
\usepackage{hyperref}
\usepackage{graphics}
\usepackage{varwidth}
\usepackage{tikz}
\usetikzlibrary{shapes,arrows}
\tikzstyle{elem} = [draw, rectangle, thick, minimum height=2em,
minimum width=2em, execute at begin node={\begin{varwidth}{14em}},
   execute at end node={\end{varwidth}}]
\tikzstyle{line} = [draw, thick, -stealth, shorten >=0pt]
\usepackage{multicol}
\usepackage{wrapfig}
\usepackage{enumitem}
\renewcommand*\familydefault{\sfdefault} 
\renewcommand{\thesection}{\Roman{section}} 
\usepackage{remreset}
\makeatletter
\@removefromreset{subsection}{section}
\renewcommand\thesubsection{\arabic{subsection}}
\makeatother
\usepackage{titlesec}
\ULCornerWallPaper{1}{logos.pdf}
\setlength{\parindent}{1em}
\setlength{\parskip}{2pt}
\usepackage{tcolorbox}
\definecolor{headerframe}{RGB}{177,178,177} % #b1b2b1
\definecolor{headercontent}{RGB}{239,239,239} % #efefef
\definecolor{tableheader}{RGB}{185,208,238}% #b9d0ee
\tcbset{mystyle/.style={
  breakable,
  enhanced,
  roundcorner = 30pt,
  outer arc = 4pt,
  arc = 4pt,
  colframe = headerframe,
  colback = headercontent,
  fonttitle=\sffamily
  }
}
\usepackage{amssymb}
\newcommand{\yes}{\makebox[0pt][l]{$\square$}{\raisebox{0.1\height}{$\times$}}}
\newcommand{\no}{\makebox[0pt][l]{$\square$}{\raisebox{0.1\height}{\phantom{$\times$}}}}
\usepackage{enumitem}
\titleformat{\section}{\normalfont\large\bfseries}{\thesection.}{4pt}{}
\titleformat{\subsection}[runin]{\normalfont\bfseries}{\thesubsection.}{4pt}{}


\newcommand{\narturo}{095012}
\newcommand{\nelisa}{096633}
\newcommand{\ncesar}{092038}
\newcommand{\nangy}{102662}
\newcommand{\nvincent}{102947}
\newcommand{\nada}{060581}
\newcommand{\niris}{103743}
\newcommand{\nsara}{100546}
\newcommand{\nroger}{090969}
\newcommand{\nigor}{093179}
\newcommand{\nromeo}{100959}
\newcommand{\nferny}{095808}
\newcommand{\arturo}{Dr.\ Jos\'{e} Arturo Berrones Santos}
\newcommand{\elisa}{Dra.\ Satu Elisa Schaeffer}
\newcommand{\cesar}{Dr.\ C\'{e}sar Emilio Villarreal Rodr\'{\i}guez}
\newcommand{\angy}{Dra.\ Mar\'{\i}a Ang\'{e}lica Salazar Aguilar}
\newcommand{\vincent}{Dr.\ Vincent Boyer}
\newcommand{\ada}{Dra.\ Ada Margarita \'{A}lvarez Socarr\'{a}s}
\newcommand{\iris}{Dra.\ Iris Abril Mart\'{\i}nez Salazar}
\newcommand{\sara}{Dra.\ Sara Ver\'{o}nica S\'{a}nchez Rodr\'{\i}guez}
\newcommand{\roger}{Dr.\ Roger Zirahu\'{e}n R\'{\i}os Mercado}
\newcommand{\igor}{Dr.\ Igor Litvinchev}
\newcommand{\romeo}{Dr.\ Romeo S\'{a}nchez Nigenda}
\newcommand{\ferny}{Dr.\ Fernando L\'{o}pez Irarragorri}
\newcommand{\simon}{Dr.\ Sim\'{o}n Mart\'{\i}nez Mart\'{\i}nez} 

\newcommand{\subdirector}{\simon}
\newcommand{\coordinador}{\cesar}


\begin{document}
\fancyhf{}
\rhead{IT-8-SPG-02-R03}
\lfoot{Revisión: 1 \\
  Vigente a partir del: 01 de agosto del 2016}

\begin{tcolorbox}
  \begin{center}

    {\bf \UANL}

    \medskip

    {\bf \fime}

    \medskip
    
    {\bf PE} \underline{\bf \doctorado}

    \medskip

    \underline{\PA}

  \end{center}
\end{tcolorbox}

\section{Datos de Identificaci\'{o}n de la Unidad de Aprendizaje:}
\subsection{Clave y nombre de la Unidad de Aprendizaje:} \fbox{{\tivd} Tesis 4}
\subsection{Frecuencia semanal:} horas de trabajo presencial \fbox{4}
\subsection{Horas de trabajo extra aula por semana:} \fbox{5}
\subsection{Modalidad:} \yes~Escolarizada \no~No escolarizada \no~Mixto
\subsection{Per\'{\i}odo acad\'{e}mico:} \yes~Semestral
\no~Tetramestral \no~Modular
\subsection{LGAC:} \underline{\odsi}
\subsection{Ubicaci\'{o}n semestral:} \fbox{4}
\subsection{\'{A}rea curricular:} \underline{\pi}
\subsection{Cr\'{e}ditos:} \fbox{6}
\subsection{Requisito:} \fbox{Tesis 3}
\subsection{Fecha de elaboraci\'{o}n:} \fbox{20/01/2010}
\subsection{Fecha de la \'{u}ltima actualizaci\'{o}n:} \fbox{10/06/2021}
\subsection{Responsable(s) del dise\~{n}o:}
\fbox{\parbox{9cm}{\begin{itemize}[label={}]
\item {\nroger~\roger}
\item {\nelisa~\elisa}
\end{itemize}}}
\newpage

\section{Presentaci\'{o}n:}

Se apoya al participante en la preparaci\'{o}n y estructuraci\'{o}n
de su trabajo de {\em tesis de doctorado}.

\section{Prop\'{o}sito(s):}

Se brinda apoyo y gu\'{\i}a sobre los aspectos fundamentales del
desarrollo de un trabajo de tesis.

\section{Competencias del perfil de egreso:}

\subsection{Competencias del perfil de egreso}

P3) Establecer comunicaci\'{o}n con los disGntos sectores de la
sociedad a fin de establecer proyectos estrat\'{e}gicos en las
distintas disciplinas de la ingenier\'{\i}a de sistemas y crear la
cultura de la creaci\'{o}n de riqueza basada en el conocimiento.


\subsection{Competencias generales a que se vincula la Unidad de Aprendizaje:}

La unidad se vincula con las siguientes competencias generales:

\phantom{space}
\begin{tabular}{|p{12cm}|p{30mm}|}
  \hline
  \cellcolor{tableheader}{\em Declaraci\'{o}n de la competencia general vinculada a la unidad
  de aprendizaje}
  & \cellcolor{tableheader}{\em Evidencia} \\ \hline
C7) Elabora propuestas acad\'{e}micas y profesionales inter, multi y
transdisciplinarias de acuerdo a las mejores pr\'{a}cticas mundiales del
\'{a}rea cient\'{\i}fica en la que trabaja para fomentar y consolidar el
trabajo colaborativo.
 & Redacci\'{o}n de la tesis, bit\'{a}cora semanal de actividades \\ \hline
C11) Practica los valores promovidos por la UANL: verdad, equidad,
honestidad, libertad, solidaridad, respeto a la vida y a los dem\'{a}s,
respeto a la naturaleza, integridad, \'{e}tica profesional, justicia y
responsabilidad, en su \'{a}mbito personal y profesional para contribuir a
construir una sociedad sostenible.
 & Redacci\'{o}n de la tesis, bit\'{a}cora semanal de actividades \\ \hline
C12) Construye propuestas innovadoras basadas en la comprensi\'{o}n
hol\'{\i}stica de la realidad incluyendo los diferentes campos cient\'{\i}ficos
para contribuir a superar los retos del ambiente global
interdependiente.
 & Redacci\'{o}n de la tesis, bit\'{a}cora semanal de actividades \\ \hline
C13) Asume el liderazgo que le ha otorgado el dominio de las ciencias,
comprometido con las necesidades sociales y profesionales para
promover el cambio social pertinente.
 & Redacci\'{o}n de la tesis, bit\'{a}cora semanal de actividades \\ \hline
\end{tabular}

\newpage
    
\subsection{Competencias espec\'{\i}ficas y nivel de dominio a que se vincula la unidad de aprendizaje:}

La unidad se vincula con las siguientes competencias espec\'{\i}ficas:

\phantom{space}
\begin{tabular}{|p{28mm}|p{3mm}|p{3mm}|p{3mm}|p{3mm}|p{28mm}|p{20mm}|p{24mm}|p{20mm}|}
\hline
\cellcolor{tableheader}{{\em Competencia Espec\'{\i}fica}}
& \rotatebox{90}{\cellcolor{tableheader}{Nivel I Inicial}}
& \rotatebox{90}{\cellcolor{tableheader}{Evidencia}}
& \rotatebox{90}{\cellcolor{tableheader}{Nivel II B\'{a}sico}}
& \rotatebox{90}{\cellcolor{tableheader}{Evidencia}}
& \rotatebox{90}{\cellcolor{tableheader}{Nivel III Aut\'{o}nomo}}
& \rotatebox{90}{\cellcolor{tableheader}{Evidencia}}
& \rotatebox{90}{\cellcolor{tableheader}{Nivel IV Estrat\'{e}gico\phantom{xxx}}}
& \rotatebox{90}{\cellcolor{tableheader}{Evidencia}}
\\ \hline


E3) Comunicar efectivamente los resultados obtenidos mediante la
ingenier\'{\i}a de sistemas, tanto con pares de las distintas disciplinas
acad\'{e}micas, as\'{\i} como con los diferentes sectores de la sociedad para
la generaci\'{o}n de bienestar y riqueza en base a la innovaci\'{o}n.

& & & &
& Comunica efectivamente trabajo original de investigaci\'{o}n en foros o publicaciones cientificas y tecnol\'{o}gicas.
& Redacci\'{o}n de la tesis, bit\'{a}cora semanal de actividades
& Establece junto con distintos sectores en la academia, la industria o
la sociedad en general, proyectos innovadores de car\'{a}cter estrat\'{e}gico.

& Redacci\'{o}n de la tesis, bit\'{a}cora semanal de actividades
\input{end_aut_est}

\section{Representaci\'{o}n gr\'{a}fica:}

\input{diag-tesis-doc.tex}

\newpage

\section{Estructuraci\'{o}n en cap\'{\i}tulos, etapas o fases de la unidad de
  aprendizaje:}

\subsection{Desarrollo de las fases de la Unidad de Aprendizaje:}

\section{Estructuraci\'{o}n en cap\'{\i}tulos, etapas o fases de la unidad de
  aprendizaje:}

\subsection{Desarrollo de las fases de la Unidad de Aprendizaje}

\quad

Orientaci\'{o}n al alumno para proseguir el tema de tesis, donde
deber\'{a}n seguirse el problema a tratar, objetivos perseguidos,
justificaci\'{o}n, planteamiento de hip\'{o}tesis y metodolog\'{\i}a a
aplicar.  Presentaci\'{o}n peri\'{o}dica del avance del trabajo para
su revisi\'{o}n, orientaci\'{o}n y validaci\'{o}n del grado de
avance.

La sesiones son de cuatro horas cada una y son veinte semanas en
total.

\paragraph{Unidades tem\'{a}ticas}

\quad

\begin{description}[itemsep=-2pt]
\item[UT1] Planteamiento del problema (2 semanas)
\item[UT2] Contextualizaci\'{o}n del trabajo (4 semanas)
\item[UT3] Descripci\'{o}n de la soluci\'{o}n propuesta (3 semanas)
\item[UT4] Evaluaci\'{o}n de la soluci\'{o}n propuesta (4 semanas)
\item[UT5] Elementos formales de un trabajo de tesis (5 semanas)
\end{description}

\paragraph{Temario semanal}

\quad


\begin{enumerate}[itemsep=-2pt]
\item Actualizaci\'{o}n del calendarizaci\'{o}n de actividades
\item UT1: Revisi\'{o}n de la redacci\'{o}n de la introducci\'{o}n
\item UT1: Revisi\'{o}n de la hip\'{o}tesis y de los objetivos
\item UT2: Revisi\'{o}n de la redacci\'{o}n de los antecedentes
\item UT2: Un primer borrador de la clasificaci\'{o}n de trabajos relacionados
\item UT2: Un primer borrador del an\'{a}lisis comparativo de trabajos relacionados
\item UT2: Un primer borrador de la identificaci\'{o}n del \'{a}rea de oportunidad
\item UT3: Principios avanzados para la redacci\'{o}n de la metodolog\'{\i}a
\item UT3: Principios avanzados para la redacci\'{o}n del dise\~{n}o de la soluci\'{o}n propuesta
\item UT3: Principios avanzados la redacci\'{o}n de la implementaci\'{o}n de la soluci\'{o}n propuesta
\item UT4: Principios para la redacci\'{o}n del dise\~{n}o experimental
\item UT4: Principios para la redacci\'{o}n de reportaje de resultados experimentales
\item UT4: Principios para la redacci\'{o}n del an\'{a}lisis de experimentos
\item UT4: Principios para la redacci\'{o}n de la discusi\'{o}n de los experimentos
\item UT5: Principios para la redacci\'{o}n de las conclusiones
\item UT5: Principios para la redacci\'{o}n de trabajo a futuro
\item UT5: Principios para la redacci\'{o}n del formato de la bibliograf\'{\i}a
\item UT5: Principios para la redacci\'{o}n del resumen
\item UT5: Principios para la redacci\'{o}n de los agradecimientos y la autobiograf\'{\i}a
\item Entrega de portafolio de evidencias
\end{enumerate}


\paragraph{Elementos de competencia}

\quad

\paragraph{Elementos de competencia:}

\quad

\begin{tabular}{|p{28mm}|p{30mm}|p{30mm}|p{30mm}|p{30mm}|}
  \hline
  \cellcolor{tableheader}Evidencias de aprendizaje 
  & \cellcolor{tableheader}Criterios de desempe\~{n}o
  & \cellcolor{tableheader}Actividades de aprendizaje
  & \cellcolor{tableheader}Contenidos
  & \cellcolor{tableheader}Recursos \\ \hline


Avance en la bit\'{a}cora. 

& 

Precisi\'{o}n del an\'{a}lisis de avance; nivel de detalle de la
planeaci\'{o}n de actividades pendientes.

&

Redacci\'{o}n de la bit\'{a}cora; actualizaciones en el documento de
tesis.

&

Estructuraci\'{o}n y estilo de un trabajo formal de tesis.

&

Material en la p\'{a}gina web de la unidad y la literatura citada;
paquete {\LaTeX} para redacci\'{o}n cient\'{\i}fica; libros de
texto. \\ \hline
  
\end{tabular}

No habr\'{a} examen. Son 19 avances semanales m\'{a}s la entrega del
portafolio, otorgando por m\'{a}ximo 5 puntos por semana con la
siguiente escala: \begin{description}[itemsep=-2pt]
\item[NP]{= tarea omitida}
\item[5]{= excede lo que se esperaba}
\item[4]{= cumple con lo que se esperaba}
\item[3]{= d\'{e}bil en alcance y/o calidad}
\item[2]{= d\'{e}bil en ambos alcance y calidad}
\item[1]{= sin contribuciones o m\'{e}ritos aunque fue entregada}
\item[0]{= completamente inadecuado en alzance y calidad}
\end{description}


{\bf Ponderaci\'{o}n espec\'{\i}fica}

\input{act20.tex}



\newpage

\section{Producto integrador de aprendizaje de la unidad:}
\subsection{Producto integrador de Aprendizaje:} 

\quad

El producto final se documenta como un {\em portafolio de evidencias},
en el cual el contenido espec\'{\i}fico de cada fase depende del tema
de tesis: cada estudiante mantendr\'{a} una bit\'{a}cora semanal de
avances, reportando lo discutido y avanzado en cada semana. Se incluye
al final del portafolio la tesis completa en su estado actual, con
firma de Visto Bueno (indicando la fecha) en la portada por su asesor
de tesis o todos coasesores en su caso.

Las bit\'{a}coras son individuales; se recomienda estudiar juntos y
discutir las soluciones, pero no se tolera ning\'{u}n tipo de plagio
en absoluto, ni de otros estudiantes ni de la red ni de libros ---
toda referencia bibliogr\'{a}fica tiene que ser apropiadamente citada.






Siendo {\bf cuarto} semestre, no se espera que un alumno concluya a
ning\'{u}n cap\'{\i}tulo en particular --- se visitan de forma
sistem\'{a}tica todos los elementos de un trabajo de tesis de
doctorado para analizar el avance actual y calendarizar de manera
estructurada el trabajo pendiente para semestres posteriores.


\section{Fuentes de apoyo y consulta:}

\subsection{Fuentes de apoyo y consulta}

\subsubsection{B\'{a}sicas}

\begin{itemize}[itemsep=0em]
  
\item{Secci\'{o}n de los Reglamentos de la UANL, FIME y el posgrado
    que se relacionan con la realizaci\'{o}n del proyecto de tesis.}
  
\item{S.\ {\sc Gimbel}: {\em Exploring the Scientific Method: Cases
      and Questions}, University of Chicago Press (abril 15, 2011),
    ISBN-10: 0226294838.}
  
\item{ H.L.\ {\sc \'{A}vila Baray}: {\em Introducci\'{o}n a la
      metodolog\'{\i}a de la investigaci\'{o}n}, 2006, Edici\'{o}n
    electr\'{o}nica. ISBN-10: 84-690-1999-6}

\item{Manuales diversos de redacci\'{o}n cient\'{\i}fica.}
  
\end{itemize}

\subsubsection{Complementarias}

Art\'{\i}culos cient\'{\i}ficos especializados relacionados a los
temas tratados, de preferencia publicados en revistas internacionales
indizados recientes.




 
\label{final} % last page

\newpage

\pagestyle{plain}

\vspace*{3cm}

{\bf Autoriz\'{o}:} \coordinador

\vspace*{2cm}

  \begin{center}
  {\sc Alere Flammam Veritatis}
  
  Ciudad Universitaria, \today

\vspace*{4cm}
  
  \begin{tabular}{p{6cm}cp{7cm}}
    \cline{1-1}
    \cline{3-3}    
    {\bf \coordinador} &
                                                            \phantom{xxx} &{\bf Vo.\ Bo.\ \subdirector} \\
    Coordinador Acad\'{e}mico &  &Subdirector de Estudios de Posgrado \\
    Posgrado en Ingenier\'{\i}a de Sistemas & & Facultad de Ingenier\'{\i}a Mec\'{a}nica y El\'{e}ctrica
                                                                   
  \end{tabular}
\end{center}


\label{final} % last page
\newpage

\pagestyle{plain}

\vspace*{3cm}

{\bf Autoriz\'{o}:} \coordinador

\vspace*{2cm}

  \begin{center}
  {\sc Alere Flammam Veritatis}
  
  Ciudad Universitaria, \today

\vspace*{4cm}
  
  \begin{tabular}{p{6cm}cp{7cm}}
    \cline{1-1}
    \cline{3-3}    
    {\bf \coordinador} &
                                                            \phantom{xxx} &{\bf Vo.\ Bo.\ \subdirector} \\
    Coordinador Acad\'{e}mico &  &Subdirector de Estudios de Posgrado \\
    Posgrado en Ingenier\'{\i}a de Sistemas & & Facultad de Ingenier\'{\i}a Mec\'{a}nica y El\'{e}ctrica
                                                                   
  \end{tabular}
\end{center}

\end{document}
