\section{Evaluaci\'{o}n integral de procesos y productos:}

Las tareas son individuales; se recomienda estudiar juntos y discutir
las soluciones, pero no se tolera ning\'{u}n tipo de plagio en
absoluto, ni de otros estudiantes ni de la red ni de libros --- toda
referencia bibliogr\'{a}fica tiene que ser apropiadamente citada. La
entrega se realiza por un repositorio p\'{u}blico que debe reflejar todas
las fases del trabajo. 

No habr\'{a} examen.  Son 17 tareas (A1--A17) que reportan avances
semanales de aplicaci\'{o}n de la lectura de la semana para el
proyecto del estudiante, otorgando por m\'{a}ximo 5 puntos por
tarea: \begin{description}[itemsep=-2pt]
\item[NP]{= tarea omitida}
\item[5]{= excede lo que se esperaba}
\item[4]{= cumple con lo que se esperaba}
\item[3]{= d\'{e}bil en alcance y/o calidad}
\item[2]{= d\'{e}bil en ambos alcance y calidad}
\item[1]{= sin contribuciones o m\'{e}ritos aunque fue entregada}
\item[0]{= completamente inadecuado en alzance y calidad}
\end{description}
 El proyecto final (A18) otorga un
m\'{a}ximo de 15 puntos, evaluados en los siguientes
rubros \begin{enumerate}[itemsep=0em]
\item{Variedad de t\'{e}cnicas de empleadas}
\item{Cobertura y validez de la experimentaci\'{o}n}
\item{Claridad y relevancia de los resultados}    
\item{Calidad de visualizaci\'{o}n cient\'{\i}fica}
\item{Calidad de redacci\'{o}n cient\'{\i}fica}
\end{enumerate}
 con la escala: \begin{description}[itemsep=0em]
\item[3]{= cumple con lo que se esperaba}
\item[2]{= d\'{e}bil en alcance y/o calidad}
\item[1]{= d\'{e}bil en ambos alcance y calidad}
\item[0]{= inadecuado en alzance y calidad}
\end{description}


\paragraph{Ponderaci\'{o}n espec\'{\i}fica}

\quad

\scalebox{0.85}{
  \begin{tabular}{|c|cccccccccccccccccc|c|}
    \hline
    \rotatebox{90}{\cellcolor{tableheader}{\bf Actividad\phantom{xx}}}
    & A1
    & A2
    & A3
    & A4
    & A5
    & A6
    & A7
    & A8
    & A9
    & A10
    & A11
    & A12
    & A13
    & A14
    & A15
    & A16
    & A17
    & A18
    & {\bf Total}
    \\
    \hline
    \rotatebox{90}{\cellcolor{tableheader}{\bf Ponderaci\'{o}n\phantom{xx}}}
    & 5\%
    & 5\%
    & 5\%
    & 5\% 
    & 5\%
    & 5\%
    & 5\%
    & 5\%
    & 5\%
    & 5\%
    & 5\%
    & 5\% 
    & 5\%
    & 5\%
    & 5\%
    & 5\%
    & 5\%
    & 15\%
    & 100\%		
    \\ \hline
  \end{tabular}}
  
